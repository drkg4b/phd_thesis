\documentclass[a4paper,10pt,twoside,notitlepage]{article}

\usepackage{layaureo}
\usepackage{enumitem}

% To apply this style to all the description environments. Needed to keep the
% item style on multiple lines. Revert with [style=same-line] for single
% environments.
\setlist[description]{style=nextline}

\title{Possible Defense Questions}

\begin{document}

\maketitle

\begin{description}
\item[Other ways to detect dark matter?]

\item[Other motivation for SUSY (CP violation, Sakharov, GUT)?]

\item[Why is important that pseudorapidity intervals are invariant?] The
  momentum of the parton is not known. Two partons colliding in opposite
  directions can have any range of momentum from 0 to the sum of the momenta
  they had so the (spread along z axis... elaborate)

\item[Can you explain all the terms in the luminosity formula?] The
  instantaneous luminosity depends on the beam parameters and is given by:
  \begin{equation}
    \label{eq:1}
    \mathcal{L} = \frac{N^2_b n_b f_{rev} \gamma}{4 \pi \epsilon_n \beta^*} F
  \end{equation}
  where $N_b$ is the number of particles per bunch, $n_b$ is the number of
  bunches per beam, $f_{rev}$ is the revolution frequency, $\gamma$ is the
  relativistic gamma factor, $\epsilon_n$ the normalized transverse beam
  emittance, the beta function is a measure of the transverse beam size and
  $\beta^*$ is the value of the beta function at the interaction point and $F$
  is the geometric reduction factor due to the crossing angle of the beams at
  the interaction point.

  The number of particles per bunch is squared because we have two bunches and
  this represents the possible interactions of a bunch containing $N$ protons
  with another bunch containing $N$ protons. The revolution frequency can be
  calculated dividing the speed of light by the LHC circumference obtaining
  $\sim$ 11~kHz (also remember the 40~MHz LHC frequency). The bunches don't
  collide head on in order to avoid collisions in the whole length of the bunch
  but at an angle which is called \emph{crossing angle}. The beam emittance
  $\epsilon$ is the normalized volume occupied by the beam in the phase
  space. The $\beta$ function is how much the beam is squeezed in the $xy$
  plane. The product $\beta^* \epsilon$ could be replaced by $\sigma_x
  \sigma_y$, the surface of the beam.

\item[How to measure absolute luminosity?] In order to measure the luminosity,
  we need to measure the number of protons in a bunch ($N_b$) and the area of
  the beam in the transverse plane ($\sigma_x \sigma_y$). To this end, from
  classical electromagnetism, if we have a charge in empty space $Q$ and a
  conductor plate not touching it, we have a \emph{mirror charge}. If $Q$ is
  moving then also the mirror charge in the conductor is moving giving thus a
  current. By calibrating this at LHC it is possible to measure it is possible
  to measure the number of protons in each bunch. To measure the
  $\sigma_x \sigma_y$ the so called \emph{van der Meer} scans are
  performed. These allow to measure the interaction rate when sweeping the beams
  transversely across each other.

\item[Why TileCal in barrel and LAr in forward regions?] LAr technology is
  radiation resistant while the scintillators tend to give less light after some
  time spent in a high radiation environment.

\item[Why LAr has an accordion shape, why we use LAr?]

\item[Strenghts and weaknesses of ATLAS and CMS]

\item[Why neutrinos cannot be dark matter candidates?] Can deduct the kinematic
  of particles from temperature. Neturino masses are small and thus they have a
  lot of kinetic energy (from the big bang), from simulations it can be seen
  that the gravitational pull is not enough to bound them in galaxies thus we
  need something heavier than neutrinos.

\item[How can we generate so much MC events to simulate for all the data we
  collect?] We generate selected processes with smaller cross sections and not
  the whole possibles pp collision outcomes.

\item[Why some group chose a set of working points (JVT)] Its a compromise
  between efficiency and purity of samples. Generally more than one working
  point is provided to accomodate the need of all the different analysis.

\item[The difference between different MC generators?] They have different
  hadronization models (empirical because qcd is not perturbative anymore at the
  energy scale when quarks and gluons hadronize).

\item[Why do we use renorm and fact scale times 0.5 and 2 for systematics?] The
  renormalization and factorization scale are not physical parameters and are
  entirely due to the perturbativity we use to calculate cross section at the
  theory level. The choise of these variations to estimate the systematics is
  totally arbitrary. This choice is done by theorists also and we adopt it in
  order to be able to compare the results. Moreover a variation of 50\% (0.5)
  and 100\% (2) is a very big variation and in this regard this choice is
  conservative.

\item[What is a PV and a secondary vertex?]

\item[What cosmological measurements have been made that suggest that the
  content of visible matter in the universe is just 5\%?]

\item[Give an example of homogeneous calorimeter, what are the benefits?]


  - How does SCT measure the PV and secondary vertex? Is what they really do
  directly or is done indirectly? How?

  - Why there is no excess in the high met at 13 TeV? Is it because at 8 TeV was
  just a fluctuation? Is it because of the increased energy?

  - Why do you think you have contributed enough for a PhD?

  - What do your different models solve and not solve, with respect to the SM
  problems you have formulated in chapter 1? How do your limits play into that?

  - Why do you quote neutrino masses for the flavor eigenstates?

  - How do you give masses to the neutrinos?

  - Is there more physics predicted from the ADD model? Do we expect other
  particles?

  - Is the diphoton excess compatible with a graviton model? With your graviton
  model?

  - In the CR fit is the agreement between data and MC good also before fit? Why
  don't we rely directly on the MC cross sections?

  - Why don't we use pure theoretical cross section but need the data driven
  method? What are the limitations of the different MC generators? Why aren't
  they perfect or good enough?

  - What is the order of the V + jets simulation, LO, NLO? Up to how many
  partons?

  - Why do we chose this normalization scheme for the transfer factors?

  - How is the jet response function derived?

  - Why the inverted $\delta \phi$ min cut can make a multi--jet control region?

  - How are we sure that the events selected in the way described in the NCB
  section come from BIB?

  - What is the jet energy scale and how the resolution contributes to the
  systematic uncertainties?

  - Why are the expected limits getting a bit worse for higher mass gaps (25
  GeV)?

  - What is a prompt electron?

  - Is the same likelihood used for all electrons eta and ET but with different
  cuts in different eta and ET or are you actually using a different likelihood
  function in the different bins? It seems to me the later would be better.

  - What is a local track segment?

  - How is the pile-up correction for jets performed?

  - How do you classify the energy deposits as electromagnetic or hadronic?

  - How is the tight criterion for jets defined?

  - Why most systematic uncertainties cancel out with the data driven method?

  - Why are we even doing monophotons when monojet always provides better
  sensitivity?

  - Why do the DM plots look the way they do? Eg. why do the sensitivity go to
  zero at low wimp mass, and decrease again for high wimp masses? Do all these
  experiments make the same assumptions?

  - What is an axial-vector mediator?

  - Does SUSY solve naturalness at the current limits?

  - What is naturalness good for? Seeing that it motivates most of your
  searches. Has naturalness ever motivated a discovery before?

  - Can you get the proper relic density with neutralinos? With UED? With
  gravitinos?

  - Is the graviton interaction really an effective interaction? The Feynman
  diagrams don’t look that way.

  - What is a general-purpose detector? Can it make coffee? (Here I think the
  answer is yes, but you would probably not want to drink it)

  - How do the new LVPS affect other things? Such as for example the timing?

  - Is the non-iterative OF a lot worse in terms of performance than the
  iterative?

  - Why can you not create b or c quarks in hadronization?

  - Why does our 8 TeV trigger plateau earlier than the 13 TeV trigger?

  - How do we actually measure the efficiency of a trigger?

  - Why did we increase the number of jets from the 7 TeV analysis?

  - When is it better to use one region and not a multi-region fit?

  - How do we determine how many events to generate?

  - The Pythia settings - are they tested on SM physics so we know what to
  expcect for radiation?

  - What is the pros and cons for the two different ways of estimating the top
  systematics?

  - What are the pros and cons of BLUE vs the Likelihood metod?

  - Your control regions in the 8 TeV analysis clearly perform worse at high
  MET. Why should we then trust the background estimate at high MET? And do you
  understand why you had this problem at 8 TeV and not at 13 TeV?

  At high met we have a single inclusive bin, for this reason the k-factor is
  unique and not tailored for each bin.

  %% From Olle's PhD defence

  - What happens if the energy deposit is less than 2 (like $1.8~\sigma$), does
  this cell gets clustered as well?

  - Is there other physical correlation to noise other than the vicinity to LVPS
  that was observed (fig 4.4)?

  - Do we still use the double Gaussian fit?

  - Why and how can I trust the truncation methos? All the events are used
  during the estimation and efficiency studies, how can you throw them out
  aftewards without affecting the estimation done before?

  It is a recursive procedure, we weight down (not throw away) events above the
  excluded value of md and recalculate the limit until the uncertainty of the
  truncated and not truncated limits are within 1%. As can be seen from the shat
  distribution (suppressed and not suppressed) the limits are set with not
  weighted events.

  - The requirement of MET from 8 TeV to 13 TeV and also the efficiency changed,
  was because the trigger got better?

  - The number of jets is an inclusive selection or an exclusie one? (do we
  always look for less than 4 jets or we require 1 or 2 or 3 jets?)

  - Why was the $Z \to ee$ region dropped in favor of the $Z \to \nu \nu$?

  - Why in the 13 TeV the pile up weight is not used?

  - Why is the $\Gamma$ so different from $m_0$ in table 8.1?

  - How do you see the evolution of this search in the next three years? What
  can be done to improve the mesurement?

  - What if ATLAS finds something in a region already excluded by a direct
  detection experiment like Ice Cube, how would you reconcile the results?

  the results depend on the model so it might be that the experiments are
  probing different models.

  - How do you tell apart all the different signals that we are probing with the
  monojet signature in case of an excess?

  %% From my thesis:

  - Why not group u and s quarks for example?

  The three familes of quarks grouped in the standard way provide an almost
  diagonal CKM matrix thus making it easier to use in calculations.

  - Draw a Feynman diagram of the $\tau \to \pi^+ \pi^+ \pi^-$ process

  - Where do the tau meson in cosmic rays come from?

  - What is a spinor?

  - Why are $P_L$ and $P_R$ called chiral operators? What is a chiral operator?

  - What is the Plack scale? And the Planck mass?

  - Why is the mass splitting importatnt to mention?

  because the cross section is enhanced by the number of gravitons.

  - If the graviton has a lifetime what are its decays?

  - Why do we talk of graviton mass?

  Because there is an extra momentum term in the extra dimensions which is
  modeled as a mass.

  - What are the cosmological motivation that exclude ADD n = 2?

  Some supernovae energy would disappear in the extra dimensions.

  - Is there a t-channel production for the graviton?

  - Why does the truncation not matter much?

  Need to calculate $\mu$ for the reduction in number of events given by the
  truncation.

  - Regarding the W -> l nu processes where l = mu or an electron, why not just
  lower the electron requirements to veto more of these events?

  Jets sometimes can fake electrons, these kind of fake are likely to be
  reconstructed as loose electrons thus lowering the electron quality on the
  veto could lead to rejecting signal events where a jet is mistakenly
  reconstructed as an electron.

  - Why not using a Z -> ee control region?

  It is not beneficial. It was tried but the systematic uncertainties are very
  small and thus this kind of region doesn't help reducing the systematics in
  other regions. This is because the Z -> ee process is very well measured and
  precise in ATLAS.

  - Comment on the choice of md used in the generation of the samples

  Values close to the one excluded in run 1 were chosen in order to simplify the
  comparison.

  - What are the jet quality requirements? What purpose do they serve? How much
  signal might be lost due to these cuts?

  - Why do you have a b-jet veto?

  - How is the electron energy scale determined?

  - Why do you need a systematic for the electron resolution?

  - About the matching: so how do you deal with the fact that your ADD signal is
  at the lowest order?

  - what is “soft radiation evolution”?

  Theory Chapter ============== - How was the electron discovered?

  - How did the idea of the photon came along?

  - p19 L3 “Big Classification”?

  - Main difference b/w bosons and fermions?

  - what is the principle of gauge theories ?

  - SU(2)xU(1) what does this symmetry operate upon?

  - What is gauge invariance?

  - Why does gauge invariance imply the gauge bosons to be massless? Are all the
  gauge

  - In what sense are the electromag. and weak interactions unify?

  - What are the possible color charges carried by a quark? by an anti-quark? by
  a gluon?

  - p20 L2 manages!

  - About lepton numbers Le, Lmu, Ltau: are they conserved?

  - How do we know there are 3 types of neutrinos? Couldn’t they just be all the
  same?

  - How does a W boson decay? Z boson? Top?

  - How does a b quark decay?

  - How does a c quark decay?

  - How does a s quark decay?

  - Can a c quark decay into a u quark?

  - What is the baryon number of an antineutron? of a pion? of a Kaon?

  - How could you detect a graviton as a particle experimentally?

  - Table 2.1 why do we have these quantum numbers?

  - Table 2.1 why is T different from lepton$_R$ and lepton$_L$?
\end{description}
\end{document}
%%% Local Variables:
%%% mode: latex
%%% TeX-master: t
%%% End:
