Den stora hadron kollideraren (Large Hadron Collider eller LHC) vid CERN är
världens kraftfullaste partikelacceleratoranläggning. Den krockar protoner med
protoner och även tunga joner. Sedan 2015 har LHCs proton-proton kollisionerna
en mass-centrum-energi som uppgår till 13 TeV. ATLAS är en av fyra stora
experiment som registrerar LHCs kollisioner.

Tile Calorimeter är den centrala delen av ATLAS hadron kalorimetern. Den mäter
upp enstaka hadroner, hadron kvastar, tau partiklar och den totala
rörelsemängden. För att noggrant kunna mäta dessa partiklar krävs en exakt model
av det elektroniska bruset i kalorimetern. Arbetetet för att mäta och studera
bruset samt beräkna relaterade kalibraringskonstanter som används för
precisionsfysik i ATLAS presenteras.

Avhandlingen presenterar resultat från två olika sökningar efter nya fenomen
m.h.a ATLAS experimentets data. Det första arbetet är en sökning efter
produktion av supersymmeriska kvarkar, så kallade skvarkar, som sönderfaller
till vanliga kvarkar och den supersymmetriska partikeln
$\tilde{\chi}_{\, 1}^{\, 0}$. $\tilde{\chi}_{\, 1}^{\, 0}$ skulle kunna utgöra
universums mörk materia. Det andra arbetet är en sökning efter stora extra
rumsdimensioner i Arkani-Hamed, Dimopoulos, Dvali (ADD) modellen. Båda arbeten
använder kollisioner som uppvisar stor odetekterad rörelsemängd och en
högenergetisk hadron kvast i motsatt riktning.

Sökningen efter skvarkar använder data från 2015, sammanlagt 3.2~$\ifb$, och
sökningen efter extra rumsdimensioner använder data från 2015 och 2016,
sammanlagt 36.1~$\ifb$. I båda fallen stämmer det experimentella utfallet
överens med bakgrundsberäkningarna. Därmed utesluts supersymmetriska modeller
med skvarkar lättare än 608 GeV (för en masskillnad mot
$\tilde{\chi}_{\, 1}^{\, 0}$ ner till 5 GeV). För modeller med extra
rumsdimensioner, nya gränser sätts för den effektiva Planck massan $\md$ i ADD
modeller up till 7.7~TeV för två extra rumsdimensioner och upp till 4.8~TeV för
modeller med sex extra rumsdimensioner.
%%% Local Variables:
%%% mode: latex
%%% TeX-master: "search_for_DM_LED_with_ATLAS"
%%% End:
