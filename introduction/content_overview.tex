An overview of the theoretical framework of the Standard Model is given in
\cref{cha:theoretical-overview}, it describes how the different elementary
particles discovered so far are organized and the main ideas that lead to the
current formulation of the Standard Model. \cref{cha:beyond-stand-model}
presents the hierarchy and dark matter problems and introduces supersymmetry and
large extra dimensions. The \gls{lhc} and the \gls{atlas} experiment are
introduced in \cref{cha:exper-appar} where the different sub-systems that make
up the \gls{atlas} detector are described. Studies on the stability of the
electronic noise in the hadronic calorimeter and work to derive noise thresholds
for the reprocessing and used in hadronic jet reconstruction in 2011 data are
described in \cref{cha:noise-studies-with}. In order to search for physics
beyond the standard model it is necessary to define criteria based on measured
quantities to identify for example electrons, muons, jets, missing energy and
displaced vertices from $b$-quark decays in an \gls{atlas} collision. The
criteria for identifying these particles and objects are described in some
detail in \cref{cha:phys-objects-reconst} expanding in particular on some
aspects that are specific to the searches presented in this thesis. The
methodology and results for a search for supersymmetry in a compressed scenario
with an en energetic jet and large missing energy with the 2015 data are
presented in \cref{cha:monojet-signature}. The results of the search for large
extra dimensions in the \gls{add} model scenario with the combined 2015 + 2016
data are presented in \cref{cha:2016-monoj-analys}. Finally
\cref{cha:conclusions} presents an overall conclusion of this thesis.
%%% Local Variables:
%%% mode: latex
%%% TeX-master: "../search_for_DM_LED_with_ATLAS"
%%% End:
