An overview of the theoretical framework of the Standard Model is given in
\cref{cha:theoretical-overview}, it describes how the different particles
discovered so far are organized and the main ideas that lead to the current
formulation of the Standard Model. \cref{cha:beyond-stand-model} presents the
hierarchy and dark matter problems and introduces supersymmetry and large extra
dimensions. The large hadron collider and the \gls{atlas} experiment are
introduced in \cref{cha:exper-appar} where the different sub-systems that are
part of the \gls{atlas} detector are described. Studies on the stability of the
electronic noise in the hadronic calorimeter used for the reprocessing of the
2011 data are described in
\cref{cha:noise-studies-with}. \cref{cha:phys-objects-reconst} contains details
of the physical objects used for the analyses presented in this thesis. The
results of the search for supersymmetry in a compressed scenario with an en
energetic jet and large missing energy with the 2015 data are presented in
\cref{cha:monojet-signature} and the results of the search for large extra
dimensions in the \gls{add} model scenario with the combined 2015 + 2016 data
are presented in \cref{cha:2016-monoj-analys}. Finally \cref{cha:conclusions}
summarizes the results of the electronic noise studies and of both searches.
%%% Local Variables:
%%% mode: latex
%%% TeX-master: "../search_for_DM_LED_with_ATLAS"
%%% End:
