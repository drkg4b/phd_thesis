The Standard Model of particle physics is the theory used to describe the
elementary constituents of matter and their interactions. Through the years is
has been tested by many experiments and despite its success it cannot explain,
among other problems, the so called hierarchy and dark matter problems described
in \cref{cha:beyond-stand-model}. Supersymmetry is an extension of the Standard
Model that could solve these issues by introducing new particles. The lightest
of these particles, the so called neutralino (and denoted by
$\widetilde{\chi}_{\, 1}^{\, 0}$), in the context of a minimal supersymmetric
model, could be produced in squark pair production with $\squarkprod$ and,
lacking electromagnetic and strong interaction~\cite{MSSMIntro}, escape
detection. With an energy in the center of mass of $\sqrt{s} = 13$~TeV, the
\gls{lhc} could be able to produce such kind of particles, the ATLAS detector
could be able to infer their presence by the momentum unbalance they would
create. This thesis presents the result of the search for compressed
supersymmetric squark--neutralino signal with the ATLAS detector in the
3.2~$\ifb$ delivered in 2015 in an experimental signature with jets and large
missing transverse momentum in the final state.
%%% Local Variables:
%%% mode: latex
%%% TeX-master: "../search_for_DM_LED_with_ATLAS"
%%% End:
