The \gls{sm} of particle physics is the theory used to describe the elementary
constituents of matter and their interactions and through the years is has been
tested by many experiments~\cite{SMTests}. Despite its success it cannot explain
for instance the so called hierarchy and dark matter problems (see
\cref{cha:beyond-stand-model} for more details). Supersymmetry is an extension
of the Standard Model that solves the hierarchy problem introducing a
supersymmetric partner to each \gls{sm} particle canceling in this way the
contribution to the quantum correction to the Higgs mass of the Standard Model
particles that make it close to the Planck scale. The lightest of these
particles is called \emph{neutralino} (and denoted by
$\widetilde{\chi}_{\, 1}^{\, 0}$) and being the lightest stable particle which
interacts weakly with normal matter provides a Dark Matter candidate. \gls{led}
in the \gls{add} model scenario sets the Planck scale at the order of the weak
scale formally eliminating the distinction between the two energy scales and
introduces large extra spatial dimensions in which only the \emph{graviton} (the
hypothesized particle that mediates the gravitational interaction) is allowed to
propagate in order to recover the strength of the gravitational interaction.

In the context of a minimal supersymmetric model the neutralino could be
produced in squark pair production with $\squarkprod$ and, lacking
electromagnetic and strong interaction~\cite{MSSMIntro}, escape
detection. Similarly gravitons could be produced in association with jets and
experiencing only the gravitational interaction, leave undetected. The \gls{lhc}
is a proton-proton and heavy ion collider designed to operate at a center of
mass energy of $\sqrt{s} = 13$~TeV located at \gls{cern}. The \gls{lhc} could be
able to produce such kind of particles and the two main general purpose
detectors \gls{atlas} and \gls{cms} could be able to infer their presence by the
momentum unbalance they would create. This thesis presents the result of the
search for compressed supersymmetric squark-neutralino signal with the ATLAS
detector in the 3.2~$\ifb$ delivered in 2015 and for large extra dimensions in
the \gls{add} model scenario using 36.1~$\ifb$ collected by \gls{atlas} in 2015
and 2016 in an experimental signature with jets and large missing transverse
momentum in the final state.
%%% Local Variables:
%%% mode: latex
%%% TeX-master: "../search_for_DM_LED_with_ATLAS"
%%% End:
