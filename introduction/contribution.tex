My contribution to the ATLAS experiment started in early 2013 by studying the
electronic noise in the hadronic calorimeter as part of my work to become a
signing author of the \gls{atlas} collaboration. The Tile Calorimeter is
designed to measure jets, tau particles, missing momentum and for the energy
reconstruction of hadrons, thus having an up-to-date description of the noise in
the detector is important for most physics analysis in ATLAS\@. For the
reprocessing of the 2011 data I developed a set of python scripts to study the
noise constants variations over several calibration runs. Based on these scripts
I carried out new studies of noise evolution some of which are described in
\cref{cha:noise-studies-with}. I presented the performance of the \gls{atlas}
Tile calorimeter in a poster session at the ``XXVII International Symposium on
Lepton Photon Interactions at High Energies (2015)'' and appear in the
proceedings of the conference on the ``Proceedings of
Science''~\cite{TileCalPerformanceBertoli}.

I later started analyzing the ATLAS data joining the monojet team in the effort
of trying to answer relevant Standard Model open questions. During Run~I my
contribution was limited to the study of the parton distribution function
systematic uncertainties associated to the cross section and acceptance for
extra dimension models.

In Run~II the \gls{atlas} software framework was radically changed along with
the data format thus the Stockholm analysis code had to be largely rewritten. I
am the main author of this software that allows to produce end-user last
analysis stage ROOT~\cite{CERNROOT} files on the \gls{cern} computing grid that
are small enough to be imported to the department computers for final stage
analysis and statistical interpretation. In the 2015 Run~II analysis an upper
cut on the number of jets was introduced, the efficiency of such a cut was
studied using a multivariate method based on the TMVA~\cite{TMVA} package and
the results, being in agreement with other studies performed in the group,
implemented in the analysis. I performed for the first time studies of the
sensitivity of the monojet analysis to the compressed supersymmetric
squark-neutralino model and demonstrated the sensitivity of the analysis to this
channel. Finally I have performed the analysis of the 2015 data and derived
myself the exclusion limits for the compressed squark-neutralino model. For the
2016 Run~II analysis I used the combined 2015 + 2016 data collected by the
\gls{atlas} detector to study the large extra dimension in the \gls{add} model
scenario, I calculated the theoretical systematic uncertainties affecting the
cross section and acceptance for this signal and implemented them in the fit. I
then calculated the exclusion limits on the effective Planck scale for five
large extra spatial dimensions. The results presented in this thesis led to the
following publications:
\begin{enumerate}[A -]
\item \emph{{Search for new phenomena in final stages with an energetic jet and
      large missing transverse momentum in $pp$ collisions at $\sqrt{s} = 13$
      TeV using the ATLAS detector}} given in Ref.~\cite{MonoJetPaper}.
\item \emph{{Search for dark matter and other new phenomena in events with an
      energetic jet and large missing transverse momentum using the ATLAS
      detector}} given in Ref.~\cite{MonoJetPaper2016}. This conference note is
  being converted to a paper and it is in the final stages of approval.
\end{enumerate}
As mentioned earlier I also contributed to the calculations of the systematic
uncertainties related to the parton distribution functions for the \gls{add}
large extra dimension model that was part of the monojet paper:
\begin{enumerate}[C -]
\item \emph{Search for new phenomena with the monojet and missing transverse
    momentum signature using the ATLAS detector in proton–proton
    collisions} given in Ref.~\cite{RunIPaper}.
\end{enumerate}
Finally, as stated in the beginning of this section, I am the author of the
\gls{atlas} Tile calorimeter performance poster for the 2015 Lepton Photon
conference that resulted in the following conference proceeding:
\begin{enumerate}[D -]
\item \emph{{Performance of the ATLAS Tile Calorimeter}} given in
  Ref.~\cite{TileCalPerformanceBertoli}.
\end{enumerate}

The work presented in this thesis is based on previous results already reported
in my licentiate thesis with the title \emph{Search for Supersymmetry in
  Monojet Final States with the ATLAS Experiment} and for this reason some of
the chapters in this thesis are freely taken from the licentiate one. In
particular \cref{cha:theoretical-overview} has been largely revisited,
\cref{cha:beyond-stand-model} has been extended, \cref{cha:exper-appar} is taken
from the licentiate and some typo have been corrected, a section has been added to
\cref{cha:noise-studies-with}, \cref{cha:phys-objects-reconst} has been reviewed
to include the changes added to the analysis during the search for large extra
dimensions, \cref{cha:monojet-signature} has been revisited and corrected,
\cref{cha:2016-monoj-analys} only appears in this thesis and
\cref{cha:conclusions} has been updated to summarize the results of
\cref{cha:2016-monoj-analys}.
%%% Local Variables:
%%% mode: latex
%%% TeX-master: "../search_for_DM_LED_with_ATLAS"
%%% End:
