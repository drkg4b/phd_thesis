My contribution to the ATLAS experiment started in early 2013 by studying the
electronic noise in the hadronic calorimeter as part of my work to become a
signing author of the \gls{atlas} collaboration. The Tile Calorimeter is
designed to measure jets, tau particles, missing momentum and for the energy
reconstruction of hadrons, thus having an up-to-date description of the noise in
the detector is important for most physics analysis in ATLAS\@. For the
reprocessing of the 2011 data I developed a set of python scripts to study the
noise constants variations over several calibration runs. Some of the software I
produced is still in use today. Based on these scripts I carried out new studies
of noise evolution some of which are described in \cref{cha:noise-studies-with}
and produced noise calibration constants used in the reprocessing of the
\gls{atlas} data which is later used in precision measurements performed with
the full \gls{atlas} run~1 (2010 - 2013) dataset. I presented the performance of
the \gls{atlas} Tile calorimeter on behalf of the \gls{atlas} collaboration in a
poster session at the ``XXVII International Symposium on Lepton Photon
Interactions at High Energies (2015)'', this work appears in the proceedings of
the conference in Ref.~\cite{TileCalPerformanceBertoli}.

I later started analyzing the ATLAS data joining the monojet team in the effort
of trying to answer relevant Standard Model open questions. During run~1 my
contribution was limited to the study of the parton distribution function
systematic uncertainties associated to the cross section and acceptance for
extra dimension models.

For \gls{lhc} run~2 (2015 - 2018) the \gls{atlas} software framework was
radically changed along with the formats for data analysis thus the Stockholm
analysis code had to be largely rewritten. I am the main author of this software
that applies calibrations to electrons, muons and jets, calculates systematic
uncertainties and selects the events required for data analysis. This software
also produces end-user last analysis stage ROOT~\cite{CERNROOT} files on the
\gls{lhc} computing grid that are small enough to be imported to the department
computers for final stage analysis and statistical interpretation. I performed
for the first time studies that demonstrated the sensitivity of the monojet
analysis to the compressed supersymmetric squark-neutralino model. In order to
ensure orthogonality with other \gls{susy} searches it was decided to apply an
upper cut on the number of jets. I performed optimization studies of such cut
along with signal region definition using a multivariate method based on the
TMVA~\cite{TMVA} package in order to retain sensitivity to the \gls{susy}
compressed. I contributed to the development of the limit setting framework and
to its optimization that allowed to reduce the turn around time from many days
down to few hours allowing to quickly check and debug the analysis
results. Finally I have performed the analysis of the 2015 data and derived
myself the exclusion limits for the compressed squark-neutralino model and
studied the theoretical systematics for this model.

The second analysis presented in this thesis is based on the combined 2015 +
2016 data collected by the \gls{atlas} detector and is used to derive new
experimental constraints on the large extra dimension in the \gls{add} model
scenario. For this thesis I calculated the theoretical systematic uncertainties
affecting the cross section and acceptance for this signal and implemented them
in the fit. I then calculated the exclusion limits on the effective Planck scale
for five large extra spatial dimensions. The results presented in this thesis
led to the following publications:
\begin{enumerate}[A -]
\item \emph{{Search for new phenomena in final states with an energetic jet and
      large missing transverse momentum in $pp$ collisions at $\sqrt{s} = 13$
      TeV using the ATLAS detector}} given in Ref.~\cite{MonoJetPaper}.
\item \emph{{Search for dark matter and other new phenomena in events with an
      energetic jet and large missing transverse momentum using the ATLAS
      detector}} given in Ref.~\cite{MonoJetPaper2016}. This conference note is
  being converted to a paper and is in the final stages of \gls{atlas} approval.
\end{enumerate}
As mentioned earlier I also contributed to the calculations of the systematic
uncertainties related to the parton distribution functions for the \gls{add}
large extra dimension model that was part of the run~1 monojet paper:
\begin{enumerate}[C -]
\item \emph{Search for new phenomena with the monojet and missing transverse
    momentum signature using the ATLAS detector in proton–proton
    collisions} given in Ref.~\cite{RunIPaper}.
\end{enumerate}
Finally, as stated in the beginning of this section, I am the author of the
\gls{atlas} Tile calorimeter performance poster for the 2015 Lepton Photon
conference that resulted in the following conference proceeding:
\begin{enumerate}[D -]
\item \emph{{Performance of the ATLAS Tile Calorimeter}} given in
  Ref.~\cite{TileCalPerformanceBertoli}.
\end{enumerate}

The text of this thesis relies in part on my licentiate thesis with the title
\emph{Search for Supersymmetry in Monojet Final States with the ATLAS
  Experiment}~\cite{MyLicentiate} and for this reason some of the chapters in
this thesis are taken or adapted from the licentiate thesis. In particular
\cref{cha:theoretical-overview} has been largely revisited,
\cref{cha:beyond-stand-model} has been extended, \cref{cha:exper-appar} is taken
to a large extent from the licentiate, a section has been added to
\cref{cha:noise-studies-with}. \cref{cha:phys-objects-reconst} has been revised
and extended to include aspects relevant to the analysis of the 2015 and 2016
data for the search for large extra dimensions, \cref{cha:monojet-signature} has
been revisited and expanded, \cref{cha:2016-monoj-analys} is entirely new.
%%% Local Variables:
%%% mode: latex
%%% TeX-master: "../search_for_DM_LED_with_ATLAS"
%%% End:
