My contribution to the ATLAS experiment started in early 2013 by studying the
electronic noise in the hadronic calorimeter. The Tile Calorimeter is designed
to measure jets, tau particles, missing momentum and for the energy
reconstruction of hadrons, thus having an up--to--date description of the noise
in the detector is important for most physics analysis in ATLAS\@. During the
reprocessing of the 2011 data I developed a set of python scripts in order to
study the noise constants variation over several calibration runs. Based on
these scripts I carried out new studies of noise evolution some of which are
described in \cref{cha:noise-studies-with}. I presented the performance of the
ATLAS TileCal were presented in a poster at the ``XXVII International Symposium
on Lepton Photon Interactions at High Energies (2015)'' and appear in the
proceedings of the conference on the ``Proceedings of Science''.

I later started analyzing the ATLAS data joining the monojet team in the effort
of trying to answer relevant Standard Model open questions. During Run~I my
contribution was limited to the study of some of the systematic uncertainties
associated to the cross section for extra dimension models. In Run~II the ATLAS
software framework was radically changed along with the data format thus the
Stockholm analysis code had to be largely rewritten. I am the main contributor
to this software that allows to produce end--user last analysis stage ROOT files
from the ATLAS centrally produced data sets.

In the Run~II analysis an upper cut on the number of jets was introduced, the
efficiency of such a cut was studied using a multivariate method based on the
TMVA package and the results, being in agreement with other studies performed in
the group, implemented in the analysis. I performed for the first time studies
of the sensitivity of the monojet analysis to the compressed supersymmetric
squark--neturalino model and demonstrated the sensitivity of the analysis to
this channel. Finally I have performed the analysis of the 2015 data and
computed myself the exclusion limits for the compressed squark--neutralino
model. The results of this study are summarized in the paper in
ref.~\cite{MonoJetPaper}.
%%% Local Variables:
%%% mode: latex
%%% TeX-master: "../search_for_DM_LED_with_ATLAS"
%%% End:
