TileCal is the central hadronic calorimeter of the ATLAS experiment covering the
$|\eta| < 1.7$ region. It is designed for energy measurement of hadrons, jets,
tau particles and also contributes to the measurement of missing transverse
energy (see Section~\ref{sec:miss-transv-energy}). TileCal is a scintillator
steel non compensating sampling calorimeter, the scintillation light produced in
the tiles is transmitted by \glspl{wsf} to \glspl{pmt}. The analog signals from
the PMTs are amplified, shaped and digitized by sampling the signal every
25~ns. The TileCal front end electronics read out the signals produced by about
10000 channels measuring energies ranging from 30~MeV to 2~TeV. The readout
system is responsible for reconstructing the data in real time. The digitized
signals are reconstructed with the Optimal Filtering algorithm (see
Section~\ref{sec:optimal-filtering}), which computes for each channel the signal
amplitude, time and quality factor at the required high rate.

TileCal is designed as one \gls{lb} covering the $|\eta| < 1.0$ range and two
\gls{eb} in the $0.8 < |\eta| < 1.7$ range. The barrels are further divided,
according to their geometrical position on the $z$-axis, in partitions called
EBA, LBA, EBC and LBC (see Section~\ref{sec:coordinate-system}). Each partition
consists of 64 independent wedges (see Figure~\ref{fig:tile_mod}) called
\emph{modules} assembled in azhimut ($\phi$). The LBA and EBA partitions are
shown in Figure~\ref{fig:tile_cells}.

\begin{figure}[!h]
  \centering
    \includegraphics[width=.5\linewidth]{tile_module}
    \caption{Cut away showing an individual TileCal module along with the
      optical read out and design of a TileCal module~\cite{TileModule}.}
    \label{fig:tile_mod}
\end{figure}

Between the LB and the EB there is a 600~mm gap needed for the ID and the LAr
cables, electronics and services. Part of the gap contains the \gls{itc}, a
detector designed to maximize the active material while leaving enough space for
services and cables. The ITC is an extension of the EB and it occupies the 0.8 <
$|\eta|$ < 1.6 region. The combined 0.8 < $|\eta|$ < 1.0 part is called
\emph{plug} and in the 1.0 < $|\eta|$ < 1.6 region, for space reasons, the ITC
is not interleaved with an absorber and is only composed of scintillator
material. The scintillators between 1.0 < $|\eta|$ < 1.2 are called \emph{gap
  scintillators}, while those between 1.2 < $|\eta|$ < 1.6 are called
\emph{crack scintillators}. The plug and the gap scintillators mainly provide
hadronic shower sampling while the crack scintillator, which extends to the
region between the barrel and the end-cap cryostats, samples the electromagnetic
shower in a region where the normal sampling is impossible due to the dead
material of the cryostat walls and the ID cables.

TileCal is also divided in longitudinal layers, the A, BC and D layers as shown
in \cref{fig:tile_cells}. The two innermost layers have a
$\Delta \eta \times \Delta \phi$ segmentation of $0.1 \times 0.1$ while in the
outermost, the segmentation is $0.1 \times 0.2$. Each layer is logically divided
into \emph{cells} (also shown in \cref{fig:tile_cells}) by grouping together in
the same PMT the fibers coming from different scintillator tiles belonging to
the same radial depth. The gap/crack scintillators are also called E layer
cells.

The energy resolution for jets in ATLAS is:
\begin{equation}
  \label{eq:65}
  \frac{\sigma_E}{E} = \frac{50\%}{\sqrt{E}} \oplus 3\%
\end{equation}
for $|\eta| < 3$. The 3\% constant term becomes dominant for high energy hadrons
where an increase in energy resolution is expected~\cite{TileCal}.

\begin{figure}[!h]
  \centering
    \includegraphics[width=\linewidth]{tile_cells}
    \caption{Schematic view of the TileCal layer and cell structure in a plane
      containing the beam axis $z$~\cite{TileCalPub}.}
    \label{fig:tile_cells}
\end{figure}
%%% Local Variables:
%%% mode: latex
%%% TeX-master: "../search_for_DM_LED_with_ATLAS"
%%% End:
