Hadrons lose energy through strong interaction with the calorimeter
material. The strong interaction is responsible for the production of energetic
secondary hadrons with momenta typically at the GeV scale and nuclear reactions
such as excitation or nucleon spallation in which neutrons and protons are
released from the nuclei with a characteristic energy at the MeV scale.

These energetic hadrons are protons, neutrons and pions. On average, 1/3 of the
pion produced are neutral pions which decay to photons
($\pi \rightarrow \gamma \gamma$). The photons produced this way will initiate
an electromagnetic shower as described in Section~\ref{sec:electr-show}
transferring energy from the hadronic part to the electromagnetic shower inside
the hadronic shower. The electromagnetic component of the shower does not
contribute any more to hadronic processes. The nucleons released by excitation
or nuclear spallation, require an energy equal to their binding energy to be
released and are not recorded as a contribution to the calorimeter signal thus
producing a form of \emph{invisible energy}. Some detectors can compensate for
the loss of invisible energy, these are called \emph{compensating
  calorimeters}~\cite{Calorimetry}.
%%% Local Variables:
%%% mode: latex
%%% TeX-master: "../search_for_DM_LED_with_ATLAS"
%%% End:
