High energy electrons and photons lose energy mainly by \emph{radiation} and
\emph{conversion} respectively. When electrons with energies greater than
$\sim 10$~MeV interact with the electromagnetic field of the absorber nuclei
\emph{Bremsstrahlung} can occur. High energy photons produce mostly
electron--positron pairs.

Electrons and photons with a sufficient amount of energy interacting with an
absorber, produce secondary photons through Bremsstrahlung or secondary
electrons and positrons by pair production. These secondary particles will
produce more particles through the same mechanisms giving rise to a shower of
particles with progressively lower energies. When the energy loss in the shower
is dominated by ionization and thermal excitation of the active material atoms,
the number of particles in the shower stops growing and there is no further
shower development. The above process goes on until the energy of the electrons
falls below a critical energy, $\epsilon$, where ionization and excitation
becomes the dominant effects~\cite{Calorimetry}.
%%% Local Variables:
%%% mode: latex
%%% TeX-master: "../search_for_DM_LED_with_ATLAS"
%%% End:
