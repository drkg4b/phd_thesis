The seven samples are used to reconstruct the amplitude of the pulse using the
\gls{of} method. The estimate of the amplitude is given by:
\begin{equation}
  \label{eq:80}
  \hat{A} = \sum_{i = 0}^7 a_i S_i
\end{equation}
where $S_i$ are the digitized samples expressed in ADC counts and $a_i$ are
computed weights that minimize the effect of the electronic noise on the
amplitude reconstruction. The procedure minimizes the variance of the amplitude
distribution. In order to make the amplitude reconstruction independent from
phase and signal baseline due to electronic noise (\emph{pedestal}), the
following constraints are used:
\begin{equation}
  \label{eq:81}
  \sum_{i = 0}^7 g_i a_i = 0
\end{equation}
\begin{equation}
  \label{eq:82}
  \sum_{i = 0}^7 g'_i a_i = 0
\end{equation}
\begin{equation}
  \label{eq:83}
  \sum_{i = 0}^7 a_i = 0
\end{equation}
where $g_i$ and $g'_i$ are the pulse shape function from the shaper and its
derivative~\cite{OptimalFilter}.
%%% Local Variables:
%%% mode: latex
%%% TeX-master: "../search_for_DM_LED_with_ATLAS"
%%% End:
