The Large Hadron Collider is the most powerful particle accelerator built to
date. The LHC is a proton-proton and heavy ion collider, in 2015 it operated at
an unprecedented center of mass energy of $\sqrt{s} = 13$~TeV. This thesis
presents the results of the search for physics beyond the Standard Model of
particle physics in a compressed supersymmetric squark-neutralino model. The
present work uses an experimental signature with a single high energetic jet and
large missing transverse energy, so called monojet signature. The search is
carried out using an integrated luminosity of 3.2~$\ifb$ recorded by the ATLAS
experiment in 2015. No significant excess compared to the Standard Model
prediction has been observed thus a 95\% CL limit has been set on the production
of squark pairs with the subsequent decay $\squarkprod$. Squark masses up to
608~GeV are excluded for $\msquark - \mneutralino = 5$~GeV significantly
improving earlier results.

The Tile Calorimeter is the ATLAS hadronic calorimeter covering the central
region of the detector. It is designed to measure hadrons, jets, tau particles
and missing energy. In order to accurately be able to properly reconstruct these
physical objects a careful description of the electronic noise is required. This
thesis presents the work done in updating, monitoring and studying the noise
calibration constants used in the processing of data and the identification of
hadronic jets. These studies showed an unexpected variation over time of the
cell noise and further investigation led to discover that the tile noise filter
was not behaving as expected in some situations in approximately 5\% of the
detector cells.
%%% Local Variables:
%%% mode: latex
%%% TeX-master: "search_for_DM_LED_with_ATLAS"
%%% End:
