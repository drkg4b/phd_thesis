The \gls{lhc} is the most powerful particle accelerator built to date. It is a
proton-proton and heavy ion collider which in 2015 and 2016 operated at an
unprecedented center of mass energy of $\sqrt{s} = 13$~TeV. The Tile Calorimeter
is the ATLAS hadronic calorimeter covering the central region of the
detector. It is designed to measure hadrons, jets, tau particles and missing
energy. In order to accurately be able to properly reconstruct these physical
objects a careful description of the electronic noise is required. This thesis
presents the work done in updating, monitoring and studying the noise
calibration constants used in the processing and identification of hadronic jet
in the 2011 data.

The \gls{lhc} proton-proton data collected by the \gls{atlas} experiment at
13~TeV is used in this thesis to perform two searches for new phenomena: a
search for supersymmetric particles in the compressed supersymmetric
squark-neutralino model and a search for heavy gravitons in the large extra
dimension \gls{add} model. This work exploits an experimental signature with a
high energy hadronic jet and large missing transverse energy, often referred to
as monojet signature.

The search for supersymmetry presented in this thesis is based on the data
recorded by the \gls{atlas} experiment in 2015 which represents an integrated
luminosity of $3.2~\ifb$. No significant excess compared to the Standard Model
prediction is observed and exclusion limits are set on squark pair production
with the subsequent squark decay $\squarkprod$. The limits are set as function
of the neutralino mass and focus on the least experimentally constrained
situation with a small mass difference between the squark and the
neutralino. Squark masses up to 608~GeV are excluded for
$\msquark - \mneutralino = 5$~GeV. The second work presented in this thesis is a
search for heavy gravitons in the large extra dimension model using the full
2015 + 2016 dataset, representing an integrated luminosity of $36.1~\ifb$. A
good agreement between data and Standard Model prediction is observed and
exclusion limits are set on the effective Planck scale $\md$. Values of $\md$
lower than 7.7 and 4.8~TeV for two and six hypothesized large extra dimensions
are excluded. Both searches significantly improve earlier results.
%%% Local Variables:
%%% mode: latex
%%% TeX-master: "search_for_DM_LED_with_ATLAS"
%%% End:
