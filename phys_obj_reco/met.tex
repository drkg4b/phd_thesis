Due to the conservation of momentum and the fact that the proton bunches are
parallel to the $z$-axis, the sum of the momenta of the collision products in
the transverse plane should sum to zero. Any energy imbalance is known as
\gls{met}, it may indicate weakly interacting stable particles (neutrinos within
the SM, new particles in beyond SM models) or non reconstructed physical objects
that escape the detector acceptance like for example a muon that goes into a
hole of the muon system (e.g.\ supports for ATLAS) cannot be detected and give
rise to fake $\met$.  Physical objects that are fully reconstructed and
calibrated such as electrons, photons, hadronically decaying tau-leptons, jets
or muons are called \emph{hard objects} and are used to compute the missing
transverse momentum in an event~\cite{MET}. The $x$ and $y$ components of the
$\met$ can be written as:
\begin{equation}
  \label{eq:94}
  E_\mathrm{\, x(y)}^\mathrm{\, miss} = E_\mathrm{\, x(y)}^\mathrm{\, miss,\, e} +
  E_\mathrm{\, x(y)}^\mathrm{\, miss,\, \gamma} + E_\mathrm{\, x(y)}^\mathrm{\,
    miss,\, \tau} + E_\mathrm{\, x(y)}^\mathrm{\, miss,\, \text{jets}} +
  E_\mathrm{\, x(y)}^\mathrm{\, miss,\, \mu} + E_\mathrm{\, x(y)}^\mathrm{\, miss,\,
    \text{soft}}
\end{equation}
where the terms for jets, charged leptons and photons are the negative sum of
the momenta of the respective calibrated object while the \emph{soft term} is
reconstructed from the transverse momentum deposited in the detector that is not
already associated to hard objects. It may be reconstructed by means of
calorimeter-based methods, the so called \gls{cst}, or using track-based methods
known as \gls{tst}.

The CST is reconstructed using energy deposits in the calorimeters which are not
associated to hard objects, it arise from soft radiation accompanying the hard
scatter event and from underlying event activity. A downside of the CST is its
vulnerability to pile up.

The TST is built from tracks not associated to any hard object, tracks can be
associated to vertices and thus to a particular $pp$ collision, making this
method robust against pile-up. This method is, however, insensitive to soft
terms coming from neutral particles that do not leave a track in the ID, thus
the TST $\met$ is combined with calorimeter-based measurements for hard objects.

Due to its stability against pile-up, this analysis uses the TST $\met$ term.
Moreover, the muons are treated as invisible particles in the $\met$
reconstruction (i.e. $E_\mathrm{\, x(y)}^\mathrm{\, miss,\, \mu} = 0$).
%%% Local Variables:
%%% mode: latex
%%% TeX-master: "../search_for_DM_LED_with_ATLAS"
%%% End:
