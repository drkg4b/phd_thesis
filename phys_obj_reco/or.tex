During object reconstruction, it may happen that different algorithms identify
the same track and cluster as different types of particles, this results in a
duplicate object. In physics analyses a decision must be made on which
interpretation to give to the reconstructed object, this process is called
\gls{or}~\cite{Alison:1536507}.

In this analysis, an overlap removal is applied to electrons, muons and jets
that pass the baseline criteria and the following objects are removed:
\begin{enumerate}[A -]
\item Remove jet in case any pair of jet and electron satisfies
  $\Delta R(j, e) < 0.2$.
\item Remove electron in case any pair of jet and electron satisfies
  $0.2 < \Delta R(j, e) < 0.4$.
\item Remove muon in case any pair of muon and jet with at least 3 tracks
  satisfies $\Delta R(j, \mu) < 0.4$.
\item Remove jet if any pair of muon and jet with less than 3 tracks satisfies
  $\Delta~R(j, \mu)~<~0.4$.
\end{enumerate}
In the 2016 analysis the A condition is replaced by:
\begin{enumerate}[E -]
\item In case any pair of jet and electron or muon satisfy
  $\Delta R(j, e~\mathrm{or}~\mu ) < 0.2$, the $b$-tagging is adjusted to the 85\%
  efficiency working point and:
  \begin{itemize}
  \item The jet is not a $b$-jet: remove the jet and keep the electron or muon.
  \item The jet is a $b$-jet: since the jet is likely coming from a
    semi-leptonic $b$ decay, keep the jet and remove the electron or the muon.
  \end{itemize}
\end{enumerate}
%%% Local Variables:
%%% mode: latex
%%% TeX-master: "../search_for_DM_LED_with_ATLAS"
%%% End:
