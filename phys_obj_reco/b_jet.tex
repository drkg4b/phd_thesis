Jets coming from the hadronization of a $b$ quark are commonly referred to as
\emph{$b$-jets} and their identification as \emph{$b$-tagging}. A multivariate
discriminant, the MV2, is used in \gls{atlas} to discriminate $b$-jets from
light- ($u$, $d$, $s$ or gluon) and $c$-jets (jets coming from the hadronization
of the $c$ quark). Information from secondary vertexes, $b$-hadron decay chains
and a log likelihood ratio based on the probability of the tracks to originate
from given flavor quark are combined it the multivariate
discriminant~\cite{BTagging}. In the 2016 analysis the bottom quarks are
identified using the MV2c10 algorithm where the $b$-jets are treated as signal
and the background is a mixture of 7\% $c$ quarks and 93\% light quarks. The
presence of $c$ quarks as background in the training sample is low in order to
enhance the charm rejection. A working point that provides a 77\% $b$-jet
efficiency have been chosen.
% The information from two different classes of algorithms, the impact-parameter
% and the secondary vertex based ones, is combined in a multivariate discriminant,
% the MV2, in order to identify $b$-jets which are then used in physics analyses
% within \gls{atlas}~\cite{BTagging}. In the impact-parameter based algorithm uses
% ratios of $b$- and light-flavor jets hypotheses defined from the probability
% density functions of the signed impact parameter significance (defined as
% $d_0/\sigma_{d_0}$ for the transverse impact parameter and
% $z_0 \sin \theta/\sigma_{z_0 \sin \theta}$ for the longitudinal one) of the jet
% tracks which are then combined in a single log likelihood ratio
% discriminant. The sign of the significance of the impact parameter is positive
% if the transverse impact parameter $d_0$ is in front the \gls{pv} with respect
% to the jet direction. It is negative if it is behind~\cite{BTaggingIP}.
%%% Local Variables:
%%% mode: latex
%%% TeX-master: "../search_for_DM_LED_with_ATLAS"
%%% End:
