The cell noise described in \cref{sec:cell-noise} is used in the
\emph{topological clustering} algorithm~\cite{JetCluster}, for identification of
the real energy deposits in the calorimeters. The algorithm assumes that the
noise in all the calorimeter cells is normally distributed with significance
(the ratio between the deposited energy and the parameter $\sigma$ used to
describe the cell noise) expressed in units of Gaussian sigmas. In the
algorithm, cluster of cells called \emph{topoclusters}, are formed by comparing
the energy deposit in a cell for a significant incompatibility with a noise only
hypothesis. The algorithm starts by finding the \emph{seed cells} with
$E > 4 \sigma$ where $\sigma$ is the measured RMS of the energy distribution for
every cell in the pedestal run. The second step is to add to the seeds neighbor
cells that satisfy the $E > 2 \sigma$ condition. Finally an additional level of
cells with $E > 0$ is added to the perimeter of the cluster and the algorithm is
ended. At this point the splitting algorithm is applied to separate the
topoclusters based on the local energy maxima.
%%% Local Variables:
%%% mode: latex
%%% TeX-master: "../search_for_DM_LED_with_ATLAS"
%%% End:
