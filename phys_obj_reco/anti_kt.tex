The $\antikt$ algorithm is a sequential recombination algorithm. It defines two
distances $d_{ij}$ and $d_{iB}$. The distance $d_{ij}$ between the physical
objects $i$ and $j$ is defined as:
\begin{equation}
  \label{eq:80}
  d_{ij} = \min(k_{ti}^{-2}, k_{tj}^{-2}) \frac{\Delta_{ij}^2}{R^2}
\end{equation}
where $\Delta_{ij}^2 = (\eta_i - \eta_j)^2 + (\phi_i^2 - \phi_j^2)$ and $\eta_i$
is the rapidity, $\phi_i$ is the azimuthal angle, $k_{ti}$ is the transverse
momentum of the object $i$ and $R$ is the radius parameter that controls the
size of the jet. The distance $d_{iB}$ between the object $i$ and the beam (B)
defined as:
\begin{equation}
  \label{eq:81}
  d_{iB} = k_{ti}^{-2}.
\end{equation}
This distance is meant to distinguish between hard and soft terms. The algorithm
identifies the smallest of the two distances and, if it is $d_{ij}$, it
recombines the $i$ and $j$ objects while if it is $d_{iB}$, it calls $i$ a jet
and removes it from the list of inputs to the algorithm. The distances are
recalculated and the procedure reiterated until there are no more
objects~\cite{Antikt}.
%%% Local Variables:
%%% mode: latex
%%% TeX-master: "../search_for_DM_LED_with_ATLAS"
%%% End:
