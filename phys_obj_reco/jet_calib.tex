The \emph{electromagnetic scale} is the baseline signal scale of the ATLAS
calorimeters, it is established during test beam measurements with electrons and
it accounts correctly for the energy deposited by particle interacting
electromagnetically. The topoclusters introduced in \cref{sec:topocluster} are
evaluated at the electromagnetic scale and used as an input to the $\antikt$
algorithm outlined in \cref{sec:anti-k_t}. Due to detector effects such as the
non compensating nature of the ATLAS hadronic calorimeter (see
\cref{sec:hadronic-shower}), energy loss due to inactive regions in the detector
or particle showers not fully contained in the calorimeter, the energy of the
jets measured at the electromagnetic scale is lower than the true energy of the
original jet. These effects are calibrated using \gls{mc} simulations and a
correction, referred to as the \gls{jes}, is applied in order to recover the
correct energy scale of the jets~\cite{JESIntro}.

The basic jet calibration scheme that applies the \gls{jes} correction to the
\gls{em} scale is usually referred to as \gls{em} + \gls{jes} and uses $\pt$ and
$\eta$ dependent corrections derived from simulations. The correction in this
case is derived by matching jets measured in the calorimeter with truth jets
reconstructed in simulations, the ratio between the true jet energy and the
electromagnetic scale matching one is taken as the correction
factor~\cite{JESEMCalibration}. The \gls{lcw} briefly introduced in
\cref{sec:trigger-system} is another calibration method that clusters
topologically connected cells, classifying them as electromagnetic or hadronic
and deriving the energy corrections from single pion MC simulation and dedicated
studies to account for the detector effects. The jets calibrated with this
method are referred to as \gls{lcw} + \gls{jes}\@. The \gls{gcw} calibration
uses the fact that electromagnetic and hadronic showers leave a different energy
deposition in the calorimeter cells, with the electromagnetic shower more
compact than the hadronic one. The energy correction are then derived for each
calorimeter cell within the jet and minimizing the energy resolution. The
\gls{gs} method starts from \gls{em} + \gls{jes} calibrated jets and corrects
for the fluctuation in particle content of the hadronic shower using the
topology of the energy deposits. The corrections are applied in a way that
leaves unchanged the mean jet energy~\cite{JetCalib}.
%%% Local Variables:
%%% mode: latex
%%% TeX-master: "../search_for_DM_LED_with_ATLAS"
%%% End:
