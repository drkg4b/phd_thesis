In order to distinguish jets coming from $pp$ collisions from those from a
non--collision origin, two jet selection criteria, \emph{loose} and \emph{tight},
are available. The loose selection criteria, provides an efficiency for
selecting jets coming from $pp$ collisions above 99.5\% for $\pt > 20$~GeV, the
tight criteria can reject even further background jets~\cite{JetEff}.

In this analysis the jets are reconstructed using the $\antikt$ algorithm with
the radius parameter R = 0.4. The \emph{baseline jets} are required to have
$|\eta| < 2.8$ and to make sure they come from a HS vertex, they need to satisfy
any of the following:
\begin{itemize}
\item The $\pt > 50$~GeV.
\item They have $20 < \pt < 50$~GeV and $|\eta| > 2.4$.
\item They have $20 < \pt < 50$~GeV, $|\eta| < 2.4$ and JVT > 0.64.
\end{itemize}
Furthermore, events in which the jets, after the overlap removal is applied,
fail the loose selection criteria are disregarded. Finally the most energetic
jet in the event (the \emph{leading jet}) is required to pass the tight
selection criteria. The \emph{good jets} require an increased $\pt$ threshold of
30~GeV and at most 4 HS jet in the event.

\begin{table}[!th]
  \centering
  \begin{tabular}{ll}
    \toprule
    \multicolumn{2}{c}{Jet Definition} \\
    \midrule \midrule
    \textbf{Baseline jet} & \textbf{Good jet} \\
    \midrule
    R = 0.4 & \emph{baseline} \\
    $|\eta| < 2.8$ & at most 4 jets \\
    \emph{Loose} selection criteria & $\pt > 30$~GeV \\
    \emph{Tight} on the leading jet & \\
    passes the overlap removal & \\
    any of: \\
    \tabitem $\pt > 50$~GeV \\
    \tabitem $20 < \pt < 50$~GeV, $|\eta| > 2.4$ \\
    \tabitem $20 < \pt < 50$~GeV, $|\eta| < 2.4$, JVT > 0.64 & \\
    \bottomrule
  \end{tabular}
  \caption{Jet definition in the monojet analysis.}
  \label{tab:jet_def}
\end{table}
%%% Local Variables:
%%% mode: latex
%%% TeX-master: "../search_for_DM_LED_with_ATLAS"
%%% End:
