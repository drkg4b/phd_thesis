Object reconstruction is the process that associates the signal left in the
detector by charged particles to physical objects through a series of
algorithms. This analysis uses electrons, muons, jets and missing transverse
momentum ($\met$). Two types of electrons, muons and jets are also defined:
\emph{baseline} and \emph{good}, where the former one is used for removal of
overlapping objects and preselection while the latter for selecting the objects
used to define the signal and control regions. In the following a brief
introduction to the identification criteria of these objects is presented.
%%% Local Variables:
%%% mode: latex
%%% TeX-master: "../search_for_DM_LED_with_ATLAS"
%%% End:
