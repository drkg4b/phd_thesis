Muons are identified using different criteria from the information provided by
the ID and the MS leading to four different types of muons. The \gls{sa} muons
use only the MS information to reconstruct the muon's trajectory; the \gls{cb},
where the track is independently reconstructed in the ID and the MS and then
combined; the \gls{st} are identified as muons only if the track in the ID is,
after being extrapolated to the MS, associated to at least one local track
segment in the MDT or CSC chambers and finally the \gls{ct} where tracks in the
ID are associated to an energy deposit in the calorimeter compatible with a
minimum ionizing particle. CB candidates perform best in terms of muon purity
and momentum resolution.

Muons are identified using quality requirements specific to each of the four
type of muons aiming at rejecting those coming from pion and kaon decays and
guarantee a robust momentum measurement. The \emph{loose} identification
criteria maximize the reconstruction efficiency and provide good muon tracks;
the \emph{medium} criteria minimize the systematic uncertainties associated to
muon reconstruction and calibration; the \emph{tight} muons maximize the purity
of the sample and the \emph{high $\pt$} selection maximize the momentum
resolution for tracks with transverse momenta above 100~GeV~\cite{MUONS}.

This analysis uses the CB muons that pass the medium identification criteria,
moreover the \emph{baseline muons} are required to have $\pt > 10$~GeV and
$|\eta| < 2.5$, the $\met$ definition and in the lepton veto used to define the
signal and control regions. The \emph{good muons} are required to pass the
baseline selection criteria, moreover $d_0 / \sigma_{d0} < 3$,
$|z_0 \sin \theta| < 0.5$~mm. The good muons are used in the one muon and
di-muon control regions.

\begin{table}[!th]
  \centering
  \begin{tabular}{ll}
    \toprule
    \multicolumn{2}{c}{Muon Definition} \\
    \midrule \midrule
    \textbf{Baseline muon} & \textbf{Good muon} \\
    \midrule
    CB muon & \emph{baseline} \\
    \emph{Medium} id. criteria & $d_0 / \sigma_{d0} < 3$ \\
    $\pt > 10$~GeV & $|z_0 \sin \theta| < 0.5$~mm \\
    $|\eta| < 2.5$ & \\
    passes the overlap removal & \\
    \bottomrule
  \end{tabular}
  \caption{Muon definitions for the monojet analysis.}
  \label{tab:mu_def}
\end{table}
%%% Local Variables:
%%% mode: latex
%%% TeX-master: "../search_for_DM_LED_with_ATLAS"
%%% End:
