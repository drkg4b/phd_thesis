In \cref{sec:electro-weak-inter} we have generated massless gauge bosons for the
electroweak interaction, in fact no term such as $M^{2} B_{\mu}B^{\mu} / 2$
appear in the Lagrangian of \cref{eq:13}. These kind of terms are not gauge
invariant and lead to divergences that would prevent using the theory for making
any prediction. Nevertheless the weak gauge bosons $W^\pm$ and $Z^0$ must be
massive in order to account for the short range of the weak interaction. A gauge
invariant way to recover both the fermion and boson masses, is to spontaneously
break the local $\sutwol \times \uoney$ electroweak symmetry. Spontaneous
breaking of a symmetry occurs in different phenomena in nature, for example when
a bowl of water freezes the ice crystals form in a particular direction which
cannot be predicted and that breaks the full rotational symmetry that the system
originally had. In the case of a ferromagnet at high temperature the spin
orientations is not well defined and globally (at the macroscopic level) this
disorder gives a rotational symmetry since, for example, the value of the
magnetization vanish. If it is cooled down below a critical temperature it may
acquire a magnetization in two possible directions but there is no way to tell
in advance which one breaking in this way the rotational invariance of the
ferromagnet for global rotations. In \cref{sec:glob-symm-break} we will see the
case of the spontaneous breaking of a global gauge symmetry. For a more detailed
explanation of the mechanism see for example~\cite{martin:particle}.
%%% Local Variables:
%%% mode: latex
%%% TeX-master: "../search_for_DM_LED_with_ATLAS"
%%% End:
