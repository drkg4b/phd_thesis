In 1964 Englert and Brout~\cite{EnglertBroutPaper}, Guralnik, Hagen and
Kibble~\cite{GuralnikPaper} and Higgs~\cite{HiggsPaper} independently published
a paper where they introduced a way to give mass to the gauge bosons and to get
rid of the Goldstone fields. It involves the spontaneous breaking of
\emph{local} gauge symmetries. In order to see how the mechanism works, let us
consider the Lagrangian:
\begin{equation}
  \label{eq:33}
  \mathcal{L} =(\partial_{\mu} \phi)^{\dagger}(\partial^{\mu} \phi) -
  \mu^{2}\phi^{\dagger}\phi - \lambda (\phi^{\dagger}\phi)^{2},
\end{equation}
where $\phi$ is a doublet of complex scalar fields with four degrees of freedom:
\begin{equation}
  \label{eq:34}
  \phi \equiv \binom{\phi_{i}}{\phi_{j}} = \frac{1}{\sqrt{2}}
  \binom{\phi_{1} + i \phi_{2}}{\phi_{3} + i \phi_{4}}.
\end{equation}
The Lagrangian is invariant under global SU(2) phase transformations:
\begin{equation}
  \label{eq:32}
  \phi(x) \rightarrow \phi'(x) = e^{i \sum_{k = 1}^{3} \epsilon^{k} T^{k}}
  \phi(x),
\end{equation}
where $T^{k} = \tau^k/2$ are the generators and
$[T^{i},T^{j}] = i \epsilon^{ijk}T^{k}$ with $i,j,k = 1,2,3$ are the commutation
relations. To achieve SU(2) invariance of the Lagrangian for local
transformations, that is when the $\epsilon_k$ parameter becomes a function of
the space-time coordinates ($\epsilon_k \equiv \epsilon_k (x^\mu)$), we need to
introduce the covariant derivative:
\begin{equation}
  \label{eq:35}
  D_{\mu} = \partial_{\mu} + i g\,\tfrac{\vec \tau}{2} \cdot \vect{W}_{\mu}(x)
\end{equation}
where $\vect{W}_{\mu}(x) = \left( W^0, W^1, W^2 \right)$ are three gauge
fields. For simplicity we can consider infinitesimal transformations of the
fields:
\begin{equation}
  \label{eq:36}
  \phi(x) \rightarrow \phi'(x) \simeq \left( 1 + i \vec \epsilon\, (x) \cdot
    \tfrac{\vec \tau}{2} \right) \phi(x),
\end{equation}
for the gauge bosons transformations we have:
\begin{equation}
  \label{eq:37}
  \vect{W}_{\mu} (x) \rightarrow \vect{W}_{\mu}(x) -
  \frac{1}{g} \partial_{\mu} \vec \epsilon \, (x) - \vec \epsilon\,(x)
  \times \vect{W}_{\mu} (x).
\end{equation}
The gauge invariant expression of the Lagrangian is then:
\begin{equation}
  \label{eq:38}
  \mathcal{L} = \left( \partial_{\mu} \phi + i g\,\tfrac{\vec \tau}{2}
  \cdot \vect{W}_{\mu} \phi \right)^{\dagger} \left(\partial_{\mu} \phi + i
  g\,\tfrac{\vec \tau}{2} \cdot \vect{W}_{\mu} \right) - V(\phi) -
  \frac{1}{4} \vect{W}_{\mu\nu} \cdot \vect{W}^{\mu\nu}
\end{equation}
where the potential is given by:
\begin{equation}
  \label{Esq:40}
  V(\phi) = \mu^{2} \phi^{\dagger} \phi + \lambda(\phi^{\dagger} \phi)^{2}
\end{equation}
and in the kinetic term $\vect{W}_{\mu\nu} \cdot \vect{W}^{\mu\nu}/4$ we have
introduced:
\begin{equation}
  \label{eq:39}
  \vect{W}_{\mu\nu} = \partial_{\mu} \vect{W}_{\nu} - \partial_{\nu} \vect{W}_{\mu}
  - g\,\vect{W}_{\mu} \times \vect{W}_{\nu}.
\end{equation}
Also in this case there are two possibilities for the potential in \cref{eq:40},
if $\mu^2 > 0$ the Lagrangian of \cref{eq:38} describes four scalar particles,
the $\phi_i$ with $i = 1, \dots, 4$ of \cref{eq:34}, with mass $\mu$ that
interacts with the three massless gauge bosons $\vect{W}$. We are interested in
the second case where $\mu^{2} < 0$ and $\lambda > 0$, the minimum of the
potential is at:
\begin{equation}
  \label{eq:40}
  \phi^{\dagger}\,\phi = \frac{1}{2} \left(\phi_{1}^{2} + \phi_{2}^{2} +
    \phi_{3}^{2} + \phi_{4}^{2} \right) = - \frac{\mu^{2}}{2 \lambda}
\end{equation}
and to study the particle spectrum we have to expand the field $\phi(x)$ around
a minimum, we can choose one of them, for example:
\begin{equation}
  \label{eq:41}
  \phi_{0} \equiv \frac{1}{\sqrt{2}} \binom{0}{v}
\end{equation}
that is $\phi_{1} = \phi_{2} = \phi_{4} = 0$ and
$\phi_{3}^{2} = - \mu^{2}/\lambda \equiv v^{2}$. Also in this case any other
vacuum choice is related to $\phi_0$ by a global phase transformation and thus
$\phi_0$ is not invariant under SU(2) transformations and the symmetry is
spontaneously broken. Making a small perturbation around this particular minimum
we have:
\begin{equation}
  \label{eq:43}
  \phi (x) = \frac{1}{\sqrt{2}} \binom{0}{v + h(x)}
\end{equation}
where $h(x)$ is a scalar field called the \emph{Higgs field}. The fact that only
one field out of the initial four remains can be better understood by
parameterizing the fluctuations around the vacuum $\phi_0$ using four real
fields $\theta_i$ with $i = 1, 2, 3$ and $h$:
\begin{equation}
  \label{eq:42}
  \phi(x) = \frac{1}{\sqrt{2}} e^{i \frac{\vec{\tau} \cdot \vec{\theta}}{v}}
  \begin{pmatrix}
    0 \\
    v + h(x)
  \end{pmatrix}
\end{equation}
and expanding at the lowest order in $1/v$:
\begin{equation}
  \label{eq:154}
  \begin{aligned}[t]
    \phi(x) & = \frac{1}{\sqrt{2}}
    \begin{pmatrix}
      1 + i \theta_3/v & i(\theta_1 - i \theta_2)/v \\
      i(\theta_1 + i \theta_2)/v &1 - \theta_3/v
    \end{pmatrix}
    \begin{pmatrix}
      0 \\
      v + h(x)
    \end{pmatrix} \\
    & \sim \frac{1}{\sqrt{2}}
    \begin{pmatrix}
      \theta_2 + i \theta_1 \\
      v + h - i \theta_3
    \end{pmatrix}.
  \end{aligned}
\end{equation}
We see that these four fields are independent and fully parametrize the
perturbation of the vacuum $\phi_0$, $\theta_i$ with $i = 1, 2, 3$ are the
Goldstone bosons that can be absorbed with a local SU(2) gauge transformation of
the form given in \cref{eq:32} where the functions
$\epsilon^k \equiv \epsilon^k(x)$ are chosen accordingly, the $\theta_1$,
$\theta_2$ and $\theta_3$ fields disappear from the Lagrangian and the
parameterization of \cref{eq:43} is found.

To generate the masses for the three gauge bosons $W^1$, $W^2$ and $W^3$, we
substitute the vacuum expectation value $\phi_0$ in the Lagrangian of
\cref{eq:38}. For the mass term we have:
\begin{equation}
  \label{eq:44}
  \begin{aligned}[t]
    \MoveEqLeft \left| i g\, \tfrac{\vec \tau}{2} \cdot \vect{W}_{\mu} \phi_{0} \right|^2 =  \\
  &= \frac{g^{2}}{2}
  \left|
    \begin{pmatrix}
      W^{3} & W^{1} - i W^{2} \\
      W^{1} + i W^{2} & - W^{3}
    \end{pmatrix}
    \binom{0}{v} \right|^2 = \\
    &= \frac{g^{2}v^{2}}{8} \left( W^{1} + i W^{2} ,\; - W^{3} \right) \binom{W^{1} -
      i W^{2}}{- W^{3}} = \\
    &= \frac{g^{2} v^{2}}{8} \left[ \left(W_{\mu}^{1} \right)^{2} +
      \left(W_{\mu}^{2} \right)^{2} +
      \left(W_{\mu}^{3} \right)^{2}\right].
  \end{aligned}
\end{equation}
By comparing this result with a general mass term for a boson $M^2 W_\mu^2/2$ we
see that the three vector gauge bosons gain a mass $M = \frac{gv}{2}$. The
Lagrangian thus describes three massive vector bosons where the three spin
degrees of freedom are a consequence of the gauge fixing used to absorb the
Goldstone bosons and a massive scalar Higgs field with mass $m_H =
2v^2\lambda$. This mechanism can be used to generate the masses of the
electroweak gauge bosons preserving the renormalizability of the theory.
%%% Local Variables:
%%% mode: latex
%%% TeX-master: "../search_for_DM_LED_with_ATLAS"
%%% End:
