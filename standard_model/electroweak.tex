In 1930 Pauli suggested the introduction of a new particle in order to explain
the continuous spectrum of in $\beta$ decays~\cite{PauliNeutrino}. Since the
neutron was not discovered yet Pauli named his hypothesized particle that way as
it had to be electrically neutral in order to conserve charge. It was not until
Fermi proposed his theory for $\beta$ decay~\cite{FermiTheory, FermiTheoryIta}
that the new particle was accepted and named \emph{neutrino}. The Lagrangian in
Fermi's theory is of the current-current type and can be written as:
% We can now see how to find out the weak interaction symmetry group, to this end,
% let us start by writing out the \emph{Hamiltonian} for the weak interaction:
\begin{equation}
  \mathcal{L}_\mathrm{weak} = \frac{G_F}{\sqrt{2}} J_\mu^\dagger
  J^\mu
  \label{eq:150}
\end{equation}
where $G_F$ is the Fermi constant and the $J_\mu$ current is given by:
\begin{equation}
  \label{eq:151}
  J_\mu = \overbar{\psi} \gamma^\mu \psi,
\end{equation}
here $\psi$, $\overbar{\psi}$ are the spinors of the fermions involved in the
interaction and $\gamma_\mu$ ($\mu =$ 0, 1, 2, 3) are the Dirac matrices. In
1949 Powell~\cite{PowellTauMeson} discovered in cosmic rays two particles that
he called the tau and the theta mesons. The tau meson could decay by weak
interaction into three pions ($\tau^+ \rightarrow \pi^+ \pi^+ \pi^-$) and the
theta decays into two pions ($\theta^+ \rightarrow \pi^+ \pi^0$). The measured
lifetime and mass of the tau and the theta mesons turned out to be the
compatible within experimental uncertainties hinting that they could be the same
particle but in that case parity\footnote{Parity is a space-time symmetry in
  which $\vec{r} \rightarrow \pvec{r}' = - \vec{r}$ and $t \rightarrow t' = t$
  where $\vec{r}$ and $t$ are the position vector and time respectively thus
  $\vec{p} \rightarrow \pvec{p}' = - \vec{p}$,
  $\vec{L} = \vec{r} \times \vec{p} \rightarrow \pvec{L}' = \vec{L} \Rightarrow
  \vec{J} \rightarrow \pvec{J}' = \vec{J}$ where $\vec{p}$, $\vec{L}$ and
  $\vec{J}$ are the momentum, the angular momentum and the total angular
  momentum respectively. For the helicity we have also
  $\lambda = \vec{J} \cdot \vec{p} / |\vec{p}| \rightarrow \lambda' = -
  \lambda$.} conservation would be violated. This is known as the
$\tau - \theta$ puzzle. In 1950 Lee and Yang proposed that parity could be
violated in weak interactions~\cite{LeeYangParityViolation} which was
experimentally confirmed in 1956 by Wu~\cite{WuExperiment}. The tau and theta
mesons were identified as the \emph{kaon} which decays are both parity
conserving ($K^+ \rightarrow \pi^+ \pi^+ \pi^-$) and parity violating
($K^+ \rightarrow \pi^+ \pi^0$). Wu's experiment involved the $\beta$ decay of
Cobalt 60 atoms to Nickel 60. The Cobalt atoms were immersed in a magnetic field
in order to align the spin of the nucleus with it. The observable in the
experiment is the product between the spin of the electron and its momentum
($\vec{J} \cdot \vec{p}$). Reverting the magnetic field is the same as applying
a parity transformation and since the momentum transforms as a vector and
changes sign under space reflection ($\vec{p} \to - \vec{p}$) but the spin
transforms as an axial vector and remains unchanged ($\vec{J} \to \vec{J}$) the
product $\vec{J} \cdot \vec{p}$ is not parity conserving and asymmetry in the
distributions of the experiment and its reflected version were observed.

In 1958 Feynman and Gell-Mann proposed a parity violating extension of Fermi's
theory~\cite{VATheory} called the V-A (vector minus axial vector) theory which
introduces the \emph{chiral} operators:
\begin{equation}
  \label{eq:152}
  P_L = \frac{1}{2}(1 - \gamma^5); \quad P_R = \frac{1}{2}(1 + \gamma^5)
\end{equation}
where $\gamma_5 = i \gamma_0 \gamma_1 \gamma_2 \gamma_3$. The chiral operators
projects the left and right chirality when applied on a spinor $\psi$. The weak
current can be re-written as:
\begin{equation}
  \begin{split}
    J_\mu & \equiv J_\mu^{(+)} = \overbar{\psi}_{\nu_e} \gamma_\mu \frac{1}{2}
    \left(1 - \gamma_5 \right) \psi_e \\
    J_\mu^\dagger & \equiv J_\mu^{(-)} = \overbar{\psi}_e \gamma_\mu \frac{1}{2}
    (1 - \gamma_5) \psi_{\nu_e},
  \end{split}
  \label{eq:153}
\end{equation}
here $J_\mu^{(+)}$ and $J_\mu^{(-)}$ are the charged weak currents and are a
combination of vector and axial vector currents. The $\frac{1}{2}(1 - \gamma^5)$
operator only selects left-handed fermions (or right-handed anti-fermions) thus
only these components participate in the weak interaction. Indicating for
simplicity $\psi_e (x)$ as $e (x)$ and $\psi_{\nu_e} (x)$ as $\nu_e (x)$,
\cref{eq:153} can be written as:
\begin{equation}
  \begin{aligned}
    J_\mu^{(+)} = \overbar{\nu}_{e_L} \gamma_\mu e_L \\
    J_\mu^{(-)} = \overbar{e}_L \gamma_\mu \nu_{e_L}
  \end{aligned}
  \label{eq:177}
\end{equation}
where the subscript $L$ indicates that the $\frac{1}{2}(1 - \gamma^5)$ have been
applied and only the left-handed fermions are selected.

One problem of the V-A theory is that the total cross section for the inelastic
neutrino scattering process $\nu_\mu + e^- \rightarrow \nu_e + \mu^-$ given
by~\cite{VAUnitaryViolationBook}:
\begin{equation}
  \label{eq:155}
  \sigma = \frac{G_F^2}{\pi}s
\end{equation}
where $s = 4E^2$ is the center of mass energy of the $\nu_\mu-e$ system grows
indefinitely with energy. In his attempt to elaborate a theory for beta decay,
Fermi assumed it to be point-like~\cite{FermiTheoryIta}.  As we now know the
short range of the weak force is due to the large masses of the mediators making
it only appear point-like in relatively low energy phenomena such as the beta
decay. Let us consider the following expansion in partial waves:
\begin{equation}
  \label{eq:156}
  \frac{\ud \sigma}{\ud \Omega} = \left| \frac{1}{2ik} \sum_{\ell = 0}^\infty
    (2 \ell + 1) A_\ell P_\ell(\cos \theta) \right|^2
\end{equation}
where $A_\ell$ is the partial wave amplitude for the orbital angular momentum
$\ell$ and $P_\ell$ are the Legendre polynomials. For point-like scattering
(also referred to as $s$-wave scattering) the $\ell = 0$ mode
dominates~\cite{VAUnitaryViolationBook} and there is no angular dependence
($P_0(\cos \theta) = 1$) thus we have:
\begin{equation}
  \label{eq:157}
  \frac{\ud \sigma}{\ud \Omega} = \frac{1}{4E^2} |A_0|^2.
\end{equation}
Requiring the unitarity ($|A_\ell| \leq 1$) for each partial wave, we have that:
\begin{equation}
  \label{eq:158}
  \frac{\ud \sigma}{\ud \Omega} \leq \frac{1}{4 E^2}
\end{equation}
and for the total cross section:
\begin{equation}
  \label{eq:159}
  \sigma = \int \frac{\ud \sigma}{\ud \Omega} \ud \Omega = 4 \pi \frac{\ud
    \sigma}{\ud \Omega} \leq \frac{\pi}{E^2}
\end{equation}
and we see that the cross section is subject to a \emph{unitarity bound}. If we
use the result of \cref{eq:155} we see that for energies grater than:
\begin{equation}
  \label{eq:160}
  E \geq \sqrt{\frac{\pi}{2 G_F}} \approx 367~\mathrm{GeV}
\end{equation}
unitarity is violated. The V-A theory is thus only valid for energies comparable
to the Fermi scale ($\approx 367$~GeV) while for higher energies the point-like
nature of the theory violates the unitarity of the scattering matrix. Feynman
and Gell-Mann~\cite{FeynmannGellManFermiInteraction} proposed that the weak
interaction could happen through the exchange of spin one bosons, this prompted
Schwinger~\cite{SchwingerGaugeTheory} to introduce a gauge theory of the weak
interactions mediated by the $W^\pm$ bosons where he also tried to include the
photon. A consistent theory for the weak interaction at high energies thus
required the exchange of mediator particles and could be obtained by analogy to
the electromagnetic case by formulating it as a gauge theory.

In order to determine the symmetry group fro the weak interactions, let us
simplify the notation by writing:
\begin{equation}
  \chi_L =
  \begin{pmatrix}
    \nu_{e_L} \\ e^-_L
  \end{pmatrix}
  \equiv
  \begin{pmatrix}
    \nu_e \\ e^-
  \end{pmatrix}
\end{equation}
and using the Pauli matrices
\begin{equation}
  \tau_\pm = \frac{1}{2}( \tau_1 \pm i \tau_2)
\end{equation}
we have for the two charged currents of \cref{eq:153}:
\begin{equation}
  \begin{split}
    J_\mu^{(+)} &= \overbar{\chi}_L \gamma_\mu \tau_+ \chi_L \\
    J_\mu^{(-)} &= \overbar{\chi}_L \gamma_\mu \tau_- \chi_L
  \end{split}
\end{equation}
we can complete a possible SU(2) symmetry group by introducing a \emph{neutral}
current:
\begin{equation}
  J_\mu^{(3)} = \overbar{\chi}_L \gamma_\mu \frac{\tau_3}{2} \chi_L = \frac{1}{2}
  \overbar{\nu}_L \gamma_\mu \nu_L - \frac{1}{2} \overbar{e}_L \gamma
  e_L
  \label{eq:161}
\end{equation}
and we thus have an \emph{isospin} triplet of weak currents:
\begin{equation}
  \label{eq:1}
  J_\mu^i = \overbar{\chi}_L \gamma_\mu \frac{\tau_i}{2} \chi_L
\end{equation}
whose \emph{generators} $T_i = \tau_i / 2$ give an $\sutwol$ algebra. The
subscript ``L`` indicates that only the left-handed chiral components of the
fermions couples to the isospin triplet of weak currents. Global transformation
of the $\sutwol$ group are of the form:
\begin{equation}
  \chi_L (x) \to \chi'_L (x) = e^{i \vec{\varepsilon} \cdotp \vec{T}} \chi_L(x)
  = e^{i \vec{\varepsilon} \cdotp \frac{\vec{\tau}}{2}} \chi_L(x),
\end{equation}
where $\chi_L$ is the \emph{fundamental representation} of the group, the right
handed fermions are singlet for the $\sutwol$, thus:
\begin{equation}
  e_{R} \to e'_{R}= e_{R}.
\end{equation}
Since we are considering the global transformations, we have no interaction, so
the Lagrangian can be written as:
\begin{equation}
  \label{eq:2}
  \mathcal{L} = \overbar{e} i \gamma^{\mu} \partial_{\mu} e + \overbar{\nu} i
  \gamma^{\mu} \partial_{\mu} \nu \equiv \overbar{\chi}_{L} i
  \gamma \partial \chi_{L} + \overbar{e}_{R} i \gamma \partial e_{R};
\end{equation}
where we have set $m_{e} = 0$ because the mass term couples right and left
components of the fermions and it is not $\sutwol$ invariant. In 1973, the
Gargamelle bubble chamber experiment~\cite{Gargamelle} detected events of the
type:
\begin{equation}
  \label{eq:3}
  \overbar{\nu}_{\mu}\eminus \rightarrow \overbar{\nu}_{\mu}\eminus \\
\end{equation}
\begin{equation}
  \label{eq:4}
  \begin{cases}
    \nu_{\mu} N \rightarrow \nu_{\mu} X \\
    \overbar{\nu}_{\mu} N \rightarrow \overbar{\nu}_{\mu} X
  \end{cases}
\end{equation}
which are evidence of a neutral current. It was natural to try to identify this
neutral current with the one of \cref{eq:161} but it was seen that the
experimentally observed current has both left-handed and right-handed components
thus the $J^{(3)}_{\mu}(x)$ current introduced above can not be used as it
involves only have left-handed fermions and needs to be combined with some other
current with also right-handed components. The electromagnetic current is
neutral and mixes left-handed and right-handed components:
\begin{equation}
  \label{eq:5}
  J_{\mu}^\mathrm{\, em} (x) = \overbar{\psi}(x) \gamma_{\mu} Q \psi (x)
\end{equation}
where $Q$ is the electric charge operator (with eigenvalue $Q = -1$ for the
electron) and is also the generator of the U(1)$_\mathrm{em}$ symmetry group of
the electromagnetism. For example in the case of the electron we have:
\begin{equation}
  \label{eq:162}
  j_\mu^\mathrm{\, em} = - \overbar{e} \gamma_\mu e = - \overbar{e}_L \gamma_\mu
  e_L - \overbar{e}_R \gamma_\mu e_R.
\end{equation}
The idea is to find the minimal symmetry group that contains the $\sutwol$ and
the U(1)$_\mathrm{em}$ generators and a current that completes the weak isospin
triplet but is invariant for $\sutwol$ transformations. To this end we can
define the \emph{hypercharge} operator:
\begin{equation}
  \label{eq:6}
  Y = 2 ( Q - T_{3}) \Rightarrow Q = T_{3} + \frac{Y}{2},
\end{equation}
for the current we can write:
\begin{equation}
  \label{eq:7}
  J_{\mu}^\mathrm{\, em} = J_{\mu}^{(3)} + \frac{1}{2} J_{\mu}^{Y}
\end{equation}
where:
\begin{equation}
  \label{eq:8}
  J_{\mu}^{Y} = \overbar{\chi}_L \gamma_{\mu} \chi_L - 2 \overbar{e}_R
  \gamma_\mu e_R.
\end{equation}
The hypercharge $Y$ generates a U(1)$_{Y}$ symmetry and since it is a $\sutwol$
singlet, leaves the Lagrangian of \cref{eq:2} invariant under the
transformations:
\begin{equation}
  \label{eq:9}
  \begin{split}
    \chi_{L}(x) \rightarrow \chi'_{L}(x) = e^{i \beta Y} \chi_{L}(x)
    \equiv e^{i \beta y_{L}} \chi_{L} \\
    e_{R}(x) \rightarrow e'_{R}(x) = e^{i \beta Y}e_{R}(x) \equiv e^{i \beta
      y_{R}}e_{R}.
  \end{split}
\end{equation}
We thus have unified the electromagnetic and weak interactions extending the
gauge group to $\sutwol \times \uoney$ and instead of having a single symmetry
group we have a direct product of groups each with his own coupling constant
thus in addition to the electric charge $e$ we will have another coupling to be
found. As mentioned in \cref{sec:standard-model} the idea of an extended
symmetry group that could unify the weak interaction and electromagnetism was
first explored by Glashow and later by Weinberg and Salam. Since we have a
direct product of symmetry groups, the generators of $\sutwol$ and those of
$\uoney$ commute. The commutation relations are give by:
\begin{equation}
  \label{eq:10}
  [T_{+},T_{-}] = 2 T_{3}; \quad [T_{3},T_{\pm}] = \pm T_{\pm}; \quad
  [Y,T_{\pm}] = [Y,T_{3}] = 0.
\end{equation}
Members of the same isospin triplet, have same hypercharge eigenvalue. The
relevant quantum numbers are summarized in \cref{tab:hyper}.
\begin{table}[htb]
  \renewcommand{\arraystretch}{1.25}
  \centering
  \begin{tabular}{l c c c c}
    \toprule
    {\bf Lepton} & $T$ & $T^{(3)}$ & $Q$ & $Y$ \\
    \midrule \midrule
    $\nu_{e}$ & $\frac{1}{2}$ & $ \phantom{-}\frac{1}{2}$ & 0 & -1 \\
    $e_{L}^{-}$ & $\frac{1}{2}$ & $-\frac{1}{2}$ & -1 & -1 \\
    \\
    $e_{R}^{+}$ & 0 & $\phantom{-}0$ & -1 & -2 \\
    \bottomrule
  \end{tabular} \quad
  \begin{tabular}{l c c c c}
    \toprule
    {\bf Quark} & $T$ & $T^{(3)}$ & $Q$ & $Y$ \\
    \midrule \midrule
    $u_{L}$ & $\frac{1}{2}$ & $\phantom{-}\frac{1}{2}$ & $\phantom{-}\frac{2}{3}$ &
                                                                                    $\phantom{-}\frac{1}{3}$ \\
    $d_{L}$ & $\frac{1}{2}$ & $-\frac{1}{2}$ & $-\frac{1}{2}$ &
                                                                $\phantom{-}\frac{1}{3}$ \\
    $u_{R}$ & 0 & $\phantom{-}0$ & $\phantom{-}\frac{2}{3}$ & $\phantom{-}\frac{4}{3}$ \\
    $d_{R}$ & 0 & $\phantom{-}0$  $$ & $-\frac{1}{3}$ & $-\frac{2}{3}$ \\
    \bottomrule
  \end{tabular}
  \caption{Weak isospin and hypercharge quantum numbers of leptons and quarks.}
  \label{tab:hyper}
\end{table}
%%% Local Variables:
%%% mode: latex
%%% TeX-master: "../search_for_DM_LED_with_ATLAS"
%%% End:
