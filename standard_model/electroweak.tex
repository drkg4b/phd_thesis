In 1930 Pauli suggested the introduction of a new particle in order to explain
the continuous spectrum of the electron in $\beta$ decays. He called it neutron
(in 1930 the neutron has not been discovered yet) as it had to be electrically
neutral in order to conserve charge. It was not until Fermi proposed his theory
for $\beta$ decay~\cite{FermiTheory} that the new particle was accepted and
named \emph{neutrino}. The Lagrangian in Fermi's theory is of the
current-current type and can be written as:
% We can now see how to find out the weak interaction symmetry group, to this end,
% let us start by writing out the \emph{Hamiltonian} for the weak interaction:
\begin{equation}
  \mathcal{L}_\mathrm{weak} = \frac{G_F}{\sqrt{2}} J_\mu^\dagger
  J^\mu
  \label{eq:150}
\end{equation}
where the $J_\mu$ current is given by:
\begin{equation}
  \label{eq:151}
  J_\mu = \overbar{\psi} \gamma^\mu \psi
\end{equation}
and $\psi$, $\overbar{\psi}$ are the spinors of the fermions involved in the
interaction. In 1949 Powell discovered in cosmic rays two particles that he
called the tau and the theta mesons. The tau could decay by weak interaction
into three pions ($\tau^+ \rightarrow \pi^+ \pi^+ \pi^-$) and the theta decays
into two pions ($\theta^+ \rightarrow \pi^+ \pi^0$). The measured lifetime and
mass of the tau and the theta mesons turned out to be the compatible within
experimental uncertainties hinting that they could be the same particle but in
that case parity conservation (a space-time symmetry) would be violated. This is
known as the $\tau - \theta$ puzzle. In 1950 Lee and Yang proposed that parity
could be violated in weak interactions~\cite{LeeYangParityViolation} which was
experimentally confirmed in 1956 by Madame Wu~\cite{WuExperiment}. The tau and
theta mesons were identified as the \emph{kaon} which decays are both parity
conserving ($k^+ \rightarrow \pi^+ \pi^+ \pi^-$) and parity violating
($k^+ \rightarrow \pi^+ \pi^0$). Wu's experiment involved the $\beta$ decay of
Cobalt 60 atoms to Nickel 60. The Cobalt atoms were immersed in a magnetic field
in order to align it with the spin of the nucleus and the observable in the
experiment is the product between the momentum of the electron and its spin
($\vec{J} \cdot \vec{p}$). Reverting the magnetic field is the same as applying
a parity transformation and since the momentum transforms as a vector and
changes sign under space reflection ($\vec{p} \to - \vec{p}$) but the spin
transforms as an axial vector and remains unchanged ($\vec{J} \to \vec{J}$) the
product $\vec{J} \cdot \vec{p}$ is not parity conserving and asymmetry in the
distributions of the experiment and its reflected version were observed. In 1958
Feynman and Gell-Mann proposed a parity violating extension of Fermi's
theory~\cite{VATheory} called the V-A (vector minus axial vector) theory which
introduces the \emph{chiral} operators:
\begin{equation}
  \label{eq:152}
  P_L = \frac{1}{2}(1 - \gamma^5); \quad P_R = \frac{1}{2}(1 + \gamma^5)
\end{equation}
where $\gamma_5 = i \gamma_0 \gamma_1 \gamma_2 \gamma_3$ and $\gamma_\mu$
($\mu =$ 0, 1, 2, 3) are the Dirac matrices. The chiral operators projects the
left and right chirality when applied on a spinor $\psi$. The weak current can
be re-written as:
\begin{equation}
  \begin{split}
    J_\mu & \equiv J_\mu^{(+)} = \overbar{\psi}_{\nu_e} \gamma_\mu \frac{1}{2}
    \left(1 - \gamma_5 \right) \psi_e \equiv \overbar{\nu}_{e_L}
    \gamma_\mu e_L \\
    J_\mu^\dagger & \equiv J_\mu^{(-)} = \overbar{\psi}_e \gamma_\mu \frac{1}{2}
    (1 - \gamma_5) \psi_{\nu_e} \equiv \overbar{e}_L \gamma_\mu \nu_{e_L},
  \end{split}
  \label{eq:153}
\end{equation}
here $J_\mu^{(+)}$ and $J_\mu^{(-)}$ are the charged weak currents and are a
combination of vector and axial vector currents where only the left chiral
fermions and the right chiral anti-fermions participate in the weak
interaction. One problem of the V-A theory is that, for example, the total cross
section for the inelastic neutrino scattering process
$\nu_\mu + e^- \rightarrow \nu_e + \mu^-$ is given
by~\cite{VAUnitaryViolationBook}:
\begin{equation}
  \label{eq:155}
  \sigma = \frac{G_F^2}{\pi}s
\end{equation}
where $s$ is the center of mass energy and it seems to grow indefinitely with
energy. If we consider the cross section expansion in partial waves:
\begin{equation}
  \label{eq:156}
  \frac{\ud \sigma}{\ud \Omega} = \left| \frac{1}{2ik} \sum_{\ell = 0}^\infty
    (2 \ell + 1) A_\ell P_\ell(\cos \theta) \right|^2
\end{equation}
where $A_\ell$ is the partial wave amplitude for the orbital angular momentum
$\ell$, since for the point-like scattering (or $s$-wave) assumed in Fermi's
theory only the $\ell = 0$ mode contributes and there is no angular dependence
($P_\ell(\cos \theta) = 1$) we see that:
\begin{equation}
  \label{eq:157}
  \frac{\ud \sigma}{\ud \Omega} = \frac{1}{4E^2} |A_\ell|^2.
\end{equation}
Requiring the unitarity ($|A_\ell| \leq 1$) for each partial wave, we have that:
\begin{equation}
  \label{eq:158}
  \frac{\ud \sigma}{\ud \Omega} \leq \frac{1}{4 E^2}
\end{equation}
and for the total cross section:
\begin{equation}
  \label{eq:159}
  \sigma = \int \frac{\ud \sigma}{\ud \Omega} \ud \Omega = 4 \pi \frac{\ud
    \sigma}{\ud \Omega} \leq \frac{\pi}{2E^2}
\end{equation}
and we see that the cross section is subject to a \emph{unitarity bound}. If we
use the result of \cref{eq:155} we see that for energies grater than:
\begin{equation}
  \label{eq:160}
  E \geq \frac{1}{\sqrt{2}} \sqrt{\frac{\pi}{G_F}} \approx 370~\mathrm{GeV}
\end{equation}
and unitarity is violated. The V-A theory is thus only valid for energies
comparable to the Fermi scale ($\approx 300$~GeV) while for higher energies the
point-like nature of the theory violates the unitarity of the scattering
matrix. A consistent theory for the weak interaction at high energies require
the exchange of mediator particles and can be obtained by analogy to the
electromagnetic case by formulating it as a gauge theory.

In order to determine the symmetry group fro the weak interactions, let us
simplify the notation by writing:
\begin{equation}
  \chi_L =
  \begin{pmatrix}
    \nu_{e_L} \\ e^-_L
  \end{pmatrix}
  \equiv
  \begin{pmatrix}
    \nu_e \\ e^-
  \end{pmatrix}
\end{equation}
and using the Pauli matrices
\begin{equation}
  \tau_\pm = \frac{1}{2}( \tau_1 \pm i \tau_2)
\end{equation}
we have:
\begin{equation}
  \begin{split}
    J_\mu^{(+)} &= \overbar{\chi}_L \gamma_\mu \tau_+ \chi_L \\
    J_\mu^{(-)} &= \overbar{\chi}_L \gamma_\mu \tau_- \chi_L
  \end{split}
\end{equation}
by introducing a \emph{neutral} current:
\begin{equation}
  J_\mu^{(3)} = \overbar{\chi}_L \gamma_\mu \frac{\tau_3}{2} \chi_L = \frac{1}{2}
  \overbar{\nu}_L \gamma_\mu \nu_L - \frac{1}{2} \overbar{e}_L \gamma e_L
\end{equation}
we have a \emph{triplet} of currents:
\begin{equation}
  \label{eq:1}
  J_\mu^i = \overbar{\chi}_L \gamma_\mu \frac{\tau_i}{2} \chi_L.
\end{equation}

Now if chose an $\sutwol$ transformation:
\begin{equation}
  \chi_L (x) \to \chi'_L (x) = e^{i \vec{\varepsilon} \cdotp \vec{T}} \chi_L(x)
  = e^{i \vec{\varepsilon} \cdotp \frac{\vec{\tau}}{2}} \chi_L(x),
\end{equation}
where $T_i = \tau_i / 2$ are the $SU(2)_L$ \emph{generators}, and think the
$\chi_L$ as the \emph{fundamental representation}, then the current triplet is a
triplet of $SU(2)_L$, the \emph{weak isotopic spin}.

The right handed fermions are singlet for the $\sutwol$, thus:
\begin{equation}
  e_{R} \to e'_{R}= e_{R}.
\end{equation}
Since we are considering the global transformations, we have no interaction, so
the Lagrangian reads:
\begin{equation}
  \label{eq:2}
  \mathcal{L} = \overbar{e} i \gamma^{\mu} \partial_{\mu} e + \overbar{\nu} i
  \gamma^{\mu} \partial_{\mu} \nu \equiv \overbar{\chi}_{L} i
  \gamma \partial \chi_{L} + \overbar{e}_{R} i \gamma \partial e_{R};
\end{equation}
for now we are bounded to set $m_{e} = 0$, in fact the mass term couples right
and left fermion's components and it is not $SU(2)_{L}$ invariant.  In 1973, the
Gargamelle bubble chamber experiment~\cite{Gargamelle} detected events of the
type:
\begin{equation}
  \label{eq:3}
  \overbar{\nu}_{\mu}\eminus \rightarrow \overbar{\nu}_{\mu}\eminus \\
\end{equation}
\begin{equation}
  \label{eq:4}
  \begin{cases}
    \nu_{\mu} N \rightarrow \nu_{\mu} X \\
    \overbar{\nu}_{\mu} N \rightarrow \overbar{\nu}_{\mu} X
  \end{cases}
\end{equation}
which are evidence of a neutral current. Further investigations yielded that the
neutral weak current is predominantly $V-A$ (i.e. left-handed) but not purely
$V-A$ so the $J^{(3)}_{\mu}(x)$ current introduced above can not be used as it
involves only left handed fermions.  We know a neutral current that mixes left
and right components namely the electromagnetic current:
\begin{equation}
  \label{eq:5}
  J_{\mu} \equiv e J_{\mu}^{(em)} = e \overbar{\psi} \gamma_{\mu} Q \psi
\end{equation}
where $Q$ is the charge operator with eigenvalue $Q = -1$ for the electron. $Q$
is the generator of the $U(1)_{(em)}$ group. So we have an isospin triplet and
we have included the right hand components, the isospin singlet, what we want to
do, is to combine them and define the hypercharge operator:
\begin{equation}
  \label{eq:6}
  Y = 2 ( Q - T_{3}) \rightarrow Q = T_{3} + \frac{Y}{2},
\end{equation}
for the current we have
\begin{equation}
  \label{eq:7}
  J_{\mu}^{(em)} = J_{\mu}^{(3)} + \frac{1}{2} J_{\mu}^{Y}
\end{equation}
where
\begin{equation}
  \label{eq:8}
  J_{\mu}^{Y} = \overbar{\psi} \gamma_{\mu} Y \psi
\end{equation}
so, by analogy, the hypercharge $Y$ generates a $U(1)_{Y}$ symmetry, and, as it
is a $SU(2)_{L}$ singlet, leaves \eqref{eq:2} invariant under the
transformations:
\begin{equation}
  \label{eq:9}
  \begin{split}
    \chi_{L}(x) \rightarrow \chi'_{L}(x) = e^{i \beta Y} \chi_{L}(x)
    \equiv e^{i \beta y_{L}} \chi_{L} \\
    e_{R}(x) \rightarrow e'_{R}(x) = e^{i \beta Y}e_{R}(x) \equiv e^{i \beta
      y_{R}}e_{R}.
  \end{split}
\end{equation}
We thus have incorporated the electromagnetic interactions extending the group
to $SU(2)_{L} \times U(1)_{Y}$ and instead of having a single symmetry group we
have a direct product of groups, each with his own \emph{coupling constant}, so,
in addition to $e$ we will have another coupling to be found.  Since we have a
direct product of symmetry groups, the generators of $SU(2)_{L}$, $T_{i}$, and
the generators of $U(1)_{Y}$, $Y$ commute, the commutation relations are:
\begin{equation}
  \label{eq:10}
  [T_{+},T_{-}] = 2 T_{3} \quad ; \quad [T_{3},T_{\pm}] = \pm T_{\pm}
  \quad ; \quad [Y,T_{\pm}] = [Y,T_{3}] = 0,
\end{equation}
member of the same isospin triplet, have same hypercharge eigenvalue; the
relevant quantum numbers are summarized in the table ~\ref{tab:hyper}.
\begin{table}[htb]
  \renewcommand{\arraystretch}{1.25}
  \centering
  \begin{tabular}{l c c c c}
    \hline
    {\bf Lepton} & $T$ & $T^{(3)}$ & $Q$ & $Y$ \\ \hline\hline
    $\nu_{e}$ & $\frac{1}{2}$ & $ \phantom{-}\frac{1}{2}$ & 0 & -1 \\
    $e_{L}^{-}$ & $\frac{1}{2}$ & $-\frac{1}{2}$ & -1 & -1 \\
    \\
    $e_{R}^{+}$ & 0 & $\phantom{-}0$ & -1 & -2 \\ \hline
  \end{tabular} \quad
  \begin{tabular}{l c c c c}
    \hline
    {\bf Quark} & $T$ & $T^{(3)}$ & $Q$ & $Y$ \\ \hline\hline
    $u_{L}$ & $\frac{1}{2}$ & $\phantom{-}\frac{1}{2}$ & $\phantom{-}\frac{2}{3}$ &
                                                                                    $\phantom{-}\frac{1}{3}$ \\
    $d_{L}$ & $\frac{1}{2}$ & $-\frac{1}{2}$ & $-\frac{1}{2}$ &
                                                                $\phantom{-}\frac{1}{3}$ \\
    $u_{R}$ & 0 & $\phantom{-}0$ & $\phantom{-}\frac{2}{3}$ & $\phantom{-}\frac{4}{3}$ \\
    $d_{R}$ & 0 & $\phantom{-}0$  $$ & $-\frac{1}{3}$ & $-\frac{2}{3}$ \\ \hline


  \end{tabular}
  \caption{Weak Isospin and Hypercharge Quantum Numbers of Leptons and Quarks}
  \label{tab:hyper}
\end{table}
%%% Local Variables:
%%% mode: latex
%%% TeX-master: "../search_for_DM_LED_with_ATLAS"
%%% End:
