Let us consider the Lagrangian for a complex scalar field $\phi$:
\begin{equation}
  \label{eq:25}
  \mathcal{L} = (\partial_{\mu} \phi^{*})(\partial_{\mu} \phi) -
  \underbrace{\mu^{2}\phi^{*}\phi - \lambda(\phi^{*}\phi)^{2}}_{V(\phi^{*}\phi)}
\end{equation}
where:
\begin{equation}
  \label{eq:26}
  \begin{split}
    \phi^{\phantom{*}} &= \frac{\phi_{1} + i \phi_{2}}{\sqrt{2}} \\
    \phi^{*} &= \frac{\phi_{1} - i \phi_{2}}{\sqrt{2}}
  \end{split}
\end{equation}
and for the Lagrangian we thus get:
\begin{equation}
  \label{eq:27}
  \mathcal{L} = \frac{1}{2} (\partial_{\mu} \phi_{1})^{2} +
  \frac{1}{2} ( \partial_{\mu} \phi_{2} )^{2} - \frac{\mu^{2}}{2} (
  \phi_{1}^{2} + \phi_{2}^{2} ) - \frac{\lambda}{4} ( \phi_{1}^{2} +
  \phi_{2}^{2} )^{2}.
\end{equation}
This Lagrangian is invariant for global gauge transformations (U(1) symmetry
group) of the form:
\begin{equation}
  \label{eq:28}
  \begin{cases}
    \phi^{\phantom{*}}(x) \rightarrow {\phi'}^{\phantom{*}}(x) = e^{\phantom{-} i
      \epsilon}
    \phi^{\phantom{*}}(x) \\
    \phi^{*} (x) \rightarrow {\phi'}^{*} (x) = e^{- i \epsilon} \phi^{*}(x).
  \end{cases}
\end{equation}
There are two possible choices for the potential:
\begin{enumerate}[A -]
\item $\mu^{2} > 0$, which gives a stable configuration around $|\phi| = 0$ with
  a unique minimum.
\item $\mu^{2} < 0$, which gives a local maximum for $\phi = 0$ and a circle of
  degenerate minima such that $\phi_{1}^{2} + \phi_{2}^{2} = v^{2}$, with
  $v^{2} = - \mu^{2} / \lambda$.
\end{enumerate}
In case A the potential has the shape of a parabola and if we perform a power
expansion around $\phi = \phi* = 0$ we obtain a massive particle ($m = \mu^2$)
with $\pm 1$ charge with an interaction given by the $\phi^4$ term and the
Lagrangian is invariant for global U(1) phase rotations. In the B case the
potential has the Mexican hat shape where all the different states corresponding
to the same minimum energy have different orientations in the complex plane thus
breaking the gauge invariance under U(1) rotations, in fact:
\begin{equation}
  \label{eq:29}
  \phi_{0} = \bra 0 \phi \ket 0 \rightarrow \frac{v}{\sqrt{2}} e^{i
    \alpha} \quad \mathrm{if} \quad \phi \rightarrow e^{i \alpha} \phi.
\end{equation}
In order to explore the physics content of this theory we make a perturbative
expansion around one particular vacuum state, we chose one for example
$\alpha = 0$ for which $\phi_{1} = v$ and $\phi_{2} = 0$ and introduce the two
perturbations $\eta(x)$ and $\xi(x)$ so that:
\begin{equation}
  \label{eq:30}
  \phi (x) = \frac{1}{\sqrt{2}} \left[ \overbrace{v + \xi(x)}^{\phi_{1}} + i
    \overbrace{\eta(x)}^{\phi_{2}} \right] \quad \eta,\xi \mathrm{\, real}
\end{equation}
and plug them in the Lagrangian of \cref{eq:27} we obtain that:
\begin{equation}
  \label{eq:31}
  \begin{aligned}
    \mathcal{L}' (\xi,\eta) &= \frac{1}{2}(\partial_{\mu} \xi)^{2} + \frac{1}{2}
    (\partial_{\mu} \eta)^{2} - \frac{1}{2}(-2 \mu^{2})\eta^{2} \\ &- \lambda v
    (\eta^{2} + \xi^{2}) \eta - \frac{1}{4}(\eta^{2} + \xi^{2})^{4} + \cdots
  \end{aligned}
\end{equation}
and as we can see, the third term looks like a mass term so that the field
$\eta$ has mass $m_{\eta}^{2} = -2 \mu^{2}$ while we have no mass term for the
field $\xi$. Thus by choosing a particular vacuum state among the infinite
degenerate ones we lost the original U(1) gauge invariance (we thus \emph{broke
  the symmetry}) and recovered the mass for the gauge boson $\eta$ but generated
an additional massless field $\xi$. The generation of a massless scalar (spin
zero) boson, known as \emph{Goldstone boson}, for each generator of the gauge
symmetry is a general property formulated in the \emph{Goldstone theorem} which
was proved by Goldstone, Salam and Weinberg in 1962~\cite{GoldstoneTheorem}.
Since no massless scalar particle were observed a new mechanism was introduced
in 1964 to generate the mass for the gauge bosons without the presence of the
Goldstone bosons. This is briefly treated in \ref{sec:higgs-model}.
%%% Local Variables:
%%% mode: latex
%%% TeX-master: "../search_for_DM_LED_with_ATLAS"
%%% End:
