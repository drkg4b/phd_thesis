Particles are classified in the Standard Model according to their properties,
the first one that can be used is the spin. Fermions have \emph{half integer}
while bosons have \emph{integer} spin. Fermions are further divided in
\emph{leptons} and \emph{quarks} which are currently assumed to be truly
elementary particles meaning that they lack substructure and can be treated as
point-like. Leptons, unlike quarks, do not have a \emph{color charge} and
therefore cannot experience the \emph{strong} interaction. There are six types
or \emph{flavors} of leptons divided in \emph{three generations} according to
their \emph{lepton number}. Each generation consists of an electrically charged
and a neutral lepton, the \emph{electron} ($e^-$) and the \emph{electron
  neutrino} ($\nu_e$) constitute the first generation, they both carry
\emph{electron number} (L$_e = 1$). The \emph{muon} ($\mu^-$) and the \emph{muon
  neutrino} ($\nu_\mu$) have a \emph{muon number} (L$_\mu = 1$) and make up the
second generation. The \emph{tau} ($\tau^-$) and the \emph{tau neutrino}
($\nu_\tau$) in the third generation carry \emph{tau number} (L$_\tau = 1$).

There are also six flavors of quarks divided into three generations each of
which consists of one $+2/3$ and one $-1/3$ electric charge quark. The first
generation groups together the \emph{up} ($u$) and \emph{down} ($d$) quarks, the
second the \emph{charm} ($c$) and \emph{strange} ($s$) while the third the
\emph{top} ($t$) and the \emph{bottom} ($b$). As mentioned earlier, each quark
also has color charge with three possibilities: \emph{Red, Green} and
\emph{Blue} (R, G, B) and \emph{baryonic number} of 1/3. All the visible
(detectable and stable) matter in the universe is made of fermions belonging to
the first generation while the other two quickly decay to the first one.

Elementary bosons of the Standard Model have \emph{spin one} and mediate the
three fundamental interactions: the \emph{electromagnetic}, the \emph{weak} and
the \emph{strong}. The \emph{photon} ($\gamma$) mediates the electromagnetic
interaction, the $W^\pm$ (with electric charge) and the $Z^0$ (neutral) bosons
are the weak interaction carriers and finally there are eight \emph{gluons} that
are responsible of carrying the strong force. The Standard Model also contains a
spin zero boson, the \emph{Higgs} boson, discussed in
\cref{sec:higgs-mechanism}. The Standard Model does not describe gravitation at
the microscopic level but it is believed to be mediated by a massless gauge
boson with spin two called the \emph{graviton} which so far (as of 2017) has not
been detected by experiments. The photon and the gluons are massless thus
according to the uncertainty principle:
\begin{equation}
  \label{eq:149}
  \Delta E \Delta t \approx mc^2 \Delta t > \frac{\hbar}{2} \quad \Rightarrow \quad
  \underbrace{c \Delta t}_{s \equiv \mathrm{range}} > \frac{\hbar}{2mc}
\end{equation}
the interaction they mediate has infinite range. This is true for the photon but
since gluons carry the color charge themselves they can self-interact thus
shortening the range of the strong interaction.

Finally the \emph{hadron} category groups together \emph{mesons} and
\emph{baryons}. Mesons are bosons that are a bound state of one quark and one
anti-quark while \emph{baryons} are fermions made of three quarks.
% \begin{table}[!hb]
%   \centering
%   \begin{tabular}{lcccc}
%     \toprule
%     \multicolumn{5}{c}{Lepton Classification} \\
%     \midrule \midrule
%     Lepton & Q & L$_e$ & L$_\mu$ & L$_\tau$ \\
%     $e$ & -1 & 1 & 0 & 0 \\
%     $\nu_e$ & \phantom{-}{0} & 1 & 0 & 0 \\
%     $\mu$ & -1 & 0 & 1 & 0 \\
%     $\nu_\mu$ & \phantom{-}{0} & 0 & 1 & 0 \\
%     $\tau$ & -1 & 0 & 0 & 1 \\
%     $\nu_\tau$ & \phantom{-}{0} & 0 & 0 & 1 \\
%     \bottomrule
%   \end{tabular}
%   \caption{Blabla}
%   \label{tab:fermion_generations}
% \end{table}

% The \gls{sm} is a theoretical model which describes the elementary constituents
% of matter and their interactions. Up to now, we discovered three kind of
% different interactions, the \emph{strong}, the \emph{electroweak} and the
% \emph{gravitational}; excluding gravity, all of them are described by means of a
% \emph{quantum field gauge theory}.

% The Standard Model is the collection of these gauge theories, it is based on the
% gauge symmetry group $SU(3)_C \times SU(2)_L \times U(1)_Y$ where $SU(3)_C$ is
% the symmetry group of the \gls{qcd}, the ``C'' subscript stands for \emph{color
%   charge} which is the conserved charge in the strong interaction. The $SU(2)_L$
% is the weak isospin group acting on \emph{left-handed} doublet of fermions while
% the $U(1)_Y$ group is the \emph{hypercharge} symmetry group. Together
% $SU(2)_L \times U(1)_Y$ form the electroweak symmetry group.

% The Standard Model also contains and has predicted the existence of
% \emph{elementary particles} that interacts between them via the forces mentioned
% above. The matter constituents are called \emph{fermions}, the interaction are
% mediated by other particles called \emph{gauge bosons}. Fermions are further
% categorized into \emph{quark} which bound to form \emph{hadrons} and interact
% through the strong force and \emph{leptons} which do not experience the strong
% force. These are the true fundamental constituents of matter; the gauge bosons
% arise from the gauge symmetry group of the Standard Model.

% The existence of all the leptons, quarks and gauge bosons is confirmed by
% experimental tests. Among the bosons, the Higgs boson is peculiar because,
% unlike the others, it carries spin 0 and it is not associated with any
% interaction, instead arises as a consequence of the \emph{spontaneously broken
%   symmetry} of the electroweak sector which is the property, responsible of
% giving mass to all the elementary particles and the weak gauge bosons.
%%% Local Variables:
%%% mode: latex
%%% TeX-master: "../search_for_DM_LED_with_ATLAS"
%%% End:
