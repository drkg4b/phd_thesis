Let us consider the term:
\begin{equation}
  \label{eq:15}
  \mathcal{L}_{int}^{ew} = - g \overbar{\chi}_{L} \gamma_{\mu}
  \frac{\vect{\tau}}{2} \chi_{L} \vect{W}_{\mu} + \frac{g'}{2}
  \overbar{\chi}_{L} \gamma_{\mu} \chi_{L} B^{\mu} + g' \overbar{e_{R}}
  \gamma_{\mu} e_{R} B^{\mu}
\end{equation}
defining:
\begin{equation}
  \label{eq:16}
  W^{\pm}_{\mu} = \frac{1}{\sqrt{2}} \left( W^{(1)} \mp i W^{(2)} \right)
\end{equation}
we can write:
\begin{equation}
  \label{eq:17}
  \mathcal{L}^{CC} = - \frac{g}{\sqrt{2}} \left(J^{(+)}_{\mu} W^{- \mu} +
    J^{(-)}_{\mu} W^{+ \mu} \right)
\end{equation}
and recognize two charged vector bosons, the $W^+$ and the $W^-$, with coupling
given by ``$g$``.
%%% Local Variables:
%%% mode: latex
%%% TeX-master: "../search_for_DM_LED_with_ATLAS"
%%% End:
