In order to apply the Higgs mechanism introduced in \cref{sec:higgs-model} to
generate the masses of the $W^\pm$ and $Z^0$ weak bosons keeping the photon
massless and preserving the renormalizability of the electroweak theory, we add
the the Lagrangian of \cref{eq:13} the Higgs Lagrangian:
\begin{equation}
  \label{eq:45}
  % \mathcal{L}_{\phi} = \left[ \left(\partial_{\mu} + i g\, \vect{T} \cdot
  %     \vect{W}_{\mu} + i \tfrac{g'}{2}\,Y B_{\mu} \right)\phi \right]^{\dagger}
  % \left[ \left(\partial_{\mu} + i g\, \vect{T} \cdot
  %     \vect{W}_{\mu} + i \tfrac{g'}{2}\,Y B_{\mu} \right)\phi \right] - V(\phi)
  \mathcal{L}_{\phi} = \left| \left(\partial_{\mu} + i g\, \vect{T} \cdot
      \vect{W}_{\mu} + i \tfrac{g'}{2}\,Y B_{\mu} \right)\phi \right|^2 - V(\phi)
\end{equation}
with
$V(\phi) = \mu^{2}\phi^{\dagger} \phi + \lambda ( \phi^{\dagger} \phi )^{2}$ and
$\lambda > 0$. To preserve $\sutwol \times \uoney$ gauge invariance for this
Lagrangian we choose four fields to be arranged in an isospin doublet with
hypercharge $Y = 1$:
 \begin{equation}
  \label{eq:46}
  \phi = \binom{\phi^{+}}{\phi^{0}} =
  \frac{1}{\sqrt{2}}
  \begin{pmatrix}
    \phi_{1} + i \phi_{2} \\
    \phi_{3} + i \phi_{4}
  \end{pmatrix}.
\end{equation}
To generate the boson masses we consider the case where in the potential
$V(\phi)$ we have $\mu^2 < 0$ and choose a particular vacuum expectation value
of $\phi$:
\begin{equation}
  \label{eq:47}
  \phi_{0} \equiv \frac{1}{\sqrt{2}} \binom{0}{v}.
\end{equation}
With this choice the $\sutwol \times \uoney$ symmetry is broken, in fact the
vacuum is no more invariant under transformations of this group, for example:
\begin{equation}
  \label{eq:163}
  \begin{aligned}[t]
    T_3 \ket{\phi_0} & = \frac{1}{2}
    \begin{pmatrix}
      1 & 0 \\
      0 & -1
    \end{pmatrix}
    \frac{1}{\sqrt{2}}
    \begin{pmatrix}
      0 \\
      v
    \end{pmatrix}
    = - \frac{1}{2 \sqrt{2}}
    \begin{pmatrix}
      0 \\
      v
    \end{pmatrix} \\
    % T_+ \ket{\phi_0} & =
    % \begin{pmatrix}
    %   0 & 1 \\
    %   0 & 0
    % \end{pmatrix}
    % \frac{1}{\sqrt{2}}
    % \begin{pmatrix}
    %   0 \\
    %   v
    % \end{pmatrix}
    % = \frac{1}{\sqrt{2}}
    % \begin{pmatrix}
    %   0 \\
    %   v
    % \end{pmatrix}
  \end{aligned}
\end{equation}
and we get that:
\begin{equation}
  \label{eq:164}
  \phi_0 \rightarrow \phi_0' = e^{i \epsilon^3 T_3} \phi_0 \simeq \left( 1 +
    \epsilon^3 T_3 \right) \phi_0 \neq \phi_0
\end{equation}
and thus the three weak gauge bosons will acquire a mass through the Higgs
mechanism. We can also see that:
\begin{equation}
  \label{eq:165}
  \begin{aligned}
    Q \ket{\phi_0} & = \left( T_3 + \frac{Y}{2} \right) \ket{\phi_0} = \left[
      \begin{pmatrix}
        1/2 & 0 \\
        0 - 1/2
      \end{pmatrix}
      +
      \begin{pmatrix}
        1/2 & 0 \\
        0 & 1/2
      \end{pmatrix}
    \right] \frac{1}{\sqrt{2}}
    \begin{pmatrix}
      0 \\
      v
    \end{pmatrix} \\
    & =
    \begin{pmatrix}
      1 & 0 \\
      0 & 0
    \end{pmatrix}
    \frac{1}{\sqrt{2}}
    \begin{pmatrix}
      0 \\
      v
    \end{pmatrix} = 0
  \end{aligned}
\end{equation}
and the vacuum is invariant for U(1)$_\mathrm{em}$ transformations:
\begin{equation}
  \label{eq:166}
    \phi_0 \rightarrow \phi_0' = e^{i \epsilon(x) Q} \phi_0 = \phi_0 \quad
    \forall \epsilon(x)
\end{equation}
and this invariance ensures that the photon will not gain a mass through the
Higgs mechanism ($m_\gamma = 0$). To summarize, the choice of a particular
vacuum expectation value $\phi_0$ breaks the $\sutwol \times \uoney$ symmetry
and by applying the Higgs mechanism the masses of the weak gauge bosons are
generated but since $\phi_0$ is invariant under transformations of the
U(1)$_\mathrm{em}$ subgroup the photon remains massless.

As seen in \cref{sec:glob-symm-break} the gauge bosons masses are obtained
simply substituting the vacuum expectation value $\phi_{0}$ in the Lagrangian:
\begin{equation}
  \label{eq:167}
  \phi(x) \rightarrow \frac{1}{\sqrt{2}}
  \begin{pmatrix}
    0 \\
    v + h(x)
  \end{pmatrix},
\end{equation}
for the mass term we have:
\begin{equation}
  \label{eq:48}
  \begin{aligned}
     \MoveEqLeft \left| \left( g\,\tfrac{\vec \tau}{2} \cdot \vect{W}_{\mu} +
      \tfrac{g'}{2}
      B_{\mu} \right) \phi \right|^{2} = \\
    &= \frac{1}{8} \left|
      \begin{pmatrix}
        g W^{3}_{\mu} + g' B_{\mu} & g(W^{1}_{\mu} - i W^{2}_{\mu}) \\
        g(W^{1}_{\mu} + iW^{2}_{\mu}) & -g W^{3}_{\mu} + g' B_{\mu}
      \end{pmatrix}
      \binom{0}{v} \right|^{2} \\
    &= \frac{1}{8} v^{2}g^{2} \left[ \left(W^{1}_{\mu} \right)^{2} + \left(W^{2}_{\mu} \right)^{2} \right]
    + \frac{1}{8} v^{2} \left(g' B_{\mu} - g W_{\mu}^{3} \right) \left( g' B^{\mu} - g
    W^{3
      \mu} \right) \\
    &= \left(\frac{1}{2} gv \right)^{2} W^{+}_{\mu} {W^-}^{\mu} + \frac{1}{8} v^{2}
    \begin{pmatrix}
      W_{\mu}^{3} & B_{\mu}
    \end{pmatrix}
    \begin{pmatrix}
      g^{2} & - g g' \\
      - g g' & g^{2}
    \end{pmatrix}
    \begin{pmatrix}
      W^{3 \mu} \\ B^{\mu},
    \end{pmatrix}
  \end{aligned}
\end{equation}
where we used $W^{\pm} = ( W^{1} \mp i W^{2} ) / \sqrt{2}$. The first term of
the equation is similar to the expected mass term for a charged boson $M^2
W^+_\mu {W^-}^\mu$ and by comparison we conclude that:
\begin{equation}
  \label{eq:49}
  M_{W} = \frac{1}{2} g v.
\end{equation}
The last term of \cref{eq:48} is off diagonal in the $(W_\mu^3, B_\mu)$ basis
but can be diagonalized solving the eigenvalue equation in terms of the
eigenstates $Z_\mu$ and $A_\mu$:
\begin{equation}
  \label{eq:168}
  \begin{aligned}
    A^{\mu} & = \frac{g' W_{\mu}^{3} + g B_{\mu}}{\sqrt{g^{2} +
        {g'}^{2}}} \\
    Z^{\mu} & = \frac{g W_{\mu}^{3} - g' B_{\mu}}{\sqrt{g^{2} +
        {g'}^{2}}},
  \end{aligned}
\end{equation}
and we can write:
\begin{equation}
  \label{eq:169}
  \begin{aligned}
    \MoveEqLeft \frac{1}{8} v^{2} \left[ g^{2} \left(W_{\mu}^{3} \right)^{2} - 2
      g g' W_{\mu}^{3} B^{\mu} + g'^{2} B_{\mu}^{2} \right] = \\
    & = \frac{v^2}{8} \left[g W^{3}_{\mu} - g B_{\mu} \right]^2 + 0 \left[g'
      W^{3}_{\mu} - g' B_{\mu} \right]^{2} = \\
    & = \frac{v^2}{8} \left(g^2 + {g'}^2\right) Z_\mu Z^\mu + 0 \left(g^2 +
      {g'}^2 \right) A_\mu A^\mu.
  \end{aligned}
\end{equation}
The general mass term for neutral fields can be written as:
\begin{equation}
  \label{eq:170}
  \frac{1}{2} M_Z^2 Z_\mu Z^\mu + \frac{1}{2} M_A^2 A_\mu A^\mu
\end{equation}
and comparing this with \cref{eq:169} we have that:
\begin{equation}
  \label{eq:171}
  \begin{aligned}
    M_A & = 0 \\
    M_Z & = \frac{1}{2} v \sqrt{g^2 + {g'}^2}.
  \end{aligned}
\end{equation}
which are the mass term for the photon and for the neutral weak vector boson
fields.

After resolving the mass term we can explore the form of the Lagrangian in
\cref{eq:45} after the substitution of \cref{eq:167}:
% After the substitution of \cref{eq:167} and resolving the mass terms, the
% Lagrangian in \cref{eq:45} becomes:
\begin{equation}
  \label{eq:174}
  \mathcal{L}_\phi = \frac{1}{2} \left( \partial_\mu h \right)^2 +
  \frac{g^2}{4}(v + h)^2 W_\mu^+ {W^-}^\mu + \frac{1}{8}(v + h)^2 Z_\mu Z^\mu -
  V(\phi),
\end{equation}
the terms:
\begin{equation}
  \label{eq:175}
  \frac{g^2v}{2} W_\mu^+ {W^-}^\mu h \quad \mathrm{and} \quad \frac{g^2v}{4}
  Z_\mu Z^\mu h
\end{equation}
show that the coupling between the Higgs and the weak gauge fields is
proportional to their mass while the Higgs quadratic terms ($h^2$) represents
the interaction of two Higgs fields with two vector bosons. For the potential
$V(\phi)$ we have:
\begin{equation}
  \label{eq:172}
  V(\phi) = \frac{1}{2} (- 2 \mu^2) h^2 + \lambda v h^3 + \frac{\lambda}{4} h^4
\end{equation}
and the first term describes the mass of the Higgs boson:
\begin{equation}
  \label{eq:173}
  m_h = \sqrt{2 \mu^2}
\end{equation}
Since the value of $\mu^2$ is not known the mass of the Higgs boson cannot be
predicted and must be inferred from experiments. In 2012 the
ATLAS~\cite{ATLASHiggs} and CMS~\cite{CMSHiggs} experiments at CERN discovered
the Higgs boson needed to complete the Standard Model of weak interactions.
% Recalling that
% \begin{equation}
%   \label{eq:52}
%   \frac{g'}{g} = \tan \theta_{W}
% \end{equation}
% it is possible to write
% \begin{equation}
%   \label{eq:53}
%   \begin{split}
%     A_{\mu} &= \phantom{-} \cos \theta_{W} B_{\mu} + \sin \theta_{W}
%     W^{3}_{\mu}
%     \\
%     Z_{\mu} &= - \sin \theta_{W} B_{\mu} + \cos \theta_{W}
%     W^{3}_{\mu}.
%   \end{split}
% \end{equation}
% Thus the mass eigenstates are a massless vector boson, $A_{\mu}$ and a
% massive gauge boson $Z_{\mu}$.
%%% Local Variables:
%%% mode: latex
%%% TeX-master: "../search_for_DM_LED_with_ATLAS"
%%% End:
