% Consider the Higgs Lagrangian
% \begin{equation}
%   \label{eq:45}
%   \mathcal{L}_{\phi} = [(\partial_{\mu} + i g\, \vect{T} \cdot
%   \vect{W}_{\mu} + i \tfrac{g'}{2}\,Y B_{\mu})\phi]^{\dagger}  [(\partial_{\mu} + i g\, \vect{T} \cdot
%   \vect{W}_{\mu} + i \tfrac{g'}{2}\,Y B_{\mu})\phi] - V(\phi)
%  \end{equation}
%  with $V(\phi) = \mu^{2}\phi^{\dagger} \phi + \lambda (
%  \phi^{\dagger} \phi )^{2}$ and $\lambda > 0$. To preserve
%  $SU(2)_{L} \times U(1)_{Y}$ gauge invariance for this Lagrangian,
%  we can choose four fields to be arranged in an isospin doublet with
%  $Y = 1$
%  \begin{equation}
%   \label{eq:46}
% \phi = \binom{\phi^{+}}{\phi^{0}} \quad \mbox{with}
%   \begin{aligned}
%     &\quad \phi^{+} \equiv
%     \frac{1}{\sqrt{2}} ( \phi_{1} + i \phi_{2}) \\
%     &\quad \phi^{0} \equiv \frac{1}{\sqrt{2}} ( \phi_{3} + i
%     \phi_{4})
%   \end{aligned}
% \end{equation}
% and break the symmetry choosing
% \begin{equation}
%   \label{eq:47}
%   \phi_{0} \equiv \frac{1}{\sqrt{2}} \binom{0}{v}
% \end{equation}
% and taking $\mu^{2} < 0$ and $\lambda > 0$ in the potential
% $V(\phi)$.
The gauge bosons masses are generated simply substituting the vacuum
expectation value, $\phi_{0}$, in the Lagrangian, the relevant term is
\begin{equation}
  \label{eq:48}
  \begin{aligned}
     \MoveEqLeft \left| \left( g\,\tfrac{\vec \tau}{2} \cdot \vect{W}_{\mu} +
      \tfrac{g'}{2}
      B_{\mu} \right) \phi \right|^{2} = \\
    &= \frac{1}{8} \left|
      \begin{pmatrix}
        g W^{3}_{\mu} + g' B_{\mu} & g(W^{1}_{\mu} - i W^{2}_{\mu}) \\
        g(W^{1}_{\mu} + iW^{2}_{\mu}) & -g W^{3}_{\mu} + g' B_{\mu}
      \end{pmatrix}
      \binom{0}{v} \right|^{2} \\
    &= \frac{1}{8} v^{2}g^{2} \left[ \left(W^{1}_{\mu} \right)^{2} + \left(W^{2}_{\mu} \right)^{2} \right]
    + \frac{1}{8} v^{2} \left(g' B_{\mu} - g W_{\mu}^{3} \right) \left( g' B^{\mu} - g
    W^{3
      \mu} \right) \\
    &= \left(\frac{1}{2} gv \right)^{2} W^{+}_{\mu}W^{- \mu} + \frac{1}{8} v^{2}
    \begin{pmatrix}
      W_{\mu}^{3} & B_{\mu}
    \end{pmatrix}
    \begin{pmatrix}
      g^{2} & - g g' \\
      - g g' & g^{2}
    \end{pmatrix}
    \begin{pmatrix}
      W^{3 \mu} \\ B^{\mu},
    \end{pmatrix}
  \end{aligned}
\end{equation}
having used $W^{\pm} = ( W^{1} \mp i W^{2} ) / \sqrt{2}$. The mass
term, lead us to conclude that
\begin{equation}
  \label{eq:49}
  M_{W} = \frac{1}{2} g v.
\end{equation}
The remaining term is off diagonal
\begin{equation}
  \label{eq:50}
  \begin{split}
    \frac{1}{8} v^{2} [g^{2} (W_{\mu}^{3})^{2} - 2 g g' W_{\mu}^{3}
    B^{\mu} + g'^{2} B_{\mu}^{2} ] &= \frac{1}{8} v^{2} [g
    W^{3}_{\mu} - g B_{\mu} ]^{2} \\
    \quad &+ 0 \phantom{v^{2}} [g' W^{3}_{\mu} - g' B_{\mu} ]^{2}
  \end{split}
\end{equation}
but one can diagonalize and find that
\begin{equation}
  \label{eq:51}
  \begin{split}
    A^{\mu} &= \frac{g' W_{\mu}^{3} + g B_{\mu}}{\sqrt{g^{2} +
        g'^{2}}} \\
    Z^{\mu} &= \frac{g W_{\mu}^{3} + g' B_{\mu}}{\sqrt{g^{2} +
        g'^{2}}}
  \end{split}
\end{equation}
with $M_{A} = 0$ and $M_{Z} = v \sqrt{g^{2} + g'^{2}} / 2$ which are
the photon and neutral weak vector boson fields.
% Recalling that
% \begin{equation}
%   \label{eq:52}
%   \frac{g'}{g} = \tan \theta_{W}
% \end{equation}
% it is possible to write
% \begin{equation}
%   \label{eq:53}
%   \begin{split}
%     A_{\mu} &= \phantom{-} \cos \theta_{W} B_{\mu} + \sin \theta_{W}
%     W^{3}_{\mu}
%     \\
%     Z_{\mu} &= - \sin \theta_{W} B_{\mu} + \cos \theta_{W}
%     W^{3}_{\mu}.
%   \end{split}
% \end{equation}
Thus the mass eigenstates are a massless vector boson, $A_{\mu}$ and a
massive gauge boson $Z_{\mu}$.

We have shown in this section how the Higgs mechanism can be applied
to give mass to the gauge bosons of the electroweak model.
%%% Local Variables:
%%% mode: latex
%%% TeX-master: "../search_for_DM_LED_with_ATLAS"
%%% End:
