% As mentioned in \cref{sec:standard-model} the symmetry group of the strong
% interaction is $\suthreec$ where the subscript ``C`` stands for the \emph{color}
% charge which is a quantum number acquired by \emph{quarks}. Three R, G, B (red,
% green and blue) colors are introduced. The symmetry generated for rotations in
% the color space (R $\leftrightarrow$ G, G $\leftrightarrow$ B, B
% $\leftrightarrow$ R) is a $suthreec$ \emph{exact} symmetry, meaning that it is
% conserved by all interactions. All observed free particles are supposed to be
% \emph{colorless}, this can be achieved
The dynamic of the quarks can be described by the Lagrangian:
\begin{equation}
  \label{eq:127}
  \mathcal{L} = \bar{q}_i \left( i \gamma^\mu \partial_\mu - m \right) q_i +
  \mathcal{L}_\mathrm{int}
\end{equation}
where $q_i$ ($i$ = 1, 2, 3) are the color fields and $m$ is the mass of the
quark and it only depends on the quark flavor and not on the color charge. Let
us consider only one of the three possible color fields, thus
$q_i (x) \equiv q(x)$. To generate the interaction Lagrangian, we can impose on
the free Lagrangian:
\begin{equation}
  \label{eq:129}
    \mathcal{L}_0 = \bar{q} \left( i \gamma^\mu \partial_\mu - m \right) q
\end{equation}
gauge invariance under local $\suthreec$ transformations of the type:
\begin{equation}
  \label{eq:128}
  q(x) \rightarrow q'(x) = e^{i \sum_{k = 1}^8 \epsilon^k(x) T^k} q(x) \equiv
  \Omega(\epsilon(x)) q(x)
\end{equation}
where $T^k = \left( T^k \right)^\dag \equiv \lambda^k / 2$ ($k = 1, \dots, 8$)
are the generators of $\suthreec$, $\epsilon^k(x)$ are arbitrary space-time
functions and a common choice for the $\lambda^k / 2$ are the Gell-Mann
matrices. The commutation relation of the group generators is:
\begin{equation}
  \label{eq:130}
  \left[ T^k, T^l \right] = i f^{klm} T^m
\end{equation}
where $f^{klm}$ are the structure constants of the group and it can be seen that
not all the generators commute thus the symmetry group is non-Abelian
(i.e.
$\Omega(\epsilon_1) \Omega(\epsilon_2) \neq \Omega(\epsilon_2)
\Omega(\epsilon_1)$). If we consider the derivative of the transformed color
field:
\begin{equation}
  \label{eq:131}
  \begin{aligned}
    \partial_\mu q \rightarrow \partial_\mu q' & = \Omega(\epsilon) \partial_\mu q
    + \left( \partial_mu \Omega(\epsilon) \right) q \\ \quad & \equiv \Omega
    \left[ \partial_\mu + \Omega^{-1} \partial_\mu \Omega \right] q \neq
      \Omega \partial_\mu q
    \end{aligned}
\end{equation}
and we see that the invariance under the $\Omega(\epsilon(x))$ transformation is
broken. We need to define a derivative such that:
\begin{equation}
  \label{eq:132}
  D_\mu q \rightarrow \left( D_\mu q \right)' = \Omega(\epsilon(x)) D_\mu q
  \equiv \left( \Omega D_\mu \Omega^{-1} \right) \Omega q \equiv D_\mu' q',
\end{equation}
to this end we introduce eight vector fields $G_\mu^k(x)$ ($k = 1, \dots, 8$)
such that:
\begin{equation}
  \label{eq:133}
  D_\mu q(x) = \left( \partial_\mu + i g T^k G_\mu^k \right) q(x)
\end{equation}
where we use the conventions on repeated indexes $T^k G_\mu^k \equiv \sum_k T^k
G_mu^k(x)$. We want to find how the $G_\mu^k(x)$ fields transform under the
wanted transformation property of \cref{eq:132}, to this end:
\begin{equation}
  \label{eq:134}
  \left( D_\mu q \right)' = \left( \Omega \partial_\mu + \partial_\mu \Omega + i
    g T^k {G'}_\mu^k \Omega \right) q \equiv \Omega D_\mu q = \Omega
  \left( \partial_\mu + i g T^k G_\mu^k \right)q
\end{equation}
from which:
\begin{equation}
  \label{eq:135}
  T^k {G'}_\mu^k = \Omega T^k G_\mu^k \Omega^{-1} + \frac{i}{g}
  \left( \partial_\mu \Omega \right) \Omega^{-1} = \Omega T^k G_\mu^k
  \Omega^{-1} - \frac{i}{g} \Omega \partial_\mu \Omega^{-1}.
\end{equation}
Considering the infinitesimal transformations of the type:
\begin{equation}
  \label{eq:136}
  \Omega(\epsilon) \sim 1 + i \epsilon^k T^k; \quad \Omega^{-1}(\epsilon) \sim 1 -
  i \epsilon^k T^k,
\end{equation}
from \cref{eq:135}, we obtain:
\begin{equation}
  \label{eq:137}
  \begin{aligned}
    \MoveEqLeft T^k {G'}_\mu^k = \left( 1 + i \epsilon^l T^l \right) T^m
    G_\mu^m \left( 1 - i \epsilon^l T^l \right) \\
    & - \frac{i}{g} \left( 1 + i \epsilon^l T^l \right)\left( -i \partial_\mu
      \epsilon^k \right) T^k \\
    & \sim T^k G_\mu^k + i \epsilon^l \underbrace{\left[ T^l, T^m
      \right]}_{if^{lmk} T^k} G_\mu^m - \frac{1}{g}\left( \partial_\mu
      \epsilon^k \right) T^k \\
    & \equiv T^k \left[ G_\mu^k - f^{klm} \epsilon^l G_\mu^m -
      \frac{i}{g} \partial_\mu \epsilon^k \right].
   \end{aligned}
\end{equation}
And the looked for transformations are:
\begin{equation}
  \label{eq:138}
  G_\mu^k(x) \rightarrow {G'}_\mu^k(x) = G_\mu^k(x) -
  f^{klm}\epsilon^l(x)G_\mu^m(x) - \frac{i}{g}\partial_\mu \epsilon^k(x)
\end{equation}
and we are sure that the Lagrangian:
\begin{equation}
\mathcal{L}(q, D_\mu q)\label{eq:139} = \bar{q}\left( i \gamma^\mu D_\mu - m \right) q
\end{equation}
is invariant under $\suthreec$ local gauge transformations. The $G_\mu^k(x)$ ($k
= 1, \dots, 8$) fields are called \emph{gluons} and are the carriers of the
strong force between quarks as a consequence of the color charge.

To be able to write the full Lagrangian we need to find the equivalent of
$F_{\mu\nu} = \partial_\mu A_\nu - \partial_\nu A_\mu$ of the Abelian case also
for the gluons. This is needed since up to now the gluon were not treated as a
dynamic degree of freedom and we need to add their kinetic term in the
Lagrangian. It can be seen that in the Abelian case:
\begin{equation}
  \label{eq:140}
  \left[D_\mu, D_\nu \right] = -ieF_{\mu\nu}
\end{equation}
thus, by analogy, we can calculate:
\begin{equation}
  \label{eq:141}
  \begin{aligned}
    \MoveEqLeft \left[D_\mu, D_\nu \right] = \left[\partial_\mu + ig T^k G_\mu^k, \partial_\nu +
      ig T^k G_\nu^k \right] \\
    & = \cancel{\left[\partial_\mu, \partial_\nu \right]} + ig
    \left[T^kG_\mu^k, \partial_\nu \right] + ig \left[\partial_\mu, T^kG_\nu^k
    \right] - g^2 \left[T^kG_\mu^k, T^kG_\nu^k
    \right] \\
    & = ig \left( \partial_\mu T^kG_\nu^k - \partial_\nu T^k G_\mu^k \right) -
    g^2 \underbrace{\left[T^k, T^l \right]}_{if^{klm}T^m} G_\mu^k G_\nu^k \\
    & = ig T^m \left( \partial_\mu G_\nu^m - \partial_\nu G_\mu^m - g f^{mkl}
      G_\mu^k G_\nu^l\right) \\
    & \equiv ig T^m G_{\mu\nu} \equiv ig G_{\mu\nu}
  \end{aligned}
\end{equation}
where $G_{\mu\nu}^k = \partial_\mu G_\nu^k -
\partial_nu G_\mu^k - g f^{klm} G_\mu^l G_\nu^m$ and
$G_{\mu\nu} = T^k G_{\mu\nu}^k$ have been defined. Finally indicating with
$G_\mu \equiv T^k G_\mu^k$ we have:
\begin{equation}
  \label{eq:142}
  G_{\mu\nu} = \partial_\mu G_\nu - \partial_\nu G_\mu + ig \left[ G_\mu, G_\nu \right].
\end{equation}
The field strength tensor of \cref{eq:142} transforms according to:
\begin{equation}
  \label{eq:144}
  G'_{\mu\nu} = \Omega G_{\mu\nu} \Omega^{-1}
\end{equation}
and for any product of these tensors we have:
\begin{equation}
  \label{eq:145}
  G'_{\mu\nu} G'_{\rho \sigma} \dots G'_{\lambda \tau} = U \left( G_{\mu\nu}
    G_{\rho \sigma} \dots G_{\lambda \tau} \right) U^{-1}
\end{equation}
and the trace of these products is gauge invariant:
\begin{equation}
  \label{eq:146}
  \mathrm{Tr}(G'_{\mu\nu} \dots G'_{\lambda \tau}) = \mathrm{Tr}(UG_{\mu\nu}
  \dots G_{\lambda \tau}U^{-1}) = \mathrm{Tr}(G_{\mu\nu} \dots G_{\lambda \tau}).
\end{equation}
Thus in order to retain local gauge invariance for the gluon kinetic term the
simplest Lagrangian can be written as:
\begin{equation}
  \label{eq:143}
  \mathcal{L}_G = \mathrm{const~Tr}(G_{\mu\nu} G^{\mu\nu}) \equiv
  \mathrm{const~}\mathrm{Tr}(T^k T^l) G_{\mu\nu}^k G^{\mu\nu l}
\end{equation}
using the known property of the Gell-Mann matrices $\mathrm{Tr}(T^k T^l) =
\mathrm{Tr}(\lamnbda^k \lambda^l)/2$ and choosing the constant term we have:
\begin{equation}
  \label{eq:147}
  \mathcal{L}_G = - \frac{1}{2} \mathrm{Tr}(G_{\mu\nu}G^{\mu\nu}) = -
  \frac{1}{4} \sum_{k = 1}^8 G_{\mu\nu}^k G^{\mu\nu k}.
\end{equation}
The full Lagrangian can finally be written as:
\begin{equation}
  \label{eq:148}
  \mathcal{L} = \bar{q} \left( i \gamma^\mu D_\mu - m \right) q - g \left(
    \bar{q} \gamma^\mu T^k q \right) G_\mu^k - \frac{1}{4}G_{\mu\nu}^k G_k^{\mu\nu}
\end{equation}
where the first term is the free Lagrangian, the second one describes the
interaction between quarks and gluons and the final is the term for free gluon
motion and the gluon self interaction.
%%% Local Variables:
%%% mode: latex
%%% TeX-master: "../search_for_DM_LED_with_ATLAS"
%%% End:
