It is possible to set the start of elementary particle physics in 1897 with the
discovery of the electron by Thompson~\cite{ThompsonBook} but it was not until
the beginning of the 1900 when the idea of the photon came along that the first
big classification of particles could be made. In modern particle physics we
classify particles in elementary \emph{fermions}, the matter constituents, and
\emph{bosons}, the force carriers, and describe interactions using the
principles of \emph{gauge theories}.

In 1961 Glashow~\cite{GlashowPaper} introduced a model for the weak and
electromagnetic interactions based on the SU(2)$\times$U(1) symmetry
group. Requiring gauge invariance under phase transformations imply that the
gauge bosons are massless but it was known that they should be about hundred
times heavier than a proton in order to account for the short range of the weak
interaction. In 1967 and 1968 Weinberg~\cite{WeinbergPaper} and
Salam~\cite{SalamPaper} independently published a paper where they used the
SU(2)$\times$U(1) symmetry group but by introducing the spontaneous breaking of
the gauge symmetry recovered the masses of the weak force gauge bosons while
leaving the photon massless (see \cref{sec:higgs-mechanism}). In order for the
breaking of the symmetry to occur, the \emph{Higgs} ($H$) field must be
introduced. The Higgs field is a massive scalar particle (spin zero) that,
unlike the other gauge bosons, is not the mediator of any interaction but is
responsible of giving mass to the particles coupling to it. The Higgs boson was
discovered in 2012~\cite{ATLASHiggs,CMSHiggs}. At high energies the theory
developed by Glashow, Weinberg and Salam unify the weak and the electromagnetic
interactions in the \emph{electroweak} one and is known as the Standard Model of
electroweak interactions.

Currently the Standard Model of particle physics also includes the \emph{strong}
interaction and is the theoretical model that describes the properties of all
the known particles and their interactions. It is based on the
$\suthreec \times \sutwol \times \uoney$ symmetry group where $\suthreec$ is the
symmetry group of the \gls{qcd} and the ``C'' stands for the color charge
carried by quarks and gluons, the $\sutwol$ is the weak isospin group acting on
\emph{left-handed} doublet of fermions while the $\uoney$ group is the
\emph{hypercharge} symmetry group. The Standard Model is a very successful and
well tested theory that manage to explain most of the observed phenomena in
particle physics. Some of the still open questions of the \gls{sm} are addressed
in \cref{cha:beyond-stand-model}.
%%% Local Variables:
%%% mode: latex
%%% TeX-master: "../search_for_DM_LED_with_ATLAS"
%%% End:
