The \gls{sm} is a theoretical model which describes the elementary constituents
of matter and their interactions. Up to now, we discovered three kind of
different interactions, the \emph{strong}, the \emph{electroweak} and the
\emph{gravitational}; excluding gravity, all of them are described by means of a
\emph{quantum field gauge theory}.

The Standard Model is the collection of these gauge theories, it is based on the
gauge symmetry group $SU(3)_C \times SU(2)_L \times U(1)_Y$ where $SU(3)_C$ is
the symmetry group of the \gls{qcd}, the ``C'' subscript stands for \emph{color
  charge} which is the conserved charge in the strong interaction. The $SU(2)_L$
is the weak isospin group acting on \emph{left-handed} doublet of fermions while
the $U(1)_Y$ group is the \emph{hypercharge} symmetry group. Together
$SU(2)_L \times U(1)_Y$ form the electroweak symmetry group.

The Standard Model also contains and has predicted the existence of
\emph{elementary particles} that interacts between them via the forces mentioned
above. The matter constituents are called \emph{fermions}, the interaction are
mediated by other particles called \emph{gauge bosons}. Fermions are further
categorized into \emph{quark} which bound to form \emph{hadrons} and interact
through the strong force and \emph{leptons} which do not experience the strong
force. These are the true fundamental constituents of matter; the gauge bosons
arise from the gauge symmetry group of the Standard Model.

The existence of all the leptons, quarks and gauge bosons is confirmed by
experimental tests. Among the bosons, the Higgs boson is peculiar because,
unlike the others, it carries spin 0 and it is not associated with any
interaction, instead arises as a consequence of the \emph{spontaneously broken
  symmetry} of the electroweak sector which is the property, responsible of
giving mass to all the elementary particles and the weak gauge bosons.
%%% Local Variables:
%%% mode: latex
%%% TeX-master: "../search_for_DM_LED_with_ATLAS"
%%% End:
