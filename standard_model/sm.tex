We could say that elementary particle physics started in 1897 with the discovery
of the electron by Thompson~\cite{ThompsonBook} but it was not until the period
between 1900 and 1924 when the photon was discovered that the first big
classification of particles could be made. In fact, in modern particle physics,
we classify particles in two big groups: \emph{fermions} and \emph{bosons}. The
first are the matter constituents while the latter includes the interactions
carriers.

Fermions have \emph{half integer spin} and are further divided in \emph{leptons}
and \emph{quarks} which are currently assumed to be truly elementary particles
meaning that they lack substructure and can be treated as point-like. There are
six types or \emph{flavors} of leptons divided in \emph{three generations}
according to their \emph{lepton number}. Each generation consists of a charged
lepton and a neutral one, the \emph{electron} ($e^-$) and the \emph{electron
  neutrino} ($\nu_e$) constitute the first generation, they both carry
\emph{electron number} (L$_e = 1$). The \emph{muon} ($\mu^-$) and the \emph{muon
  neutrino} ($\nu_\mu$) have a \emph{muon number} (L$_\mu = 1$) and belongs to
the second generation. The \emph{tau} ($\tau^-$) and the \emph{tau neutrino}
($\nu_\tau$) in the third generation carry \emph{tau number} (L$_\tau =
1$). There are also the six flavors of quarks divided in three generations each
of which consists of a quark with $+2/3$ and the other with $-1/3$ electric
charge. The first generation groups together the \emph{up} ($u$) and \emph{down}
($d$) quarks, the second the \emph{charm} ($c$) and \emph{strange} ($s$) while
the third the \emph{top} ($t$) and the \emph{bottom} ($b$). Each quark also have
\emph{color charge} with three possibilities: \emph{Red, Green} and \emph{Blue}
(R, G, B). All the visible (detectable) matter in the universe is made of
fermions belonging to the first generation while the other two are heavier
copies of the first.

Bosons have \emph{integer spin} and can be elementary or composite. The
elementary ones are called \emph{gauge bosons} and mediate the interactions
between fermions. So far we know four fundamental interactions: the
\emph{electromagnetic}, the \emph{weak}, the \emph{strong} and the
\emph{gravitational} interaction. The \emph{photon} ($\gamma$) mediates the
electromagnetic interaction, the $W^\pm$ (with electric charge) and the $Z^0$
(neutral) bosons are the weak interaction carriers and finally there are eight
\emph{gluons} that are responsible of carrying the strong force. The photon, the
three weak force gauge bosons and the gluons all have spin one. The photon and
the gluons are massless thus according the the uncertainty principle:
\begin{equation}
  \label{eq:149}
  \Delta E \Delta t \approx mc^2 \Delta t > \frac{\hbar}{2} \quad \Rightarrow \quad
  \underbrace{c \Delta t}_{s \equiv \mathrm{range}} > \frac{\hbar}{2mc}
\end{equation}
the interaction they mediate should have infinite range. This is true for the
photon but since gluons carry the color charge themselves they can self-interact
thus shortening the range of the strong interaction. The gravitational
interaction is believed to be mediated by a massless gauge boson with spin two
called the \emph{graviton} which so far (as of 2017) has not been detected by
experiments. In 1961 Glashow~\cite{GlashowPaper} introduced a model for the weak
and electromagnetic interactions based on the SU(2)$\times$U(1) symmetry
group. Requiring gauge invariance under phase transformations imply that the
gauge bosons were massless but it was known that they should be about hundred
times heavier than a proton in order to account for the short range of the weak
interaction. In 1967 and 1968 Weinberg~\cite{WeinbergPaper} and
Salam~\cite{SalamPaper} independently published a paper where they used the
SU(2)$\times$U(1) symmetry group but by introducing the spontaneously brake of
the gauge symmetry recovered the masses of the weak force gauge bosons while
leaving the photon massless. In order for the braking of the symmetry to occur,
the \emph{Higgs} ($H$) field must be introduced. The Higgs field is a massive
scalar particle (spin zero) that, unlike the other gauge bosons, is not the
mediator of any interaction but is responsible for the masses of the particles
coupling to it. The Higgs boson was discovered in 2012. The theory developed by
Glashow, Weinberg and Salam unify the weak and the electromagnetic interactions
in the \emph{electroweak} one at high enough energies. The composite bosons are
divided in two categories: the \emph{mesons} which are a bound state of one
quark and one anti-quark and the \emph{hadrons}, made of three quarks.

The \gls{sm} of particle physics is the theoretical model that describes the
properties of the known particles and their interactions. It is based on the
$\suthreec \times \sutwol \times \uoney$ symmetry group where $\suthreec$ is the
symmetry group of the \gls{qcd} and the ``C'' stands for the color charge
carried by quarks and gluons, the $\sutwol$ is the weak isospin group acting on
\emph{left-handed} doublet of fermions while the $\uoney$ group is the
\emph{hypercharge} symmetry group. The \gls{sm} is a very successful and well
tested theory that manage to explain most of the observed phenomena in particle
physics. Some of the still open questions of the \gls{sm} are addressed in
\cref{cha:beyond-stand-model}.
% \begin{table}[!hb]
%   \centering
%   \begin{tabular}{lcccc}
%     \toprule
%     \multicolumn{5}{c}{Lepton Classification} \\
%     \midrule \midrule
%     Lepton & Q & L$_e$ & L$_\mu$ & L$_\tau$ \\
%     $e$ & -1 & 1 & 0 & 0 \\
%     $\nu_e$ & \phantom{-}{0} & 1 & 0 & 0 \\
%     $\mu$ & -1 & 0 & 1 & 0 \\
%     $\nu_\mu$ & \phantom{-}{0} & 0 & 1 & 0 \\
%     $\tau$ & -1 & 0 & 0 & 1 \\
%     $\nu_\tau$ & \phantom{-}{0} & 0 & 0 & 1 \\
%     \bottomrule
%   \end{tabular}
%   \caption{Blabla}
%   \label{tab:fermion_generations}
% \end{table}

% The \gls{sm} is a theoretical model which describes the elementary constituents
% of matter and their interactions. Up to now, we discovered three kind of
% different interactions, the \emph{strong}, the \emph{electroweak} and the
% \emph{gravitational}; excluding gravity, all of them are described by means of a
% \emph{quantum field gauge theory}.

% The Standard Model is the collection of these gauge theories, it is based on the
% gauge symmetry group $SU(3)_C \times SU(2)_L \times U(1)_Y$ where $SU(3)_C$ is
% the symmetry group of the \gls{qcd}, the ``C'' subscript stands for \emph{color
%   charge} which is the conserved charge in the strong interaction. The $SU(2)_L$
% is the weak isospin group acting on \emph{left-handed} doublet of fermions while
% the $U(1)_Y$ group is the \emph{hypercharge} symmetry group. Together
% $SU(2)_L \times U(1)_Y$ form the electroweak symmetry group.

% The Standard Model also contains and has predicted the existence of
% \emph{elementary particles} that interacts between them via the forces mentioned
% above. The matter constituents are called \emph{fermions}, the interaction are
% mediated by other particles called \emph{gauge bosons}. Fermions are further
% categorized into \emph{quark} which bound to form \emph{hadrons} and interact
% through the strong force and \emph{leptons} which do not experience the strong
% force. These are the true fundamental constituents of matter; the gauge bosons
% arise from the gauge symmetry group of the Standard Model.

% The existence of all the leptons, quarks and gauge bosons is confirmed by
% experimental tests. Among the bosons, the Higgs boson is peculiar because,
% unlike the others, it carries spin 0 and it is not associated with any
% interaction, instead arises as a consequence of the \emph{spontaneously broken
%   symmetry} of the electroweak sector which is the property, responsible of
% giving mass to all the elementary particles and the weak gauge bosons.
%%% Local Variables:
%%% mode: latex
%%% TeX-master: "../search_for_DM_LED_with_ATLAS"
%%% End:
