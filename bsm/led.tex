In 1919 Theodor Kaluza proposed that \gls{gr} in 4 + 1 dimensions could describe
both, gravity and electromagnetism in a unified geometrical
description~\cite{Kaluza}. In order to account for the four dimensional
character of the space-time the \emph{cylindrical conditions} are introduced:
\begin{equation}
  \label{eq:98}
  \begin{aligned}
    \partial_5 g_\mathrm{\, MN} & = 0 \\
    \partial_5 x^\mu & = 0 \\
    \partial_5 x'{^\mathrm{\, M}} & = \delta^\mathrm{\, M}_{5},
  \end{aligned}
\end{equation}
which means that in no reference frame the metric and the first four coordinates
depend on the fifth dimension. In this context it is possible to express the
metric tensor as:
\begin{equation}
  \label{eq:100}
  g_{\mathrm{\, MN}} =
  \left(
    \begin{array}{@{}c|c@{}}
      g_{\mu\nu} + \phi A_\mu A_\nu & \phi A_\nu \\
      \hline
      \phi A_\nu & g_{55} \equiv \phi
    \end{array}
  \right)
\end{equation}
where $\phi$ is a scalar field and $A^\mu$ is a gauge vector that can be
identified with the electromagnetic vector potential. The line element is given
by:
\begin{equation}
  \label{eq:101}
  \ud s^2_{5 \mathrm{D}} = g_{\mu\nu} \ud x^\mu \ud x^\nu + \phi \left( A_\nu
    \ud x^\nu + \ud x^5 \right)^2
\end{equation}
and the invariant volume element is:
\begin{equation}
  \label{eq:102}
  \ud^5 x \sqrt{- \mathrm{det}(g_{\mu\nu})} = \ud^4 x \ud x^5 \phi^{1/2} \sqrt{-
  \mathrm{det}(g_{\mu\nu}^{4 \mathrm{D}})}.
\end{equation}
With this notation it is possible to write the \gls{gr} Hilbert-Einstein action
in five dimensions as:
\begin{equation}
  \label{eq:103}
  S^{5 \mathrm{D}} = - \frac{1}{16 \pi G_5} \int \ud^5 \sqrt{-
    \mathrm{det}(g_{\mathrm{\, MN}}) R^{5 \mathrm{D}}}
\end{equation}
where $G_5$ is the gravitational constant in five dimensions and $R^{5
  \mathrm{D}}$ is the Ricci scalar. Since the other coordinates do not depend on
the fifth, the integration on the latter can be factorized:
\begin{equation}
  \label{eq:104}
  S^{5 \mathrm{D}} = - \frac{1}{16 \pi G_5} \int \ud x^5 \int \ud^4 \sqrt{-
    \mathrm{det}(g_{\mathrm{\, MN}}) R^{5 \mathrm{D}}}.
\end{equation}
The value of the integral in~\cref{eq:104} depends on the range of $x^5$ thus
for it to have a finite value, $x^5$ must be an interval or, how it was proposed
by Oskar Klein in 1926 a circle~\cite{Klein}. Some of the consequences of the
\gls{kk} theory are briefly outlined in the next section.
%%% Local Variables:
%%% mode: latex
%%% TeX-master: "../search_for_DM_LED_with_ATLAS"
%%% End:
