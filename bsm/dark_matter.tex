Observations on galaxies rotation suggests that their mass is not enough to
generate the gravitational force needed to counteract the effect of the
centripetal force and to prevent them to torn apart~\cite{RotationCurves}. Since
observation also suggest that stars and planets in galaxies are indeed kept
together by the gravitational attraction, we conclude that there must be extra
mass that we cannot see that generate the gravity needed to hold them
together. This unknown matter goes by the name of \emph{dark matter} and is
thought to be made of \glspl{wimp}. These particles do not interact or interact
weakly with the electromagnetic force, as a consequence, dark matter does not
emit, reflect or absorb light making it hard to detect. Cosmological
measurements on the content of matter of the universe have been
performed~\cite{DMEvidence} that seems to suggest that the visible matter (the
one we can see and measure), only accounts for 5\% of the total energy of the
universe and that dark matter constitutes roughly the 27\%, the remaining 68\%
is called \emph{dark energy} and is believed to be responsible for the
accelerated expansion of the universe. Some theories~\cite{WIMPIntro} predict
that dark matter particles should be light enough to be produced at hadron
colliders but due to their ``dark'' nature, they would escape detection leading
to an energy imbalance in the detector that could be used as an hint of their
existence.
%%% Local Variables:
%%% mode: latex
%%% TeX-master: "../search_for_DM_LED_with_ATLAS"
%%% End:
