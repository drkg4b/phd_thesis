An \gls{eft} is a tool used to describe physics phenomena of a complete theory
at energies much lower than a cut-off scale ($E \ll \Lambda$). In this low
energy limit it is possible to integrate out of the action the interactions
above $\Lambda$ while retaining their information in the couplings of the low
energy Lagrangian~\cite{EFTReview}.

The \gls{add} model~\cite{ADDPaper} is an effective theory for gravity where the
cut-off scale is given by the electroweak scale $m_\mathrm{\, EW}$ instead of
the Planck scale ($M_{Pl}$). In this context the strength of the gravitational
interaction ($1/M_{Pl}$) is recovered postulating the existence of $n$ extra
compact spatial dimensions of radius $R$. The Planck scale of this ($4 + n$)
dimensional space is taken to be $\approx m_\mathrm{\, EW}$ and denoted as
$\md$. In this framework, the gravitational flux lines for two masses, $m_1$ and
$m_2$, at a distance $r \ll R$ are allowed to propagate to the extra-dimensions
leading to a potential (given by Gauss' law):
\begin{equation}
  \label{eq:108}
  V(r) \approx \frac{m_1 m_2}{\md^{n + 2}} \frac{1}{r^{n + 1}},
\end{equation}
when $r \gg R$, the gravitational flux lines do not propagate to the
extra-dimensions any more and the $1/r$ potential is retrieved:
\begin{equation}
  \label{eq:109}
  V(r) \approx \frac{m_1 m_2}{\md^{n + 2} R^n} \frac{1}{r}.
\end{equation}
The effective Planck scale is thus $M_{Pl} \approx \md^{n + 2} R^n$ where
$\md^{n + 2}$ is the scale of the fundamental complete theory in $4 + n$
dimensions. The compactification radius $R$ can be estimated by recalling that
in this framework $\md \approx m_\mathrm{\, EW}$ and by setting $M_{Pl}$ to the
observed value:
\begin{equation}
  \label{eq:110}
  R \approx 10^{\frac{30}{n} - 17}~\mathrm{cm} \times \left(
    \frac{1~\mathrm{TeV}}{m_\mathrm{\, EW}} \right)^{1 + \frac{2}{n}}.
\end{equation}
In the context of the \gls{add} framework, the weakness of gravity is understood
in terms of the graviton propagating in the extra-dimensions. The hierarchy
problem is also solved by setting the cut-off scale and thus the order of the
corrections to the Higgs mass close to the electroweak scale.
%%% Local Variables:
%%% mode: latex
%%% TeX-master: "../search_for_DM_LED_with_ATLAS"
%%% End:
