The \emph{naturalness criterion} states that: ``one such [dimensionless and
measured in units of the cut-off ($\Lambda$)] parameter is allowed to be much
smaller than unity only if setting it to zero increases the symmetry of the
theory. If this does not happen, the theory is unnatural''~\cite{thooft:gauge}.

One important concept in physics that enter in the formulation of the
naturalness principle is that of symmetries. \emph{Symmetries} are closely
connected to conservation laws through the Noether's theorem, moreover theory
parameters that are protected by a symmetry, if smaller than the unit, are not
problematic according to the naturalness criterion.

Let us consider the strength of the gravitational force, characterized by the
Newton's constant, G$_N$ and the weak force, characterized by the Fermi's
constant G$_F$, if we take the ratio of these we get:
\begin{equation}
  \label{eq:gf_gn_ratio}
  \frac{\mathrm{G}_F \hbar^2}{\mathrm{G}_N c^2} = 1.738 \times 10^{33}.
\end{equation}
The reason why this number is worth some attention is that theory parameters
close to the order of unit in the SM, may be calculated in a more fundamental
theory, if any, using constants like $\pi$ or $e$ while numbers that deviates
from one, may not have such a simple mathematical expression and thus may lead
to uncover new properties of the fundamental theory.

This number becomes even more interesting if we consider quantum effects.
\emph{Virtual particles} are off-shell ($E \neq m^2 + p^2$) and according to the
\emph{uncertainty principle}, $\Delta t \Delta E \geq \hbar / 2$, can appear out
of the vacuum for a short time that depends on the energy of the virtual
particle; according to quantum field theory, the vacuum is populated with
virtual particles. The Higgs field has the property to couple with other SM
particles with a strength proportional to their mass. All these virtual
particles have a mass determined by the cut-off energy $\Lambda$ and when the
Higgs field travels through space, couples with these virtual particles and its
mass squared gets a contribution proportional to $\Lambda$
(see~\cite{Giudice:2008bi}):
\begin{equation}
  \label{eq:delta_mh}
  \delta m_H^2 = k \Lambda^2 \text{, with } k = \frac{3 G_F}{4 \sqrt{2}
    \pi^2}(4m_t^2 - 2m_W^2 - m_Z^2 - m_H^2).
\end{equation}
Since $k \approx 10^{-2}$~\cite{Giudice:2008bi}, the value of Higgs mass
$m_H (\sim \mathrm{G}_F^{-1/2})$, should be close to the maximum energy scale
$\Lambda$ and if we assume this to be the Planck scale $M_{Pl} = G_N^{-1/2}$, the
ratio $\mathrm{G}_F/\mathrm{G}_N$, should be close to the unity which
contradicts \cref{eq:gf_gn_ratio}, this goes by the name of \emph{hierarchy
  problem}.

The large quantum corrections in \cref{eq:delta_mh} are mainly due to the fact
that in the SM, there is no symmetry protecting the mass of the Higgs field.
%%% Local Variables:
%%% mode: latex
%%% TeX-master: "../search_for_DM_LED_with_ATLAS"
%%% End:
