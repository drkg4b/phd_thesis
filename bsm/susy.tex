One possible solution to the dark matter and the hierarchy problem is achieved
introducing a symmetry, called \gls{susy}~\cite{SUSYIntro}, that relates
fermions and bosons. Supersymmetry is capable of solving the hierarchy problem
by canceling out the quantum corrections that bring $m_H$ close to $\Lambda$
thus restoring the naturalness of the SM\@. SUSY introduces a set of new
particles that are the \emph{superpartners} of the SM ones. These supersymmetric
particles or \emph{sparticles} have the same quantum numbers and couplings of
their SM counterpart but turn with a supersymmetry transformation a spin 1/2
fermionic SM particle into a bosonic spin 0 sparticle and a spin 1 SM boson in a
spin 1/2 fermion while a scalar particle becomes a spin 1/2 sparticle. The
superpartners of the SM leptons are called \emph{sleptons}, adding the ``s''
prefix (that stands for ``scalar'') to their SM fermionic partners and are
denoted with a ``$\textasciitilde$'' on top of their symbol, so for example the
SM electrons are called \emph{selectrons} and denoted with
$\tilde{e}_\mathrm{\, L}$ and $\tilde{e}_\mathrm{\, R}$ where the R and L does
not refer to their helicity (being 0-spin bosons) but to that of their SM
superpartners. Similarly for \emph{smuons} and \emph{staus}:
$\tilde{\mu}_\mathrm{\, L}$, $\tilde{\mu}_\mathrm{\, R}$,
$\tilde{\tau}_\mathrm{\, L}$, $\tilde{\tau}_\mathrm{\, R}$ and for the
\emph{squarks}: $\tilde{q}_\mathrm{\, L}$, $\tilde{q}_\mathrm{\, R}$ (with
$q = u,\, d,\, s,\, b,\, c,\, t$). The supersymmetric partners of the gauge
bosons are called \emph{gauginos}, adding the ``ino'' suffix to their SM
partners, for instance, the $W$ and $Z$ bosons have the \emph{wino}
($\widetilde{W}$) and \emph{zino} ($\widetilde{Z}$) superpartners, the photon
has the \emph{photino} ($\widetilde{\gamma}$) and the Higgs field has the
\emph{Higgsino} ($\widetilde{H}$). In addition to these new particles, the
minimal supersymmetric extension of the Standard Model, called
\gls{mssm}~\cite{MSSMIntro}, introduces three neutral and two charged Higgs
bosons leading to five physical Higgs states. The neutral superpartners of these
Higgs states ($\widetilde{H}^{\, 0}$), combine with the bino and the wino
($\widetilde{B}$, $\widetilde{W}^{\, 0}$) in four new electrically neutral mass
eigenstates called \emph{neutralinos} ($\widetilde{\chi}^{\, 0}_{\, i}$ with
$i = $~1, 2, 3, 4). The charged higgsinos ($\widetilde{H}^{\, +}$,
$\widetilde{H}^{\, -}$) and the winos ($\widetilde{W}^{\, +}$,
$\widetilde{W}^{\, -}$) mix and give rise to two electrically charged mass
eigenstates of charge $\pm 1$ called \emph{charginos}
($\widetilde{\chi}^{\, \pm}_{\, i}$ with $i = $~1, 2). In the MSSM framework a
new multiplicative quantum number can be introduced, the R--parity, that, if it
is conserved, explains the stability of the proton and is defined as:
\begin{equation}
  \label{eq:91}
  R = (-1)^{3(B - L) + 2S}
\end{equation}
where $B$ and $L$ are the baryon and lepton numbers respectively and $S$ is the
spin. It can be seen that all SM particles have $R = + 1$ while supersymmetric
particles have $R = -1$. The conservation of R--parity has important
phenomenologiacal consequences:
\begin{itemize}
\item The \gls{lsp} is stable.
\item The final decay products of SUSY particles are an odd number.
\item In collider experiments sparticles can only be produced in pairs.
\end{itemize}
If the LSP is stable and electrically neutral (dark) and interacts only weakly
with matter it makes a good candidate for non--baryonic dark
matter~\cite{WIMPS}. In the MSSM, good LSP candidates are either the neutralino
or the gravitino, the supersymmetric partner of the graviton (the hypothetical
gauge boson associated to gravitation).

No sparticle has been observed at particle colliders so far, hinting that they
must have a larger mass than their SM counterparts. If they had the same mass
they would have already been produced at particle colliders, this imply that SUSY is
a \emph{broken symmetry}. Detecting one of these supersymmetric particles would
shed light on some of the SM shortcomings and thus many searches at the current
colliders focus their attention on finding SUSY particles.

Supersymmetry is expected to solve some of the shortcomings of the Standard
Model mentioned in \cref{sec:open-quest-stand}, it is possible to estimate, from
\cref{eq:delta_mh}, the scale at which new physics is expected. Using m$_H$ =
125~GeV~\cite{PDG}, we get that $\Lambda \approx 1$~TeV. If the naturalness
criterion holds, we thus expect the two main experiments at LHC, ATLAS and CMS,
to find signal for new physics at the TeV scale.
%%% Local Variables:
%%% mode: latex
%%% TeX-master: "../search_for_DM_LED_with_ATLAS"
%%% End:
