The soft missing energy ($E_\mathrm{\, x(y)}^\mathrm{\, miss,\, \text{soft}}$)
is a component of the $\met$ calculation described in
\cref{sec:miss-transv-energy} that relies on tracks associated to the primary
vertex and not associated with any hard object such as electrons, muons, jets,
taus or photons. The detector response to the soft missing energy reconstruction
is parametrized by an energy scale, a resolution in the perpendicular direction
to the hard scatter and a resolution along the main axis of the hard
scatter. The Monte Carlo to data mismodeling with respect to these three
parameters that characterize the detector response to soft missing energy is
modeled with three systematic uncertainties: syst\_MET\_SoftTrk\_Scale,
syst\_MET\_SoftTrk\_ResoPara and syst\_MET\_SoftTrk\_ResoPerp.
% \begin{description}[font=\normalfont]
% \item[syst\_MET\_SoftTrk\_Scale:] This parameter accounts for soft track term
%   scale uncertainties in the $\met$ reconstruction.
% \item[syst\_MET\_SoftTrk\_ResoPara:] This parameter models variations of the
%   soft term parallel projection with respect to the hadronic recoil system
%   transverse momentum of the resolution of the soft track term of the
%   $\met$.
% \item[syst\_MET\_SoftTrk\_ResoPerp:] This parameter models variations of the
%   soft term perpendicular projection with respect to the hadronic recoil system
%   transverse momentum of the resolution of the soft track term of the
%   $\met$.
% \end{description}
%%% Local Variables:
%%% mode: latex
%%% TeX-master: "../search_for_DM_LED_with_ATLAS"
%%% End:
