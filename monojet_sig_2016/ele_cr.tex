The $\crele$ is enriched in $\wenuplusjets$ processes and is used to constrain
the V + jets background contamination in the signal region. Besides cuts from B
to H as defined in \cref{sec:event-selection-1}, this region has the following
specific selections:
\begin{itemize}
\item The online trigger is required to select events with exactly one electron
  in the final state:
  \begin{itemize}
  \item For the 2015 data the event selection requires a low electron $\pt$
    threshold of 24~GeV and medium likelihood quality combined with a higher
    $\pt$ threshold of 60 and 120~GeV with medium and loose quality.
  \item For the 2016 data a tight electrons with a $\pt$ threshold of 26~GeV and
    loose isolation are combined with medium and loose quality electrons with 60
    and 140~GeV $\pt$ threshold respectively in order to select the events for
    the $\crele$ region.
  \end{itemize}
\item Events with baseline muons are vetoed.
\item In order to exclude electrons in the crack region (see
  \cref{sec:atlas-tilecal}), exactly one baseline electron with $\pt > 30$~GeV
  and $|\eta| > 1.52$ or $|\eta| < 1.37$ is selected.
\item A tight isolation criteria as defined in \cref{sec:electrons} for the
  selected electrons is required. Multijet \gls{qcd} events can enter the
  $\crele$ region if they have a jet that fakes an electron. A tight isolation
  requires low calorimeter and track activity around the electron thus removing
  most jets and single pions faking electrons. \pagebreak[4]
\item The transverse mass of the $\met - e$ system is required to be
  $30 < m_\mathrm{\, T} < 100$~GeV, compatible with a $W \rightarrow e \nu$
  process.
\item The events must satisfy: $\met/\sqrt{H_\mathrm{\, T}} > 5$~GeV$^{1/2}$
  ($\met$ \emph{significance}) where $H_\mathrm{T}$ is the scalar sum of the
  $\pt$ of the electron and the jets in the final state. Multijet \gls{qcd}
  events can enter the $\crele$ region if they have a fake electron and fake
  $\met$ from mis-measured electrons or jet $\pt$. In case of fake $\met$ the
  significance takes low values, that can be suppressed with this cut.
\end{itemize}
The single electron trigger, the veto of electrons in the crack region, the
tighter isolation and the $\met$ significance criteria all aim at reducing the
multi-jet contribution in this region. In order to apply the theory corrections
on V + jets discussed in \cref{sec:corrections-vjets} the \glspl{cr} have to be
enriched in $W$ + jets events thus requiring a uniform cut definition between
this control region and the $\crwmn$. For this reason in this version of the
analysis the cut on the transverse mass is
introduced.
\cref{fig:ele_cr_plots_2016}
% \cref{fig:ele_cr_met,fig:ele_cr_jet1}
shows the $\met$ and leading jet $\pt$
distribution of the $\crele$ control region. There is a good agreement within
uncertainties between data and \gls{mc} after the background only fit.
% \begin{figure}[!htb]
%   \centering
%   \includegraphics[width=.8\linewidth]{ele_cr_et_miss_2016}
%   \caption{Observed and predicted $\met$ distributions after the background only
%     fit in the $\crele$ control region for the $\met > 250$~GeV inclusive
%     selection corresponding to IM1. The error bands in the ratio plot on the
%     bottom of the figures include statistical and systematic uncertainties. The
%     negligible contribution of \gls{ncb} and diboson backgrounds is not reported
%     in the figure.}
%   \label{fig:ele_cr_met}
% \end{figure}
% \begin{figure}[!htb]
%   \centering
%   \includegraphics[width=.8\linewidth]{ele_cr_jet1_2016}
%   \caption{Observed and predicted $\met$ distributions after the background only
%     fit in the $\crele$ control region for the $\met > 250$~GeV inclusive
%     selection corresponding to IM1. The error bands in the ratio plot on the
%     bottom of the figures include statistical and systematic uncertainties. The
%     negligible contribution of \gls{ncb} and diboson backgrounds is not reported
%     in the figure.}
%   \label{fig:ele_cr_jet1}
% \end{figure}
\begin{figure}[!htb]
  \centering
  \begin{subfigure}[t]{.48\linewidth}
    \includegraphics[width=\linewidth]{ele_cr_et_miss_2016}
    \caption{$\met$ distribution.}
    \label{fig:ele_cr_met}
  \end{subfigure}
  \begin{subfigure}[t]{.48\linewidth}
    \includegraphics[width=\linewidth]{ele_cr_jet1_2016}
    \caption{Leading jet $\pt$ distribution.}
    \label{fig:ele_cr_jet1}
  \end{subfigure}
  \caption{Observed and predicted $\met$ and leading jet $\pt$ distributions
    after the background only fit in the $\crele$ control region for the
    $\met > 250$~GeV inclusive selection corresponding to IM1. The error bands
    in the ratio plot on the bottom of the figures include statistical and
    systematic uncertainties. The negligible contribution of \gls{ncb} and
    diboson backgrounds is not reported in the figure.}
  \label{fig:ele_cr_plots_2016}
\end{figure}
%%% Local Variables:
%%% mode: latex
%%% TeX-master: "../search_for_DM_LED_with_ATLAS"
%%% End:
