The evaluation of the tune, \gls{isr} and \gls{fsr} uncertainties follows what
is done for the \gls{susy} compressed models and described in
\cref{sec:initial-final-state}. Since the Monte Carlo signal samples for
\gls{add} are generated at the \gls{lo}, there is no parton shower matching
uncertainty in this case. Five tune parameters have been varied in order to
account for uncertainties from:
\begin{itemize}
\item Underlying event effects.
\item Jet structure effects.
\item Aspects of the MC generation that provide extra jet production.
\end{itemize}
For each \gls{add} model, ten systematic samples are produced and analyzed at
the truth level, the effect of the acceptance is evaluated in the different bins
used in the analysis. The final uncertainty in each bin affects the signal
acceptance and is a common envelope valid for the different extra dimensions
models (from \gls{add} n = 2 to \gls{add} n = 6). The results are summarized in
\cref{tab:add_tune_scale}.
\begin{table}[!ht]
  \centering
  \resizebox{\linewidth}{!}{\begin{tabular}{lccccccc}
    \toprule
    \multicolumn{7}{c}{Initial and Final State Radiation} \\
    \midrule \midrule
    $\met$~[GeV] & 400-500 & 500-600 & 600-700 & 700-800  & 800-900 & 900-1000 &
                   1000-3000 \\
    \midrule
    $\Delta A(\%)$ & 7 & 7 & 10 & 13 & 18 & 13 & 9 \\
    \bottomrule
  \end{tabular}}
  \caption{Tune, initial and final state radiation uncertainties in \% on the
    acceptance of the different signal region $\met$ bins in the analysis. The
    final value is a common envelope valid for all the \gls{add} models between n = 2
    and n = 6 dimensions.}
  \label{tab:add_tune_scale}
\end{table}
%%% Local Variables:
%%% mode: latex
%%% TeX-master: "../search_for_DM_LED_with_ATLAS"
%%% End:
