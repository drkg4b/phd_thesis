Uncertainties due to the parton shower, initial and final state radiation,
factorization and renormalization scales are estimated using dedicated
$t \bar{t}$ \gls{mc} samples and used also for single top processes. This choice
of the Monte Carlo samples is motivated by the fact that the $t \bar{t}$ process
contributes for approximately 80\% of the top background ($t \bar{t}$ and single
top) in the signal region. Variation due to the matrix element generator, parton
shower model and two different models of shower radiation with a different
factorization and renormalization scale are considered. Except for the variation
due to the amount of initial and final state radiation where the semi-difference
between the higher and lower variations is considered, the remaining ones are
symmetrized and applied considering them uncorrelated. Since similar $\met$
shapes are observed and in order to avoid statistical fluctuations, the final
uncertainties are evaluated in a single region obtained by summing the yields of
the signal and control regions. To estimate the systematic uncertainty
associated with each variation a linear fit is performed. The total systematic
uncertainties on the top background are obtained summing in quadrature each
variation. The results are reported in \cref{tab:top_syst_2016}.
\begin{description}[font=\normalfont]
\item[top\_Sys:] This parameter is used to constrain the top systematic
  uncertainty in the fit.
\end{description}
\begin{table}[!ht]
  \centering
  \begin{tabular}{cccccccccc}
    \toprule
    \multicolumn{10}{c}{Top Uncertainty} \\
    \midrule \midrule
    EM1 & EM2 & EM3 & EM4 & EM5 & EM6 & EM7 & EM8 & EM9 & EM10 \\
    \midrule
    24 & 26 & 28 & 31 & 36 & 42 & 47 & 53 & 59 & 72 \\
    \bottomrule
  \end{tabular}
  \caption{Total relative systematic uncertainty in \% on the top background in
    each exclusive bin EM1-EM10 used in the analysis.}
  \label{tab:top_syst_2016}
\end{table}
%%% Local Variables:
%%% mode: latex
%%% TeX-master: "../search_for_DM_LED_with_ATLAS"
%%% End:
