Uncertainties due to a different event generator, the parton shower model and
two models of initial and final state radiation are estimated using dedicated
$t \bar{t}$ \gls{mc} samples and used also for single top processes. This choice
of the Monte Carlo samples is motivated by the fact that the $t \bar{t}$ process
contributes for approximately 80\% of the top background ($t \bar{t}$ and single
top) in the signal region.
% Variation due to the matrix element generator, parton
% shower model and two different models of shower radiation with a different
% factorization and renormalization scale are considered.
Except for the variation due to the amount of initial and final state radiation
where the semi-difference between the higher and lower variations is considered,
the change in the \gls{mc} generator and parton shower model variations are
symmetrized and applied considering them uncorrelated. Since similar $\met$
shapes are observed in signal and control regions and in order to avoid
statistical fluctuations, the final uncertainties are evaluated in a single
region obtained by summing the yields of the signal and control regions.
% In order to obtain the value of the systematic uncertainty in the center of each
% $\met$ bin used in the analysis for all three variation considered (scale,
% parton shower and initial and final state radiation), a linear fit is performed.
The total systematic uncertainties on the top background are obtained summing in
quadrature each variation in each $\met$ bin.

The results are reported in \cref{tab:top_syst_2016} where it can be seen that
the uncertainty grows with the missing energy. This growth is mildly observed in
both the parton shower and the \gls{isr} and \gls{fsr} model variations while it
is more evident when changing the event generator. The top systematic
uncertainty in this version of the analysis are compatible with those calculated
in 2015.

The introduction of the $\crtop$ control region (see \cref{sec:muon-cr-bjet}) is
motivated by these large systematic uncertainties which are constrained by the
``top\_Sys'' nuisance parameter in the global fit.
\begin{table}[!htb]
  \centering
  \begin{tabular}{cccccccccc}
    \toprule
    \multicolumn{10}{c}{Top Uncertainty} \\
    \midrule \midrule
    EM1 & EM2 & EM3 & EM4 & EM5 & EM6 & EM7 & EM8 & EM9 & EM10 \\
    \midrule
    24 & 26 & 28 & 31 & 36 & 42 & 47 & 53 & 59 & 72 \\
    \bottomrule
  \end{tabular}
  \caption{Total relative systematic uncertainty in \% on the top background in
    each exclusive bin EM1-EM10 used in the analysis evaluated in a single
    region obtained by summing the yields of the signal and control regions.}
  \label{tab:top_syst_2016}
\end{table}
%%% Local Variables:
%%% mode: latex
%%% TeX-master: "../search_for_DM_LED_with_ATLAS"
%%% End:
