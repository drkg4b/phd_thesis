The generation of Monte Carlo samples is done before or during the data taking
period when the pile-up conditions are not yet known. For this reason at the
analysis level it is necessary to re-weight the \gls{mc} pile-up conditions to
what is found during the data taking, this procedure is commonly referred to as
\emph{pile-up re-weighting}.
\begin{description}[font=\normalfont]
\item[syst\_PRW\_DATASF:] Once the Monte Carlo samples have been re-weighted to
  the same distribution of number of $pp$ interaction per bunch crossing (see
  \cref{sec:initial-final-state}) as in data, it is observed that the event
  activity in data and simulation is still slightly different. Auxiliary
  datasets are used to derive a scaling of the Monte Carlo activity to match
  that of data. This nuisance parameter represents the uncertainty on the
  scaling to match the event activity in Monte Carlo to that of data at equal
  pile-up.
% \item[syst\_JvtEfficiency:] The efficiency of the \gls{jvt} selection (see
%   \ref{sec:jet-vertex-tagger}) on jets is slightly different in data and Monte
%   Carlo. It is measured in both and corrections are derived and applied. This
%   systematic uncertainty reflects the precision to which the ratio between data
%   and Monte Carlo simulations efficiency are known.
\end{description}
%%% Local Variables:
%%% mode: latex
%%% TeX-master: "../search_for_DM_LED_with_ATLAS"
%%% End:
