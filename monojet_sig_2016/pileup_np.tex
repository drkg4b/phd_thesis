The generation of Monte Carlo samples is done before or during the data taking
period when the pile-up conditions are not known precisely. For this reason at
the analysis level it is necessary to re-weight the \gls{mc} pile-up conditions
to what is found during the data taking, this procedure is commonly referred to
as \emph{pile-up re-weighting}.
\begin{description}[font=\normalfont]
\item[syst\_PRW\_DATASF:] Once the Monte Carlo samples have been re-weighted to
  the same number of minimum bias (see \cref{sec:initial-final-state})
  collisions per bunch crossing as the data, it is observed that the event
  activity in data and simulation is still slightly different. Auxiliary
  datasets are used to derive a scaling of the Monte Carlo activity to match
  that of data. This parameter represents the uncertainty applied to the
  correction due to the pile-up re-weighting in order to account for differences
  between data and Monte Carlo.
\item[syst\_JvtEfficiency:] The efficiency of the \gls{jvt} selection on jets
  in is slightly different in data and Monte Carlo. It is measured in both and
  corrections are derived and applied. This systematic uncertainty reflects the
  precision to which the ration between data and Monte Carlo simulations
  efficiency are known.
\end{description}
%%% Local Variables:
%%% mode: latex
%%% TeX-master: "../search_for_DM_LED_with_ATLAS"
%%% End:
