The most direct way to estimate the main $\znunuplusjets$ background would be
from $Z (\rightarrow \ell^+ \ell^-) +$ jets processes where $\ell = e, \mu$.
However the smaller branching ratio of the $Z$ boson decaying to leptons
compared to the neutrinos makes the $Z (\rightarrow \ell^+ \ell^-) +$ jets
statistically limited. For this reason the decay of the $Z$ boson to leptons as
a proxy to estimate the $\znunuplusjets$ process with a statistical precision of
less than 1\% can be used up to a $\pt$ of the boson of 600~GeV, beyond this
value the $W(\rightarrow \ell \nu) +$ jets processes are statistically favored
and should be used instead~\cite{VplusJetsCorrections}.

The theoretical extrapolation of the $\znunuplusjets$ background from the
$Z (\rightarrow \ell^+ \ell^-) +$ jets and $W(\rightarrow \ell \nu) +$ jets
processes at high $\pt$ can be influenced by several factor including the choice
of the parton distribution function and the renormalization and factorization
scales therefore it is important to include in the calculations higher order
correction in \gls{qcd} and \gls{ew} processes. The V + jets samples used in
this analysis are generated at the \gls{nlo} and \gls{lo} in \gls{qcd} for up to
two and four partons respectively and at the \gls{lo} in \gls{ew} using
\textsc{sherpa-2.2.1}~\cite{SHERPAGenerator}. The samples are then normalized to
an inclusive \gls{nnlo} cross section~\cite{NNLOXs}.

One possible way to include higher order corrections to the \gls{mc} samples
relies on the re-weighting of the differential cross section of these samples to
\gls{nlo} in \gls{qcd} and \gls{nlo} plus \gls{nll} Sudakov
approximation\footnote{At high energies \gls{ew} corrections are dominated by
  single and double logarithms of the scale of the process over the mass of the
  vector boson ($s/m_\mathrm{V}^2$ where V = $W, Z$) called Sudakov
  logarithms~\cite{SudakovLogs}. In the limit when all the kinematic invariants
  are much larger than the electroweak scale (the \emph{Sudakov limit}) the
  \gls{ew} one-loop corrections can be predicted in a process invariant
  way~\cite{SudakovApproximation}. This Sudakov approximation is computationally
  easier than the full \gls{nlo} \gls{ew} correction and since it includes
  leading and sub-leading logarithmic terms it is at the \gls{nll} accuracy
  level~\cite{SudakovApproximation}.} at \gls{nnlo} in \gls{ew}. The framework
proposed in Ref.~\cite{VplusJetsCorrections} for applying higher order
corrections to the \gls{mc} samples introduces a single re-weighting factor for
all the V + jets Monte Carlo simulations with a set of relative systematic
uncertainties that can be implemented as nuisance parameters in the profile
likelihood method described in \cref{sec:stat-proc}. It is based on the formula:
\begin{equation}
  \label{eq:51}
  \frac{\ud}{\ud x} \frac{\ud}{\ud \vec{y}} \sigma^{(V)}
  (\vec{\epsilon}_\mathrm{MC}, \vec{\epsilon}_\mathrm{TH}) = \frac{\ud}{\ud x}
  \frac{\ud}{\ud \vec{y}} \sigma_\mathrm{MC}^{(V)}(\vec{\epsilon}_\mathrm{MC})
  \left[ \frac{\frac{\ud}{\ud x}
      \sigma_\mathrm{TH}^{(V)}(\vec{\epsilon}_\mathrm{TH})}{\frac{\ud}{\ud x}
      \sigma_\mathrm{MC}^{(V)}(\vec{\epsilon}_\mathrm{MC})} \right]
\end{equation}
where the one dimensional parameter $x$ is in our case the transverse momentum
of the vector boson $x = \pt^{(V)}$, $\vec{y}$ represents other kinematic
variables included in the \gls{mc} simulation, $\vec{\epsilon}_\mathrm{MC}$ and
$\vec{\epsilon}_\mathrm{TH}$ are the nuisance parameters associated with the
Monte Carlo predictions and the theoretical higher order corrections introduced
with this calculation. Finally $\sigma_\mathrm{MC}^{(V)}$ and
$\sigma_\mathrm{TH}^{(V)}$ are the production cross sections for the V + jets
samples computed from \gls{mc} and from theory respectively and directly taken
from Ref.~\cite{VplusJetsCorrections}.

The left hand side of \cref{eq:51} can be interpreted as the number of V + jets
events in a given $\met = \pt(\mathrm{V})$ and jet $\pt$ bin after theory
correction. In the right hand side the terms with the MC subscript refer to
quantities determined using \textsc{sherpa-2.2.1} V + jets simulated events
while the TH subscript refers to quantities taken from
Ref.~\cite{VplusJetsCorrections}. The first factor in the right hand side of
\cref{eq:51} can be understood as the number of V + jets events in a given
$\met = \pt(\mathrm{V})$ and jet $\pt$ bin, directly taken from \textsc{sherpa}
while the last factor, which is a ratio, can be interpreted as a transfer factor
to convert the \gls{mc} \textsc{sherpa} to the theory estimate of
Ref.~\cite{VplusJetsCorrections}.

The $\ud \sigma_\mathrm{TH}(\vec{\epsilon}_\mathrm{TH})/\ud x$ is first
calculated at the \gls{lo} and subsequently corrected independently at the
\gls{nlo} in \gls{qcd} and \gls{nlo} plus the Sudakov approximation at
\gls{nnlo} in \gls{ew}:
\begin{equation}
  \label{eq:176}
  \frac{\ud}{\ud x} \sigma_\mathrm{TH}(\vec{\epsilon}_\mathrm{TH}) =
  k_\mathrm{NLO}(x, \vec{\epsilon}_\mathrm{TH}) \cdot \left(1 + k_\mathrm{EW}(x,
    \vec{\epsilon}_\mathrm{TH}) \right) \cdot \frac{\ud}{\ud x} \sigma_\mathrm{LO}
\end{equation}
where $k_\mathrm{EW} = k_\mathrm{NLO}^\mathrm{EW} +
k_\mathrm{NNLO}^\mathrm{Sudakov}$. The systematic uncertainties associated with
this theoretical procedure are described in \cref{sec:v-+-jets}.
%%% Local Variables:
%%% mode: latex
%%% TeX-master: "../search_for_DM_LED_with_ATLAS"
%%% End:
