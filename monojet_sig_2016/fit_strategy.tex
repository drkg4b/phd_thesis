The profile likelihood method described in \cref{sec:stat-proc} is used to
combine the background estimated with the different methods outlined in
\cref{sec:backgr-estim} and the relative systematic uncertainties described in
\cref{sec:syst-uncert-1,sec:ther-syst-uncert} in a global fit. The likelihood
function is the product of Poisson probability, one for each bin used in the
analysis defined in \cref{sec:event-selection-1} and Gaussian probabilities that
model the systematic uncertainties. The Poisson probabilities include as free
parameters the normalization factors mentioned in \cref{sec:glob-simult-likel}
used to scale the relative contributions of the signal and background
predictions. In the \gls{susy} search described in \cref{cha:monojet-signature}
these factors were bin dependent while in this version of the analysis a unique
normalization factor ($k^\mathrm{V + jets}$) is used to constrain all the V +
jets samples. This is justified by the use of the theoretical framework
introduced in \cref{sec:corrections-vjets}. Analogously a single normalization
factor ($k^t$) is used for the $t \bar{t}$ and single top processes. The
nuisance parameters used to model the systematic uncertainties and described in
\cref{sec:syst-uncert-1} are the free parameters in the Gaussian
distributions.

The nuisance parameters and the normalization factors are adjusted to the
optimal value by the fit which allows for a reduction of the otherwise
relatively large theoretical and experimental systematic uncertainties (of the
order of 20 - 40\% before fit). Two fitting strategies are implemented in the
analysis:
\begin{enumerate}[A -]
\item A binned global likelihood fit simultaneous to all the exclusive $\met$
  region EM1--EM10. The nuisance parameters introduced in
  \cref{sec:syst-uncert-1} are treated as bin-wise correlated systematics and
  treated accordingly in the fit. This fit takes advantage of the shape of the
  information from the $\met$ distribution in the determination of the
  normalization of the V + jets and top processes.
\item A single bin likelihood fit to the different inclusive $\met$ regions
  IM1--IM10 (a different fit for each inclusive region). The normalization
  factors $k^\mathrm{V + jets}$, $k^t$ and the nuisance parameters in this case
  refer to the specific $\met$ region.
\end{enumerate}
\cref{tab:norm_factors_2016} summarizes the normalization factors and the
control regions used for their estimation. The $k^\mathrm{V + jets}$ and $k^t$
together with the signal strength $\mu$ are the free parameters of the global
fit.
\begin{table}[!h]
  \centering
  \begin{tabular}{lll}
    \toprule
    Control Region & Background Process & Normalization Factor \\
    \midrule \midrule
    All control regions & V + jets & $k^\mathrm{V + jets}$ \\
    $\crtop$ & $t \bar{t}$, single top & $k^t$ \\
    \bottomrule
  \end{tabular}
  \caption{Summary table of the different background processes and the
    corresponding control regions used to evaluate the normalization factors.}
  \label{tab:norm_factors_2016}
\end{table}

Since sensitivity studies on the \gls{add} models performed in previous versions
of the analysis showed no sensitivity in lower $\met$ regions, the Monte Carlo
samples for \gls{add} are generated with a 150~GeV phase space cut at the truth
level on the transverse momentum of the parton and the graviton
($\pt > 150$~GeV). This cut is fully efficient only for signal regions with
$\met \gtrsim 350$~GeV. For this reason for the model independent interpretation
of the \gls{add} signal, only the $\met > 400$~GeV signal regions are
considered.
%%% Local Variables:
%%% mode: latex
%%% TeX-master: "../search_for_DM_LED_with_ATLAS"
%%% End:
