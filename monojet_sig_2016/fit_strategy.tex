The profile likelihood method described in \cref{sec:stat-proc} is used to
combine the background estimated with the different methods outlined in
\cref{sec:backgr-estim} and the relative systematic uncertainties described in
\cref{sec:syst-uncert-1,sec:ther-syst-uncert} in a global fit. The likelihood
function is the product of Poisson probabilities, one for each bin used in the
analysis defined in \cref{sec:event-selection-1} and Gaussian probabilities that
model the systematic uncertainties. The Poisson probabilities include as free
parameters the normalization factors mentioned in \cref{sec:glob-simult-likel}
used to scale the relative contributions of the signal and background
predictions. In the \gls{susy} search described in \cref{cha:monojet-signature}
these factors were bin dependent while in this version of the analysis a unique
normalization factor ($k^\mathrm{V + jets}$) is used to constrain all the V +
jets samples. This is justified by the use of the theoretical framework
introduced in \cref{sec:corrections-vjets}. Analogously a single normalization
factor ($k^t$) is used for the $t \bar{t}$ and single top processes. The
nuisance parameters used to model the systematic uncertainties and described in
\cref{sec:syst-uncert-1} are the free parameters that follow a Gaussian
distribution of mean zero and standard deviation one.

The nuisance parameters and the normalization factors are adjusted to the
optimal value by the fit which allows for a reduction of the otherwise
relatively large theoretical and experimental systematic uncertainties (of the
order of 20 - 40\% before fit as explained in \cref{sec:syst-uncert-1}). Two
fitting strategies are implemented in the analysis:
\begin{enumerate}[A -]
\item A binned global likelihood fit simultaneous to all the exclusive $\met$
  region EM1--EM10. The nuisance parameters introduced in
  \cref{sec:syst-uncert-1} are treated as bin-wise correlated systematics (the
  same nuisance parameter will be used in all the \glspl{cr} and \glspl{sr}) and
  treated accordingly in the fit. This fit takes advantage of the shape of the
  information from the $\met$ distribution in the determination of the
  normalization of the V + jets and top processes.
\item A single bin likelihood fit to the different inclusive $\met$ regions
  IM1--IM10 (a different fit for each inclusive region). The normalization
  factors $k^\mathrm{V + jets}$, $k^t$ and the nuisance parameters in this case
  refer to the specific $\met$ region. This fit is used to derive the $\met$ and
  leading jet $\pt$ distributions in
  \cref{fig:sr_et_miss_2016,fig:sr_jet1_pt_2016}.
\end{enumerate}
\cref{tab:norm_factors_2016} summarizes the normalization factors and the
control regions most constraining for their estimation. The
$k^\mathrm{V + jets}$ and $k^t$ together with the signal strength $\mu$ are the
free parameters of the global fit.
\begin{table}[!h]
  \centering
  \begin{tabular}{lll}
    \toprule
    Background Process & Normalization Factor & Most Constraining Region \\
    \midrule \midrule
    V + jets & $k^\mathrm{V + jets}$ & $\crwmn$, $\crzmm$, $\crele$ \\
    $t \bar{t}$, single top & $k^t$ & $\crtop$ \\
    \bottomrule
  \end{tabular}
  \caption{Summary table of the different background processes and the
    corresponding control regions most constraining for the normalization
    factors.}
  \label{tab:norm_factors_2016}
\end{table}

In \cref{sec:phen-add-model} it was discussed that the \gls{add} Monte Carlo
samples generated with \textsc{pythia} are simulated with a cut of 150~GeV on
the $\pt$ of the parton and the graviton. This cut is fully efficient only for
signal regions with $\met \gtrsim 350$~GeV. For this reason for the model
dependent interpretation of the \gls{add} signal, only the $\met > 400$~GeV
signal regions are considered.
%%% Local Variables:
%%% mode: latex
%%% TeX-master: "../search_for_DM_LED_with_ATLAS"
%%% End:
