The profile likelihood method described in \cref{sec:stat-proc} is used to
combine the background estimated with the different methods outlined in
\cref{sec:backgr-estim} and the relative systematic uncertainties described in
\cref{sec:syst-uncert-1,sec:ther-syst-uncert} in a global fit. The likelihood
function is the product of Poisson probability, one for each bin used in the
analysis defined in \cref{sec:event-selection-1} and Gaussian probabilities that
model the systematic uncertainties. The Poisson probabilities include as free
parameters the normalization factors mentioned in \cref{sec:glob-simult-likel}
used to scale the relative contributions of the signal and background
predictions. The free parameters of the Gaussian distributions are the nuisance
parameters used to model the systematic uncertainties described in
\cref{sec:syst-uncert-1}. The nuisance parameters and the normalization factors
are adjusted to the optimal value by the fit which allows for a reduction of the
otherwise relatively large theoretical and experimental systematic uncertainties
(of the order of 20 - 40\%).  In this version of the analysis a unique
normalization factor is used to constrain all the V + jets samples, this is
justified by the use of the theoretical framework introduced in
\cref{sec:corrections-vjets}. Analogously a single normalization factor is used
for the $t \bar{t}$ and single top processes. \cref{tab:norm_factors_2016}
summarizes the normalization factors and the control regions used for their
estimation. The $k^\mathrm{V + jets}$ and $k^t$ together with the signal
strength $\mu$ are the tree parameters of the global fit.
\begin{table}[!th]
  \centering
  \begin{tabular}{lll}
    \toprule
    Control Region & Background Process & Normalization Factor \\
    \midrule \midrule
    All control regions & V + jets & $k^\mathrm{V + jets}$ \\
    $\crtop$ & $t \bar{t}$, single top & $k^t$ \\
    \bottomrule
  \end{tabular}
  \caption{Summary table of the different background processes and the
    corresponding control regions used to evaluate the normalization factors.}
  \label{tab:norm_factors_2016}
\end{table}
\note{Review this part}
% In order to estimate the main $\znunuplusjets$ background of the analysis a
% strategy similar to the one adopted in the 2015 analysis and outlined in
% \cref{sec:glob-simult-likel} has been adopted.
%%% Local Variables:
%%% mode: latex
%%% TeX-master: "../search_for_DM_LED_with_ATLAS"
%%% End:
