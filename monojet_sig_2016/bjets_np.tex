A common way to model disagreements between data and \gls{mc} simulations in
reconstruction and identification efficiencies is to determine a correction
factor using some auxiliary dataset for \gls{mc} to data differences. These
\emph{scale factors} suffer themselves from systematic uncertainties arising
from the specific method and dataset used to derive them.
\begin{description}[font=\normalfont]
\item[syst\_FT\_EFF\_B\_systematics:] This parameter takes into account scale
  factor uncertainties in $b$-tagging (recognizing as a $b$-jet a jet
  originating from a $b$ quark).
  \item[syst\_FT\_EFF\_C\_systematics:] This parameter accounts for tagging of
    $c$-jets uncertainties.
  \item[syst\_FT\_EFF\_Light\_systematics:] This parameter models the uncertainty
    in mistagging light quarks.
  \item[syst\_FT\_EFF\_extrapolation:] This parameter accounts for uncertainties
    due to the extrapolation procedure of the data driven calibration from the
    low $\pt$ to the high $\pt$ regime~\cite{BTagCalibration}.
  \item[syst\_FT\_EFF\_extrapolation\_from\_charm:] This parameter takes into
    account the uncertainties related to the use of the charm jet scale factors
    also for the tau jets.
\end{description}
%%% Local Variables:
%%% mode: latex
%%% TeX-master: "../search_for_DM_LED_with_ATLAS"
%%% End:
