A common way to model disagreements between data and \gls{mc} simulations in
trigger and reconstruction efficiencies is to determine a correction factor from
the fit to the data to Mote Carlo ratio and use this correction as a weight in
the \gls{mc} distributions. In \gls{atlas} this weight is commonly called a
\emph{scale factor}.
\begin{description}[font=\normalfont]
\item[syst\_FT\_EFF\_B\_systematics:] This parameter takes into account scale
  factor uncertainties in $b$-tagging (recognizing as a $b$-jet a jet
  originating from a $b$ quark).
  \item[syst\_FT\_EFF\_C\_systematics:] This parameter accounts for tagging of
    $c$-jets uncertainties.
  \item[syst\_FT\_EFF\_Light\_systematics:] This parameter models the uncertainty
    in mistagging light quarks.
  \item[syst\_FT\_EFF\_extrapolation:] \note{What is extrapolation?}
  \item[syst\_FT\_EFF\_extrapolation\_from\_charm:] \note{What is extrapolation
      from charm?}
\end{description}
%%% Local Variables:
%%% mode: latex
%%% TeX-master: "../search_for_DM_LED_with_ATLAS"
%%% End:
