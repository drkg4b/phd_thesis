The search for \glspl{led} in the \gls{add} model is carried out using the $pp$
collision data collected by the \gls{atlas} experiment during 2015 and 2016
corresponding to an integrated luminosity of 36.1~$\ifb$. The events in the
signal region are selected as described in \cref{sec:event-selection} where in
addition to cut A a combination of four different triggers as described in
\cref{tab:trigger_periods} is used to select the 2016 data. Inclusive (IM1-IM10)
and exclusive (EM1-EM9) signal regions are defined in the monojet analysis with
increasing $\met$ thresholds from 250~GeV to 1000~GeV, see \cref{tab:sr_2016}
for the exact definition of the signal regions. These different $\met$ bins are
defined in order to address different BSM signals tested with the monojet
signature. In this chapter special emphasis is placed on the \gls{add} model for
\gls{led} which has been studied by the author of this thesis.
\begin{table}[!th]
  \centering
  % The @{} is to avoid extra horizontal space
  \resizebox{\linewidth}{!}{\begin{tabular}{@{}l@{}c@{}c@{}c@{}c@{}c@{}c@{}c@{}c@{}c@{}c}
    \toprule
    \multicolumn{11}{c}{Event Selection Criteria} \\
    \midrule \midrule
    \multicolumn{11}{l}{HLT\_xe70 trigger for 2015 data} \\
    \multicolumn{11}{l}{HLT\_xe80\_tc\_lcw\_L1XE50, HLT\_xe90\_mht\_L1XE50,} \\
    \multicolumn{11}{l}{HLT\_xe100\_mht\_L1XE50 and HLT\_xe110\_mht\_L1XE50 triggers for 2016 data.} \\
    \multicolumn{11}{l}{Primary Vertex} \\
    \multicolumn{11}{l}{$\met > 250~$GeV} \\
    \multicolumn{11}{l}{Leading jet with $\pt > 250~$GeV and $|\eta| < 2.4$} \\
    \multicolumn{11}{l}{At most 4 jets} \\
    \multicolumn{11}{l}{$\Delta \phi (\mathrm{jets}, \met) > 0.4$} \\
    \multicolumn{11}{l}{Jet quality requirements} \\
    \multicolumn{11}{l}{No identified muon with $\pt > 10~$GeV or electron of
    $\pt > 20~$GeV} \\
    \midrule
    Inclusive SRs & IM1 & IM2 & IM3 & IM4 & IM5 & IM6 & IM7 & IM8 & IM9 & IM10\\
    $\met$~[GeV] & > 250 & > 300 & > 350 & > 400 & > 500 & > 600 & > 700 & > 800
    & > 900 & > 1000\\
    \midrule
    Exclusive SRs & EM1 & EM2 & EM3 & EM4 & EM5 & EM6 & EM7 & EM8 & EM9 \\
    $\met$~[GeV] & [250--300] & [300--350] & [350--400] & [400--500] &
    [500--600] & [600--700] & [700--800] & [800-900] & [900-1000] \\
    \bottomrule
  \end{tabular}}
  \caption{Definition of the signal region.}
  \label{tab:sr_2016}
\end{table}
% \begin{enumerate}[A -]
% \item The HLT\_xe70 trigger was used for the 2015 data while a combination of
%   the HLT\_xe80\_tc\_lcw\_L1XE50, HLT\_xe90\_mht\_L1XE50,
%   HLT\_xe100\_mht\_L1XE50, HLT\_xe110\_mht\_L1XE50 was used for the 2016 data.
% \item Events are required to have one primary vertex with at least two
%   associated tracks with $\pt > 0.4$~GeV to assure that the event originated
%   from a $pp$ collision.
% \item The leading jet must have $\pt > 250~$GeV and $|\eta| < 2.4$ and in
%   order to reject beam-induced and cosmic particles background the event is
%   rejected if the leading jet fails the tight cleaning criteria.
% \item In order not to overlap with other ATLAS SUSY searches, events with more
%   than four jets are rejected, see \cref{sec:jet-veto} for more details on this
%   selection.
% \item The $\met$ trigger can select multi--jet events in case of a
%   mis--reconstructed jet. In these cases the missing transverse momentum points
%   in the direction of one of the jets, this background can be suppressed by
%   imposing a minimum azimuthal angle separation between the missing transverse
%   momentum and any jet of $\Delta \phi (\mathrm{jets}, \met) > 0.4$.
% \end{enumerate}
% The additional cuts used to define the different signal and control regions
% defined in the analysis are outlined in the following dedicated sections.
%%% Local Variables:
%%% mode: latex
%%% TeX-master: "../search_for_DM_LED_with_ATLAS"
%%% End:
