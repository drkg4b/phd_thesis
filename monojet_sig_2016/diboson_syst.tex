Two different generators have been used to produce the diboson \gls{mc}
samples. After taking into account possible cross section variations of the
generators the difference in yields in each $\met$ bin of the two generators is
taken as the systematic uncertainty and added in quadrature with a 6\% theory
uncertainty on the \gls{nlo} cross section. \cref{tab:diboson_sys_2016} reports
the diboson uncertainties evaluated from the comparison between the two
generators for the different exclusive regions EM1-EM10 defined in the analysis
for the signal region and the different control regions. These results are then
added in quadrature to the 6\% uncertainty on the \gls{nlo} cross section.
\begin{table}[!ht]
  \centering
  \resizebox{\linewidth}{!}{
  \begin{tabular}{lcccccccccc}
    \toprule
    \multicolumn{11}{c}{Dibson Uncertainty} \\
    \midrule \midrule
    \quad & EM1 & EM2 & EM3 & EM4 & EM5 & EM6 & EM7 & EM8 & EM9 & EM10 \\
    \midrule
    SR & 3.6 & 1.6 & 0.4 & 3.3 & 7.3 & 11.3 & 15.3 & 19.3 & 23 & 31 \\
    $\crele$ & 2.5 & 1.0 & 0.4 & 2.5 & 5.3 & 8.2 & 11.0 & 13.9 & 16.7 & 22.4 \\
    $\crwmn$ & 0.4 & 0.2 & 0.0 & 0.3 & 0.8 & 1.2 & 1.6 & 2.0 & 2.4 & 3.3 \\
    $\crtop$ & 1.2 & 2.1 & 3.1 & 4.5 & 6.5 & 8.4 & 10.3 & 12.2 & 14.2 & 18.0 \\
    $\crzmm$ & 0.9 & 0.6 & 0.2 & 0.2 & 0.8 & 1.4 & 2.1 & 2.7 & 3.3 & 4.5 \\
    \bottomrule
  \end{tabular}}
\caption{Relative systematic uncertainties in \% on the diboson background as a
  function of the exclusive bins EM1-EM10 defined in the analysis for the signal
  region and the different control regions as evaluated from the comparison
  between the generators. These values are then added in quadrature with a 6\%
  uncertainty on the \gls{nlo} cross section.}
  \label{tab:diboson_sys_2016}
\end{table}
%%% Local Variables:
%%% mode: latex
%%% TeX-master: "../search_for_DM_LED_with_ATLAS"
%%% End:
