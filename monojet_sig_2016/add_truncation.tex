As mentioned in \cref{sec:arkani-hamed-dimop} in the \gls{add} model, the
fundamental scale of the gravitational interaction, $\md$, is brought down to
the electroweak scale ($\md \approx 1~$TeV). At these energies, due to the large
center of mass energy available at \gls{lhc}, the validity of the predictions of
the \gls{eft} become unreliable. For this reason two different sets of limits
have been calculated, one where the full cross section is used and the other
where it is weighted down by a factor $\md^4/\hat{s}^2$ for events where
$\hat{s} > \md^2$, where $\hat{s}$ is the center of mass energy of the initial
partons involved in \cref{fig:add_feynman}~\cite{LEDWeightFactor}.

As an illustration of the problem \cref{fig:shat} shows the $\hat{s}$
distribution for \gls{add} events surviving a signal region selection with
$250 < \met < 300$~GeV and $700 < \met < 800$~GeV for the n = 3 and 6 extra
dimensions models. The straight line indicates the excluded $\md$ value before
any weighting of the events. This figure shows that a fraction of the selected
events in the signal regions have a value of $\hat{s}$ that exceeds the limit of
validity of the effective field theory. The general idea of weighting the events
is therefore introduced. It consists in re-evaluating the limit on $\md$ in a
conservative way, by prescribing that all events with $\hat{s}$ exceeding the
validity limit should only weakly contribute to the limit on $\md$. Moreover
\cref{fig:shat} shows that the bulk of the $\hat{s}$ distribution moves towards
higher energy values both in the case when the number of extra dimensions and
the $\met$ are increased. The first dependence is understood considering the
growth of the graviton mass with the number of extra dimensions showed in
\cref{fig:graviton_mass}: the production of gravitons with higher mass requires
higher $\hat{s}$ of the initial partons. A similar argument holds when the
$\met$ selection is increased, higher values of missing energy imply the
production of higher momentum gravitons. As shown in \cref{fig:shat}, only a
small fraction of events is beyond the \gls{eft} validity limit. The visible
event yields are recomputed in each $\met$ bin (as defined in
\cref{sec:event-selection-1}) and for each $n$ and, as hinted by
\cref{fig:shat}, the effect of the re-weighting is negligible.
\begin{figure}[!h]
  \centering
  \begin{subfigure}{.48\linewidth}
    \includegraphics[width=\linewidth]{plot_nD3_SR250}
    \caption{ADD n = 3 for the low $\met$ region.}
    \label{fig:shat_n3_250}
  \end{subfigure}
  \begin{subfigure}{.48\linewidth}
    \includegraphics[width=\linewidth]{plot_nD3_SR700}
    \caption{ADD n = 3 for the high $\met$ region.}
    \label{fig:shat_n3_700}
  \end{subfigure}
  \begin{subfigure}{.48\linewidth}
    \includegraphics[width=\linewidth]{plot_nD6_SR250}
    \caption{ADD n = 6 for the low $\met$ region.}
    \label{fig:shat_n6_250}
  \end{subfigure}
  \begin{subfigure}{.48\linewidth}
    \includegraphics[width=\linewidth]{plot_nD6_SR700}
    \caption{ADD n = 6 for the high $\met$ region.}
    \label{fig:shat_n6_700}
  \end{subfigure}
  \caption{Generated $\hat{s}$ distribution with (blue) and without (red)
    weighting of the events for which $\hat{s} > \md^2$ for the signal regions
    where $250 < \met < 300$~GeV and $700 < \met < 800$~GeV for the ADD n = 3
    and 6 models for the 13~TeV \textsc{pythia8} Monte Carlo samples. A vertical
    line indicating the value of the excluded value of $\md$ is also reported in
    the figure.}
  \label{fig:shat}
\end{figure}

\cref{fig:vis_sigma_trunc} shows the visible cross section as a function of the
fundamental Planck scale $\md$ in the $250 < \met < 300$~GeV and
$700 < \met < 800$~GeV regions for the ADD n = 3, 6 models. The solid line is
the visible cross section for all the $\hat{s}$, while in the dashed one a
$\md^4/\hat{s}^2$ weighting factor is applied for events where
$\hat{s} > \md^2$, it can be seen that for $\md = 0$ all events get weighted
down since $\hat{s} > \md^2 = 0$. The black square is the nominal $\md$ value of
the generated sample. The ratio of the dashed red curve over the blue one is
used to scale down the exclusion limits in each bin of $\met$ in order to take
into account the effect of the validity of the \gls{eft}. This effect is however
smaller than the uncertainties on the exclusions and is thus neglected.
\begin{figure}[!h]
  \centering
  \includegraphics[width=.8\linewidth]{graviton_mass}
  \caption{Simulated heavy graviton masses for 2 to 6 extra dimensions in
    \gls{add} models computed with \textsc{pythia8} using the NNPDF23LO
    \gls{pdf} set and the A14 tunes~\cite{OllePhDThesis}.}
  \label{fig:graviton_mass}
\end{figure}
\begin{figure}[!h]
  \centering
  \begin{subfigure}{.48\linewidth}
    \includegraphics[width=\linewidth]{plot_sigma_visible_nD3_SR700}
    \caption{}
    \label{fig:sigma_vis_n3}
  \end{subfigure}
  \begin{subfigure}{.48\linewidth}
    \includegraphics[width=\linewidth]{plot_sigma_visible_nD6_SR700}
    \caption{}
    \label{fig:sigma_vis_n6}
  \end{subfigure}
  \caption{Visible cross section as a function of $\md$ for the signal region
    where $250 < \met < 300$~GeV and $700 < \met < 800$~GeV for the ADD n = 3
    and 6 models. The solid line is the visible cross section for all the
    $\hat{s}$, while in the dashed one a $\md^4/\hat{s}^2$ weighting factor is
    applied for events where $\hat{s} > \md^2$. The black square is the nominal
    $\md$ value of the generated sample as listed in \cref{tab:sigma_md_ref}.}
  \label{fig:vis_sigma_trunc}
\end{figure}
%%% Local Variables:
%%% mode: latex
%%% TeX-master: "../search_for_DM_LED_with_ATLAS"
%%% End:
