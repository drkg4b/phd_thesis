As mentioned in \cref{sec:arkani-hamed-dimop} in the \gls{add} model, the
fundamental scale of the gravitational interaction, $\md$, is brought down to
the electroweak scale ($\md \approx 1~$TeV). At these energies, due to the large
center of mass energy available at \gls{lhc}, the validity of the predictions of
the \gls{eft} become unreliable. For this reason two different sets of limits
have been calculated, one where the full cross section is used and the other
where it is weighted down by a factor $\md^4/\hat{s}^2$ for events where
$\hat{s} > \md^2$, where $\hat{s}$ is the center of mass energy of the
partons~\cite{LEDWeightFactor}.

% This is implemented in the analysis by first calculating the $\hat{s}$
% distribution for each \gls{sr}

\cref{fig:shat} shows the $\hat{s}$ distribution in the 700 < $\met$ < 800~GeV
region for the ADD n = 6 model. The straight line indicates the excluded $\md$
value before any weighting of the events, a part of the generated events are in
the region where $\hat{s} > \md^2$ where the effective field theory predictions
are expected to not be reliable. \cref{fig:vis_sigma_trunc} shows the visible
cross section as a function of the fundamental Planck scale $\md$ in the 700 <
$\met$ < 800~GeV region for the ADD n = 6 model. The solid line is the visible
cross section for all the $\hat{s}$, while in the dashed one a $\md^4/\hat{s}^2$
weighting factor is applied for events where $\hat{s} > \md^2$, it can be seen
that for $\md = 0$ all events get weighted down since $\hat{s} > \md^2 = 0$. The
horizontal line is an hypothesized typical exclusion limit on the visible cross
section and it is not used in the analysis. The intersection point between the
dashed line and the horizontal one, would represent the excluded value on
$\md$. The black square is the nominal $\md$ value of the generated sample.

\begin{figure}[!h]
  \centering
  \includegraphics[width=\linewidth]{plot_nD6_SR700}
  \caption{Generated $\hat{s}$ for the signal region where 700 < $\met$ <
    800~GeV for the ADD n = 6 model. A line indicating the value of the excluded
    value of $\md$ is also reported in the figure.}
  \label{fig:shat}
\end{figure}

\begin{figure}[!h]
  \centering
  \includegraphics[width=.8\linewidth]{plot_sigma_visible_nD6_SR700}
  \caption{Visible cross section as a function of $\md$ for the signal region
    where 700 < $\met$ < 800~GeV for the ADD n = 6 model. The solid line is the
    visible cross section for all the $\hat{s}$, while in the dashed one a
    $\md^4/\hat{s}^2$ weighting factor is applied for events where
    $\hat{s} > \md^2$. The horizontal line is the upper limit on the production
    cross section if only this region would be considered in the fit. The black
    square is the nominal $\md$ value of the generated sample.}
  \label{fig:vis_sigma_trunc}
\end{figure}
%%% Local Variables:
%%% mode: latex
%%% TeX-master: "../search_for_DM_LED_with_ATLAS"
%%% End:
