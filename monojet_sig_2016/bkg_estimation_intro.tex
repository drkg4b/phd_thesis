The sources of background that can mimic the signal are described in
\cref{sec:sources-background}. The estimation of the V + jets and $t \bar{t}$
processes in the signal region is evaluated in a data driven way where
observations in regions of the phase space where an enhancement of these
processes is expected are used to improve the \gls{mc} predictions. In the
search for \gls{susy} with 3.2~$\ifb$ and described in
\cref{cha:monojet-signature} the uncertainty on top production was one of the
biggest, being approximately 30\% for $t \bar{t}$ and single top processes in
the signal region for the IM1 selection. In this version of the analysis a
dedicated top control region was introduced in order to improve the background
estimation. The new control region dedicated to the top background is described
in \cref{sec:muon-cr-bveto}. \cref{sec:muon-cr-bjet,sec:ele-cr,sec:dimuon-cr}
describe the changes, where applicable, to the remaining control regions that
were also used in the 2015 analysis.

The multi-jet and the non-collision backgrounds are estimated with the same
technique as the 2015 data analysis and described in
\cref{sec:multi-jet-background,sec:non-coll-backgr} respectively. The multi-jet
background contribution is largest in the EM1 and IM1 signal regions where it
only contributes to about 0.4\% and 0.3\% of the total background respectively
and is negligible in the other signal regions. The \gls{ncb} contribution to the
total background is negligible in all regions.
%%% Local Variables:
%%% mode: latex
%%% TeX-master: "../search_for_DM_LED_with_ATLAS"
%%% End:
