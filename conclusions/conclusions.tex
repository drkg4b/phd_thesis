The ATLAS detector is a multipurpose detector used in the search for the Higgs
boson, supersymmetry and dark matter particles. It is divided into three main
components, the inner detector, used to track charged particles and momentum
measurement, the calorimeters which measure the energy and the muon spectrometer
which provides tracking and momentum information for the muons.

The Tile Calorimeter is the hadronic calorimeter covering the most central
region of the ATLAS detector, it is used in the measurement of hadrons, jets,
taus and the missing energy. To this end, an understanding of the electronic
noise inside the detector is crucial as it affects the signal left by the
particles crossing the calorimeter and must be known in order to determine which
energy deposits in the calorimeter are significant and should be used in the
reconstruction of jets, electrons and taus. Part of the work presented in this
thesis was to update, monitor and study the noise calibration constants for the
Tile Calorimeter to allow for processing of data and the identification of
hadronic jets. During these studies unexpected variations over time in the cell
noise were observed. Further investigation led to discover that the tile noise
filter, an algorithm used to mitigate the effect of coherent noise, was not
behaving as expected in some situations significantly affecting approximately
5\% of the cells in the TileCal.

The results of the search for new physics phenomena in a monojet with large
missing transverse momentum using the data from $pp$ collisions at LHC collected
by the ATLAS experiment during 2015 and corresponding to an integrated
luminosity of 3.2~$\ifb$ has been presented. For the first time the monojet
analysis was used to set limits on compressed light squark--neutralino
models. Selections adapted to the specific characteristics of this signal were
studied. It was shown that the generic approach provided by the global fit
provides better sensitivity to this new signal than a single signal region with
asymmetric jet and $\met$ cuts. Several jet veto criteria were studied since the
BSM signal studied in this thesis is the most sensitive to this type of cuts and
considered in the monojet analysis. It was found that cut F in
\cref{sec:event-selection} provides pile--up independent jet veto efficiency
retaining most of the signal. No significant excess in the data compared to the
SM predictions has been found thus the results have been translated into model
independent upper limits on the visible cross section with a 95\% CL and
interpreted in terms of squark pair production with $\squarkprod$. Squark masses
up to 608~GeV are excluded. This study extends limits shown in
\cref{fig:susy_exclusion} that are provided by the more classical approach to
SUSY searches.

Naturalness suggests that supersymmetric standard model partners are expected at
the TeV scale, with the high luminosity foreseen in 2016 it will be possible to
test experimentally the squark pair production with masses at that scale.
%%% Local Variables:
%%% mode: latex
%%% TeX-master: "../search_for_DM_LED_with_ATLAS"
%%% End:
