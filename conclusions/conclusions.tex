The \gls{atlas} detector is a multipurpose detector used for precision
measurements of a wide range of Standard Model processes and in the search for
new decay channels of the Higgs boson and beyond the Standard Model phenomena
such as supersymmetry, extra dimensions or dark matter particles.

The Tile Calorimeter is the hadronic calorimeter covering the most central
region of the \gls{atlas} detector up to $|\eta| < 1.7$. It plays a vital role
in measuring the energy of jets and missing energy used in this thesis to search
for \gls{susy} and extra dimensions. To this end, an understanding of the
electronic noise inside the detector is crucial as it affects the signal left by
the particles crossing the calorimeter and must be known precisely. The exact
noise level per calorimeter cell is used as an input to the algorithm that
identifies significant calorimeter energy deposits and then identifies and
measure jets, electrons and taus. Part of the work presented in this thesis was
to update, monitor and study the noise calibration constants for the Tile
Calorimeter to allow for precise reprocessing of the 2011 data and the
identification of hadronic jets. During these studies unexpected variations over
time in the cell noise were observed. Further investigation led to discover that
the tile noise filter, an algorithm used to mitigate the effect of coherent
noise, was not behaving as expected in some situations significantly affecting
approximately 5\% of the cells in the TileCal.

This thesis presented two searches for \gls{bsm} physics in \gls{atlas}. The
first one used the 2015 data corresponding to an integrated luminosity of
3.2~$\ifb$ where for the first time the monojet analysis was used to set limits
on compressed light squark-neutralino models. Selections adapted to the specific
characteristics of this signal were studied. It was shown that the generic
approach provided by the global fit gives better sensitivity to this new signal
than a single signal region with asymmetric jet and $\met$ cuts. Several jet
veto criteria were considered since the \gls{bsm} compressed \gls{susy} scenario
studied in this thesis is the most sensitive to this type of cuts and considered
in the monojet analysis. No significant excess in the data compared to the
\gls{sm} predictions has been found thus the results have been translated into
model independent upper limits on the visible cross section with a 95\% \gls{cl}
and interpreted in terms of squark pair production with $\squarkprod$. Squark
masses up to 608~GeV are excluded. The study on the 2015 dataset extends limits
shown in \cref{fig:susy_exclusion} that are provided by the more classical
approach to \gls{susy} searches.

The second analysis presented in this thesis uses the combined 2015 + 2016
dataset corresponding to an integrated luminosity of 36.1~$\ifb$ which was used
to explore the presence of large extra dimensions in the \gls{add} model
scenario. A good agreement between data and Standard Model prediction is
observed and no significant excess in data is seen thus the results are
translated in 95\%~\gls{cl} upper limit on the effective Planck scale
$\md$. Values of $\md$ of 7.7 and 4.8~TeV for two and six extra dimension
hypotheses are excluded significantly improving the results obtained by the
previous version of the analysis. Future versions of this analysis will take
advantage of an increased dataset, already during 2017 the data collected by the
\gls{atlas} detector amounts at approximately 70~$\ifb$ almost doubling the one
on which this thesis is based. From \cref{eq:181,eq:117} it is possible to
estimate that even with twice as much luminosity the excluded effective Planck
scale $\md$ only increase approximately by a factor $\sqrt{2}^{1/4}$ for $n = 2$
extra dimensions.
% Naturalness suggests that supersymmetric standard model partners are expected at
% the TeV scale, with the high luminosity foreseen in 2016 it will be possible to
% test experimentally the squark pair production with masses at that scale.
%%% Local Variables:
%%% mode: latex
%%% TeX-master: "../search_for_DM_LED_with_ATLAS"
%%% End:
