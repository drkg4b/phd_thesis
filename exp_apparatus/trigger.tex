The bunch crossing rate at LHC is 40~MHz for a bunch spacing of 25~ns (about 7
meters). Each event recorded by ATLAS requires $\approx 1.4$~MB of disk space,
with approximately 20 to 50 collisions per bunch crossing, the storage space
required to record all the events in a second would be $\approx 60$~TB. This is
not feasible thus only the most interesting events are selected and stored on
disk. The \emph{trigger system} decides whether to keep or not a collision event
for later studies, it consists of a hardware based \gls{l1} trigger and a
software based \gls{hlt}.

The L1 trigger determines \gls{rois} in the detector using custom hardware and
coarse information from the calorimeter and the muon system. The L1 trigger is
capable of reducing the event rate to 100~kHz with a decision time for a L1
accept of 2.5~$\mu$s. The RoIs from the L1 trigger are sent to the HLT where
different algorithms are run using the full detector information and reducing
the L1 output rate to 1~kHz with a processing time of about
200~ms~\cite{trigger}. %A schematic overview of the ATLAS trigger and data
%acquisition system is shown in Figure~\ref{fig:trigger_system}.

In the monojet analysis presented in this thesis, the HLT\_xe70 trigger is used,
it receives an L1 accept that selects events with a missing energy (see
Section~\ref{sec:miss-transv-energy}) greater than 50~GeV, no muons are used in
the reconstruction of the missing energy at the trigger level. The events that
survive L1 are then passed to the HLT level, where events with a missing energy
greater than 70~GeV are selected.
\begin{table}[!hb]
  \centering
  \begin{tabular}{lcc}
    \toprule
    \multicolumn{3}{c}{Trigger Used In The 2016 Data Taking} \\
    \midrule \midrule \\
    Run Range & Trigger & $\langle \mu \rangle$ \\
    296939-302393 & HLT\_xe90\_mht\_L1XE50 or HLT\_xe80\_tc\_lcw\_L1XE50 & 33.1 \\
    302737-302872 & HLT\_xe90\_mht\_L1XE50 & 34 \\
    302919-304008 & HLT\_xe100\_mht\_L1XE50 or HLT\_xe110\_mht\_L1XE50 & 35.4 \\
    304128-310216 & HLT\_xe110\_mht\_L1XE50 & 51.1 \\
    \bottomrule
  \end{tabular}
  \caption{The table reports the trigger used in the 2016 data taking. The
    average number of interactions per bunch crossing, $\langle \mu \rangle$, is
  also reported. The constant increasing of $\langle \mu \rangle$ justifies the
  choice of higher $\met$ thresholds for the trigger.}
  \label{tab:trigger_periods}
\end{table}
% \begin{figure}[!h]
%   \centering
%     \includegraphics[width=.7\linewidth]{trigger_system}
%     \caption{Schematic view of the ATLAS trigger and data acquisition system.}
%     \label{fig:trigger_system}
% \end{figure}
%%% Local Variables:
%%% mode: latex
%%% TeX-master: "../search_for_DM_LED_with_ATLAS"
%%% End:
