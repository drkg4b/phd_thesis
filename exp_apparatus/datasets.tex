In the 2015 monojet analysis presented in this thesis, the HLT\_xe70 trigger is
used, it receives an L1 accept that selects events with a missing energy (see
\cref{sec:miss-transv-energy}) greater than 50~GeV computed at the L1 trigger.
No muons are used in the reconstruction of the missing energy at the trigger
level. The events that survive L1 are then passed to the \gls{hlt} level, where
events with a missing energy (calculated from cell energy information) greater
than 70~GeV are selected.

In the 2016 analysis due to the increased luminosity and pile-up (see
\cref{sec:primary-vertex}) the trigger had to be modified. A combination of four
different triggers is used, they all pass the level one trigger selection if the
event have a missing energy of more than 50~GeV, again no muon information is
used in the missing energy reconstruction at this level. For the \gls{hlt} the
missing energy threshold is increased in steps of 10~GeV from 80~GeV up to
110~GeV for the four different triggers. The information on the missing energy
takes advantage of the \gls{mht}~\cite{MHTAlgorithm} algorithm or the
\gls{lcw}~\cite{LCWCalibration} calibration scheme depending on the trigger
used. In the \gls{mht} algorithm the missing energy is calculated as the
negative vector sum of the transverse momentum
($- \sum \pt^\mathrm{\, jets} \equiv - H_\mathrm{\, T}$) of all the jets
reconstructed with the $\antikt$ jet finding algorithm (see \cref{sec:anti-k_t})
from topoclusters (see \cref{sec:topocluster}). The \gls{lcw} calibration scheme
uses the shape of the cluster and the energy density to classify the topocluster
energy deposits as electromagnetic or hadronic improving the resolution of the
missing energy trigger.

\cref{fig:lumi_summary} shows the delivered luminosity as a function of the
average number of interactions per bunch crossing $\langle \mu \rangle$ during
the 2015 and 2016 data taking periods. \cref{tab:trigger_periods} summarizes the
different trigger combination for the 2016 dataset used in the analysis. The
increase with time of the maximum instantaneous luminosity $\mathcal{L}^{max}$
forced the use of a higher missing energy threshold in the
trigger. \cref{fig:trigger_efficiency} shows the trigger efficiency for the four
different data taking periods in the 2016 analysis as a function of the missing
energy as evaluated offline (i.e.\@ the same variable used in the analyses)
estimated using $\wmunuplusjets$ events. It can be seen that the trigger is
fully efficient from approximately 200~GeV. The monojet analyses presented in
\cref{cha:monojet-signature,cha:2016-monoj-analys} require a missing transverse
energy of at least 250~GeV and are therefore not affected by the loss of
efficiency below 200~GeV.
% receive an L1 accept for events with more than 50~GeV of missing energy, with no
% muon information for the missing energy reconstruction at this level.
% HLT\_xe80\_tc\_lcw\_L1XE50, HLT\_xe90\_mht\_L1XE50,
% HLT\_xe100\_mht\_L1XE50 and HLT\_xe110\_mht\_L1XE50. The
% HLT\_xe80\_tc\_lcw\_L1XE50 at L1 level selects events with a missing energy of
% more than 50~GeV while at the \gls{hlt} level it uses the \gls{lcw} to calibrate
% the missing energy whose information is taken from the energy deposited in the
% topocluster (tc) (see \cref{sec:topocluster}). The \gls{lcw} uses the
\begin{table}[!ht]
  \centering
    \resizebox{\linewidth}{!}{\begin{tabular}{lcc}
    \toprule
    \multicolumn{3}{c}{Trigger Used in the 2016 Analysis} \\
    \midrule \midrule
    Run Range & Trigger & $\mathcal{L}^{max} (10^{30} cm^{-2}s^{-1})$ \\
    \midrule
    296939-302393 & HLT\_xe90\_mht\_L1XE50 or HLT\_xe80\_tc\_lcw\_L1XE50 & 8761 \\
    302737-302872 & HLT\_xe90\_mht\_L1XE50 & 9854 \\
    302919-304008 & HLT\_xe100\_mht\_L1XE50 or HLT\_xe110\_mht\_L1XE50 & 10261 \\
    304128-310216 & HLT\_xe110\_mht\_L1XE50 & 13716 \\
    \bottomrule
  \end{tabular}}
\caption{The table reports the trigger used in the 2016 data taking. The maximum
  instantaneous luminosity $\mathcal{L}^{max} (10^{30} cm^{-2}s^{-1})$ is also
  reported. The constant increasing of $\mathcal{L}^{max}$ justifies the choice
  of higher $\met$ thresholds for the trigger.}
  \label{tab:trigger_periods}
\end{table}
% \begin{table}[!ht]
%   \centering
%   \begin{tabular}{lcc}
%     \toprule
%     \multicolumn{3}{c}{Trigger Used In The 2016 Data Taking} \\
%     \midrule \midrule
%     Run Range & Trigger & $\langle \mu \rangle$ \\
%     \midrule
%     296939-302393 & HLT\_xe90\_mht\_L1XE50 or HLT\_xe80\_tc\_lcw\_L1XE50 & 33.1 \\
%     302737-302872 & HLT\_xe90\_mht\_L1XE50 & 34 \\
%     302919-304008 & HLT\_xe100\_mht\_L1XE50 or HLT\_xe110\_mht\_L1XE50 & 35.4 \\
%     304128-310216 & HLT\_xe110\_mht\_L1XE50 & 51.1 \\
%     \bottomrule
%   \end{tabular}
%   \caption{The table reports the trigger used in the 2016 data taking. The
%     average number of interactions per bunch crossing, $\langle \mu \rangle$, is
%   also reported. The constant increasing of $\langle \mu \rangle$ justifies the
%   choice of higher $\met$ thresholds for the trigger.}
%   \label{tab:trigger_periods}
% \end{table}
\begin{figure}[!ht]
  \centering
    \includegraphics[width=\linewidth]{atlas_lumi_summary}
    \caption{Delivered luminosity as a function of the average number of
      interactions per bunch crossing $\langle \mu \rangle$ during the 2015 and
      2016 data taking periods at $\sqrt{s} = 13$~TeV~\cite{LumiSummaryPlots}.}
    \label{fig:lumi_summary}
\end{figure}
\begin{figure}[!ht]
  \centering
    \includegraphics[width=\linewidth]{trigger_efficiency}
    \caption{Trigger efficiency curve as a function of the missing energy for
      the 2016 dataset of 32.9~$\ifb$ estimated in $\wmunuplusjets$ events. The
      four different periods correspond to the run intervals defined in
      \cref{tab:trigger_periods}. The trigger is fully efficient from
      approximately 200~GeV.}
    \label{fig:trigger_efficiency}
\end{figure}
%%% Local Variables:
%%% mode: latex
%%% TeX-master: "../search_for_DM_LED_with_ATLAS"
%%% End:
