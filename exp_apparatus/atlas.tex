\gls{atlas} is a multi purpose detector designed to be sensitive to a large
physics signatures (supersymmetry and dark matter, briefly introduced in
Section~\ref{sec:supersymmetry} and Section~\ref{sec:dark-matter}) and to fully
take advantage of the LHC potential. It is capable of identifying photons,
electrons, muons, taus, jets and missing energy,
Figure~\ref{fig:atlas_particles} shows a schematic view of the interaction of
the different kind of particles with the ATLAS sub-detectors while
Figure~\ref{fig:atlas_overview} shows the ATLAS detector with its subsystems. In
the following sections a brief overview of the various subsystems that allow
particle identification and reconstruction is presented.

\begin{figure}[!h]
  \centering
    \includegraphics[width=.8\linewidth]{atlas_particles}
    \caption{Section of the ATLAS detector showing the interaction of different
      particle types with the sub--detectors~\cite{ATLASCrossSection}.}
    \label{fig:atlas_particles}
\end{figure}
\begin{figure}
  \centering
    \includegraphics[width=.8\linewidth]{atlas_overview}
    \caption{Overview of the ATLAS detectors with its main
      sub-detectors~\cite{ATLASPaper}.}
    \label{fig:atlas_overview}
\end{figure}
%%% Local Variables:
%%% mode: latex
%%% TeX-master: "../search_for_DM_LED_with_ATLAS"
%%% End:
