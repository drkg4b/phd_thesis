A \gls{cr} is a region of the phase space where the signal contribution is
negligible but the event selections are similar to those of the signal
region. The main reason to define control regions is to check the agreement in
shape and normalization between MC simulations and data in reconstructed
kinematic quantities. The $V$ + jets, where $V$ is either a $W$ or a $Z$ vector
boson, constitutes the main background of the monojet analysis. A pure MC
estimation of these processes, suffers from theoretical uncertainties like the
limited knowledge of the \glspl{pdf}, limited order accuracy of the Monte Carlo
generators and experimental uncertainties related to the jet energy scale and
luminosity determination. In order to estimate the contribution of these
backgrounds in the SR, a \emph{data driven} technique is used. The method aims
at reducing the systematic uncertainties by relying on information from data in
the CRs rather than on MC simulations. It can be divided in three major steps:
\begin{itemize}
\item Define CRs to select $V$ + jets event in data.
\item Calculate a transfer factor from MC predicted events in the CR to
  background estimates in the SR\@.
\item Apply the transfer factor to the observed events in the CR to obtain the
  estimate number of events from the process in the SR\@.
\end{itemize}
The CRs used to constrain the $V$ + jets backgrounds have an event selection
that differs from the SR only in the lepton veto and the missing transverse
momentum calculation. They are thus orthogonal to the SR and a minimum
contribution from a monojet-like signal is expected.
%%% Local Variables:
%%% mode: latex
%%% TeX-master: "../search_for_DM_LED_with_ATLAS"
%%% End:
