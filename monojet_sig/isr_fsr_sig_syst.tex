\gls{mb} events are soft ($\pt < 2$~GeV) inelastic interactions which constitute
a significant fraction of the hadron-hadron collisions. Understanding \gls{mb}
processes is therefore important for precision measurements of hard scatter
interactions. Perturbative \gls{qcd} is commonly used to describe parton
interactions when possible and non-perturbative \gls{qcd} for low $\pt$
processes. The free parameters of these models (\emph{tune parameters}) are
tuned in the \gls{mc} generators in order to replicate experimental data. The
variation of these parameters may lead to different number of generated events
in the \gls{mc} samples and thus to uncertainty in the acceptance. Following the
\gls{atlas} recommendation given in Ref.~\cite{MCTuningRecommendations2015} five
tune parameters have been varied in order to account for uncertainties from:
\begin{itemize}
\item Underlying event effects.
\item Jet structure effects.
\item Aspects of the \gls{mc} generation that provide extra jet production.
\end{itemize}

The presence of large numbers of high $\pt$ jets in the final state could hide
signal for new physics that involves the decay of heavy colored particles. In
order to control this kind of events the theory calculations need to describe in
a precise way the matrix element used to describe the hard process and the
subsequent parton shower into jets of hadrons in the calorimeter. A particular
N-jet event can arise from soft-radiation evolution of a N-parton final state
(matrix element jets) or from an (N-1)-parton configuration where the extra jets
appears from from hard emission during the shower evolution (parton shower
jets). A \emph{matching scheme} is defined on an event-by-event basis in order
to decide which process originated the N-jet event avoiding in this way double
counting of events~\cite{Matching}. Due to the wide range of the energy scales
spanned by the jets transverse momentum there are different scales considered in
the possible overlap between parton shower or matrix element jets. These scales
used in the matching scheme are referred to as \emph{matching scales}. The
\gls{susy} compressed signal samples are produced with
\textsc{madgraph}~\cite{MADGRAPH}, thus an additional matching scale systematic
uncertainty is considered. % set of systematic uncertainties
% need to be considered. The following uncertainties are considered at the MADGRPH
% level:
% \begin{itemize}
% \item Renormalization and factorization scale varied by a factor 0.5 and 2.
% \item Initial state radiation: vary the scale for the $gqq$ and $ggg$ vertices,
% \item Jet matching: vary the qcut for the CKKW matching 0.5, 1, 2
% \end{itemize}
For each squark mass 16 \gls{mc} samples are produced and compared at the truth
level. The resulting envelope of the systematic uncertainties affects the signal
acceptance and are added in quadrature, the results are presented in
\cref{tab:susy_tune}.
\begin{table}[!h]
  \centering
  \resizebox{\linewidth}{!}{\begin{tabular}{lccccccc}
    \toprule
    \multicolumn{8}{c}{Initial and Final State Radiation} \\
    \midrule \midrule
    $\met$ [GeV] & 250-300 & 300-350  & 350-400 & 400-500 & 500-600 & 600-700
 & > 700 \\
    \midrule
    $\Delta A$ & 11 & 18 & 11 & 8 & 7 & 9 & 8 \\
    \bottomrule
  \end{tabular}}
\caption{Theoretical uncertainty in \% on the \gls{susy} compressed spectra
  signal region acceptance as function of the $\met$ bin in the signal region,
  from tune, matching, initial and final state radiation systematic
  uncertainties. The final value is a common envelope valid for all the
  \gls{susy} compressed models.}
  \label{tab:susy_tune}
\end{table}
%%% Local Variables:
%%% mode: latex
%%% TeX-master: "../search_for_DM_LED_with_ATLAS"
%%% End:
