\gls{mb} events are inelastic $pp$ collisions characterized by low momentum
exchange, typically below approximately 2~GeV. They make up the vast majority of
the \gls{lhc} collisions. Proper Monte Carlo modeling of \gls{mb} processes is
therefore important for for proper prediction of event yields in signal and
control regions. Perturbative \gls{qcd} is commonly used to describe parton
interactions when possible and non-perturbative \gls{qcd} for low $\pt$
processes. The free parameters of these models (\emph{tune parameters}) are
tuned in the \gls{mc} generators in order to replicate experimental \gls{mb}
data. The variation of these parameters may lead to different number of
generated events in the \gls{mc} samples or different kinematic characteristics
of the simulated events and thus to an uncertainty on the signal and background
predictions. Following the \gls{atlas} recommendation given in
Ref.~\cite{MCTuningRecommendations2015} five tune parameters have been varied in
order to account for uncertainties from:
\begin{itemize}
\item Underlying event effects.
\item Jet structure effects.
\item Aspects of the \gls{mc} generation that provide extra jet production.
\end{itemize}

The presence of large numbers of high $\pt$ jets in the final state could hide
signal for new physics that involves the decay of heavy colored particles. In
order to control this kind of events the theory calculations need to describe in
a precise way the matrix element used to describe the hard process and the
subsequent parton shower into jets of hadrons in the calorimeter. A particular
N-jet event can arise from soft-radiation evolution of a N-parton final state
(matrix element jets) or from an (N-1)-parton configuration where the extra jet
appears from from hard emission during the shower evolution (parton shower
jet). A \emph{matching scheme} is defined on an event-by-event basis in order to

avoid double counting and decide which jets should be taken from the matrix
element and which ones should be taken from parton shower~\cite{Matching}. The
\gls{susy} compressed signal samples are produced with
\textsc{madgraph}~\cite{MADGRAPH}, thus an additional matching scale systematic
uncertainty is considered. % set of systematic uncertainties
% need to be considered. The following uncertainties are considered at the MADGRPH
% level:
% \begin{itemize}
% \item Renormalization and factorization scale varied by a factor 0.5 and 2.
% \item Initial state radiation: vary the scale for the $gqq$ and $ggg$ vertices,
% \item Jet matching: vary the qcut for the CKKW matching 0.5, 1, 2
% \end{itemize}


Practically the matching, tune, initial and final state uncertainties are
estimated by first generating 16 alternative \gls{mc} samples for each squark
mass with different settings and evaluating the relative variation in number of
events in the signal regions as observed at the Monte Carlo truth level. For a
given squark and neutralino mass, the different relative variations are added in
quadrature. Finally the envelope of the systematics obtained over all squark and
neutralino mass is derived in order to obtain a single systematic uncertainty
for all \gls{susy} compressed mass points. The results are presented in
\cref{tab:susy_tune}.
\begin{table}[!htb]
  \centering
  \resizebox{\linewidth}{!}{\begin{tabular}{lccccccc}
    \toprule
    \multicolumn{8}{c}{Initial and Final State Radiation} \\
    \midrule \midrule
    $\met$ [GeV] & 250-300 & 300-350  & 350-400 & 400-500 & 500-600 & 600-700
 & > 700 \\
    \midrule
    $\Delta A$ & 11 & 18 & 11 & 8 & 7 & 9 & 8 \\
    \bottomrule
  \end{tabular}}
\caption{Theoretical uncertainty in \% on the \gls{susy} compressed spectra
  signal region acceptance as function of the $\met$ bin in the signal region,
  from tune, matching, initial and final state radiation systematic
  uncertainties. The final value is a common envelope valid for all the
  \gls{susy} compressed models.}
  \label{tab:susy_tune}
\end{table}
%%% Local Variables:
%%% mode: latex
%%% TeX-master: "../search_for_DM_LED_with_ATLAS"
%%% End:
