A pure MC based prediction of the background contamination in the signal region
yield large theoretical and experimental systematic uncertainties in the shape
and normalization of the predicted distributions. For this reason a data driven
approach, where the different backgrounds are normalized using the data in CRs
that are orthogonal to the SR, is used. To extrapolate from CR to SR prediction
an MC based \emph{transfer factor} is used. This factor is a ratio of MC
predictions thus most of the systematic uncertainties either cancel out or are
significantly reduced. The extrapolation from one kinematics region to another
can be itself source of uncertainties, to avoid this, the selection in terms of
kinematic in the CRs and SRs is chosen to be the same. The expected
contributions of a background process BG$_i$, which is extracted from a control
region CR$_j$ to a signal region SR$_k$, is
given by:
\begin{equation}
  \label{eq:83}
  N_{\mathrm{\, BG}_i}^{\mathrm{\, SR}_k} = \frac{(N_\mathrm{\,
      data}^{\mathrm{\, CR}_j} - N_{\mathrm{\,
        non~BG}_i,~\mathrm{MC}}^{\mathrm{\, CR}_j})}
  {N^{\mathrm{\, CR}_j}_{\mathrm{\, BG}_i \mathrm{,~MC}}} \times
  N^{\mathrm{\, SR}_k}_{\mathrm{\, BG}_i,~\mathrm{MC}}
\end{equation}
where $N_{\mathrm{\, BG}_i}^{\mathrm{\, SR}_k}$ is the predicted number events
for background $i$ events in the signal region SR$_k$,
$N_\mathrm{\, data}^{\mathrm{\, CR}_j}$ is the observed number of events in the
control region $j$, $N_{\mathrm{\, non~BG}_i,~\mathrm{MC}}^{\mathrm{\, CR}_j}$
is the estimated number of the background contamination coming from other
processes in the given control region,
$N^{\mathrm{\, CR}_j}_{\mathrm{\, BG}_i \mathrm{,~MC}}$ is the MC prediction of
the $i$ background in the $j$ control region and
$N^{\mathrm{\, SR}_k}_{\mathrm{\, BG}_i,~\mathrm{MC}}$ is the number of the $i$
background events predicted in MC simulation in the signal region. The ratio:
\begin{equation}
  \label{eq:84}
  C_{\mathrm{BG}_i}^{\mathrm{CR}_j \rightarrow \mathrm{SR}_k} = \frac{
    N^{\mathrm{\, SR}_k}_{\mathrm{\, BG}_i,~\mathrm{MC}}}{N^{\mathrm{\,
        CR}_j}_{\mathrm{\, BG}_i \mathrm{,~MC}}}
\end{equation}
is the transfer factor used to extrapolate from CR$_j$ to SR$_k$. The
normalization factors used in the normalization of the BG expectation in the SRs
is given by:
\begin{equation}
  \label{eq:85}
  \mu_{j} = \frac{(N_\mathrm{\, data}^{\mathrm{\, CR}_j} - N_{\mathrm{\,
        non~BG}_i,~\mathrm{MC}}^{\mathrm{\, CR}_j})}{N^{\mathrm{\,
        CR}_j}_{\mathrm{\, BG}_i \mathrm{,~MC}}}.
\end{equation}
The different normalization factors are not independent since different
processes can enter several CRs and thus the background subtraction term,
$N_{\mathrm{\, non~BG}_i,~\mathrm{MC}}^{\mathrm{\, CR}_j}$, gets contributions
from the other normalization factors. To properly treat the correlations, the
normalization factors are obtained from a simultaneous fit of all the CRs
referred to as the \emph{global fit}. A summary of the different processes and
the corresponding CR used to extract the normalization factor is given in
Table~\ref{tab:norm_factors}.
\begin{table}[!th]
  \centering
  \begin{tabular}{lll}
    \toprule
    Control Region & Background Process & Normalization Factor \\
    \midrule \midrule
    CR$_\mathrm{\, wmn}$ & $W (\rightarrow \mu \nu),\, Z (\rightarrow \nu \bar{\nu})$ & $\mu_\mathrm{\, wmn}$ \\
    CR$_\mathrm{\, ele}$ & $W (\rightarrow e \nu),\, W (\rightarrow \tau \nu),\,
                           Z/\gamma^* (\rightarrow \tau^+ \tau^-)$ & $\mu_\mathrm{\, ele}$ \\
    CR$_\mathrm{\, zmm}$ & $Z/\gamma^* (\rightarrow \mu^+ \mu^-)$ & $\mu_\mathrm{\, zmm}$ \\
    \bottomrule
  \end{tabular}
  \caption{Summary table of the different background processes and the
    corresponding control regions used to evaluate the normalization factors.}
  \label{tab:norm_factors}
\end{table}

The global fit can be performed in two different ways:
\begin{enumerate}
\item The background only hypothesis, fits only the control regions in order to
  predict the background in the signal region. This fit is used to set model
  independent limits.
\item The signal plus background hypothesis, fits both the signal and control
  regions with a sum of background and specific signal. The normalization of the
  specific BSM signal is a free parameter. This fit is used to exclude specific
  models.
\end{enumerate}
%%% Local Variables:
%%% mode: latex
%%% TeX-master: "../search_for_DM_LED_with_ATLAS"
%%% End:
