The \emph{p-value} is the quantity used in the \gls{lhc} frequentist
approach~\cite{StatProcedure} to quantify the level of agreement between the
observed data and some hypothesis $H$. Assuming that $H$ is true, the p-value is
defined as the probability to observe data of equal or greater incompatibility
with $H$. In order to \emph{discover} a new physics signal, the \emph{null
  hypothesis}, $H_0$, is defined to include only known \gls{sm} processes
(background) and tested against an \emph{alternate hypothesis}, $H_1$, that
describes both the looked after signal and the background. If the signal cannot
be discovered the roles of the two hypotheses are inverted (the \emph{signal plus
  background} hypothesis becomes the null hypothesis and the \emph{background
  only} hypothesis becomes the alternate one) and \emph{exclusion limits} are
set instead. The conventionally accepted p-value to claim a discovery is set at
$2.87 \times 10^{-7}$ while to exclude a signal hypothesis a threshold p-value
of 0.005, often quoted as 95\%~\gls{cl}, is used.

In an experiment that measures a variable $x$ for each event in the signal
sample and constructs an histogram $\boldsymbol{n} = (n_1, \dots, n_N)$ out of
these measurements, the expectation value of $n_i$ can be written as:
\begin{equation}
  \label{eq:88}
  E[n_i] = \mu s_i + b_i
\end{equation}
where $s_i$ and $b_i$ are the mean number of signal and background entries in
the $i$-th bin and $\mu$ is the \emph{signal strength} parameter of the signal
process with $\mu = 0$ indicating the background only hypothesis and $\mu = 1$
the nominal signal one. The parameters of the physical model, unknown properties
of the response of the detector as well as other parameters that are not of
interest for the physics search are called \emph{nuisance parameters} and
indicated with $\boldsymbol{\theta}$. These can be evaluated constructing
histograms of some kinematic variable of interest in a control region where
mostly background events are expected. Indicating as
$\boldsymbol{m} = (m_1, \dots, m_M)$ one of such histograms, the expectation
value of the $i$-th bin, $m_i$, is given by:
\begin{equation}
  \label{eq:89}
  E[m_i] = u_i(\boldsymbol{\theta})
\end{equation}
where $u_i$ are calculable quantities that depend on the nuisance parameters
$\boldsymbol{\theta}$. With these definitions, the \emph{likelihood} function
can be written as the product of Poisson probabilities for the different bins:
\begin{equation}
  \label{eq:115}
  L(\mu, \boldsymbol{\theta}) = \prod_{j = 1}^N \frac{(\mu s_j +
    b_j)^{n_j}}{n_j!} e^{-(\mu s_j + b_j)} \prod_{k = 1}^M
  \frac{u_k^{m_k}}{m_k!} e^{-u_k}
\end{equation}

The p-value is determined at \gls{lhc} using the \emph{test
  statistics}~\cite{StatProcedure}:
\begin{equation}
  \label{eq:86}
  t_\mu = - 2 \ln \lambda(\mu)
\end{equation}
where $\lambda(\mu)$ is the \emph{profile likelihood ratio} defined as:
\begin{equation}
  \label{eq:87}
  \lambda(\mu) = \frac{L(\mu, \hat{\hat{\boldsymbol{\theta}}})}{L(\hat{\mu},
    \hat{\boldsymbol{\theta}})}.
\end{equation}
Here $\hat{\mu}$ and $\hat{\boldsymbol{\theta}}$ are the maximum likelihood
estimators, $L(\hat{\mu}, \hat{\boldsymbol{\theta}})$ is the maximized likelihood
function and $\hat{\hat{\boldsymbol{\theta}}}$ is the value of the parameters
$\boldsymbol{\theta}$ that maximizes the likelihood for a fixed value of the
signal strength $\mu$.
%%% Local Variables:
%%% mode: latex
%%% TeX-master: "../search_for_DM_LED_with_ATLAS"
%%% End:
