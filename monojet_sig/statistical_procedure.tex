The \emph{p-value} is the quantity used in the \gls{lhc} frequentist
approach~\cite{StatProcedure} to quantify the level of agreement between the
observed data and some hypothesis $H$. Assuming that $H$ is true, the p-value is
defined as the probability to observe data of equal or greater incompatibility
with $H$. In order to \emph{discover} a new physics signal, the \emph{null
  hypothesis}, $H_0$, is defined to include only known \gls{sm} processes
(background) and tested against an \emph{alternate hypothesis}, $H_1$, that
describes both the looked after signal and the background. If the signal cannot
be discovered the roles of the two hypotheses are inverted (the \emph{signal
  plus background} hypothesis becomes the null hypothesis and the
\emph{background only} hypothesis becomes the alternate one) and \emph{exclusion
  limits} are set instead. The conventionally accepted p-value to claim a
discovery is set at $2.87 \times 10^{-7}$ while to exclude a signal hypothesis a
threshold p-value of 0.05 is used.

In an experiment that measures a variable $x$ for each event in the signal
sample and constructs an histogram $\boldsymbol{n} = (n_1, \dots, n_N)$ out of
these measurements, the expectation value of $n_i$ can be written as:
\begin{equation}
  \label{eq:103}
  E[n_i] = \mu s_i + b_i
\end{equation}
where $s_i$ and $b_i$ are the mean number of signal and background entries in
the $i$-th bin and $\mu$ is the \emph{signal strength} parameter of the signal
process with $\mu = 0$ indicating the background only hypothesis and $\mu = 1$
the nominal signal one. The parameters of the physical model, unknown properties
of the response of the detector as well as other parameters that are not of
interest for the physics search are called \emph{nuisance parameters} and
indicated with $\boldsymbol{\theta}$. These can be evaluated constructing
histograms of some kinematic variable of interest in a control region where
mostly background events are expected. Indicating as
$\boldsymbol{m} = (m_1, \dots, m_M)$ one of such histograms, the expectation
value of the $i$-th bin, $m_i$, is given by:
\begin{equation}
  \label{eq:104}
  E[m_i] = u_i(\boldsymbol{\theta})
\end{equation}
where $u_i$ are calculable quantities that depend on the nuisance parameters
$\boldsymbol{\theta}$. With these definitions, the \emph{likelihood} function
can be written as the product of Poisson probabilities for the different bins:
\begin{equation}
  \label{eq:105}
  L(\mu, \boldsymbol{\theta}) = \prod_{j = 1}^N \frac{(\mu s_j +
    b_j)^{n_j}}{n_j!} e^{-(\mu s_j + b_j)} \prod_{k = 1}^M
  \frac{u_k^{m_k}}{m_k!} e^{-u_k}
\end{equation}

The p-value is determined at \gls{lhc} using the \emph{test
  statistics}~\cite{StatProcedure}:
\begin{equation}
  \label{eq:106}
  t_\mu = - 2 \ln \lambda(\mu)
\end{equation}
where $\lambda(\mu)$ is the \emph{profile likelihood ratio} defined as:
\begin{equation}
  \label{eq:107}
  \lambda(\mu) = \frac{L(\mu, \hat{\hat{\boldsymbol{\theta}}})}{L(\hat{\mu},
    \hat{\boldsymbol{\theta}})}.
\end{equation}
Here $\hat{\mu}$ and $\hat{\boldsymbol{\theta}}$ are the maximum likelihood
estimators, $L(\hat{\mu}, \hat{\boldsymbol{\theta}})$ is the maximized likelihood
function and $\hat{\hat{\boldsymbol{\theta}}}$ is the value of the parameters
$\boldsymbol{\theta}$ that maximizes the likelihood for a fixed value of the
signal strength $\mu$. It can be seen that $0 \leq \lambda(\mu) \leq 1$ with a
value close to one corresponding to a good agreement between the data and $\mu$,
this means that high values of the test statistic in \cref{eq:106} correspond to
bigger incompatibility between data and the hypothesized $\mu$.

For the discovery of new physics signals, the test statistics in \cref{eq:106} is
modified to:
\begin{equation}
  \label{eq:108}
  q_0 =
  \left\{
    \begin{array}{lr}
      - 2 \ln \lambda(0)  & \hat{\mu} \geq 0 \\
      0 & \hat{\mu} < 0
    \end{array}
    \right.
\end{equation}
where in the model it is assumed that $\mu \geq 0$ and the likelihood ratio in
\cref{eq:107} is evaluated for $\mu = 0$ (the background only hypothesis). The
p-value is given in this case by:
\begin{equation}
  \label{eq:109}
  p_0 = \int_{q_0, \mathrm{obs}}^\infty f(q_0 | 0)~\ud q_0
\end{equation}
where $f(q_0 | 0)$ is the \gls{pdf} of $q_0$ under the background only
hypothesis. Excluding the $\mu = 0$ hypothesis corresponds to the discovery of a
new signal. In the large sample limit it is possible to use the result from
Wald~\cite{WaldApproximation}:
\begin{equation}
  \label{eq:110}
  - 2 \ln \lambda(\mu) = \frac{(\mu - \hat{\mu})^2}{\sigma^2} + \mathcal{O}(1/\sqrt{N}),
\end{equation}
where $N$ is the sample size and $\hat{\mu}$ follows a Gaussian distribution
with mean $\mu'$ and standard deviation $\sigma$, to approximate the test
statistics in \cref{eq:108} to:
\begin{equation}
  \label{eq:111}
  q_0 = \left\{
    \begin{array}{lr}
      \hat{\mu}^2/\sigma & \hat{\mu} \geq 0 \\
      0 & \hat{\mu} < 0
    \end{array}
    \right.
\end{equation}
where $\hat{\mu}$ is distributed with a Gaussian shape with mean $\mu'$ and
standard deviation $\sigma$. From this result it is possible to
show~\cite{StatProcedure} that the p-value for the background only hypothesis
($\mu = 0$)is given by:
\begin{equation}
  \label{eq:112}
  p_0 = 1 - F(q_0|0)
\end{equation}
where $F(q_0|0) = \Phi \left( \sqrt{q_0} - \mu' / \sigma \right)$ is the
cumulative distribution function of $f(q_0|0)$ and $\Phi$ is the cumulative
distribution function of a standard Gaussian with zero mean and unit standard
deviation.

In order to exclude a signal strength $\mu$ upward fluctuations of the data are
not regarded as less compatible with $\mu$ than the obtained data and for this
reason the test statistic is set to zero for $\hat{\mu} > \mu$ and the base test
statistic given in \cref{eq:106} is modified to:
\begin{equation}
  \label{eq:121}
  q_\mu =
  \left\{
    \begin{array}{lr}
      - 2 \ln \lambda(\mu)  & \hat{\mu} \leq \mu \\
      0 & \hat{\mu} > \mu
    \end{array}
    \right.
\end{equation}
where $\lambda(\mu)$ is the likelihood ration of \cref{eq:107}. The p-value can
be calculated with:
\begin{equation}
  \label{eq:50}
  p_\mu = \int_{q_\mu, \mathrm{obs}}^\infty f(q_\mu | \mu)~\ud q_\mu
\end{equation}
where $f(q_\mu | \mu)$ is the \gls{pdf} of $q_\mu$ under the hypothesis
$\mu$. Using the result from Wald~\cite{WaldApproximation}, \eqref{eq:121} can
be approximated to a chi-square distribution with one degree of
freedom~\cite{StatProcedure}:
\begin{equation}
  \label{eq:122}
  q_\mu =
  \left\{
    \begin{array}{lr}
      \frac{(\mu - \hat{\mu})^2}{\sigma^2}  & \hat{\mu} < \mu \\
      0 & \hat{\mu} > \mu
    \end{array}
    \right.
\end{equation}
where $\hat{\mu}$ follows a normal distribution centered around $\mu'$ and
standard deviation $\sigma$. In this approximation the p-value for the $\mu$
hypothesis is~\cite{StatProcedure}:
\begin{equation}
  \label{eq:123}
  p_\mu = 1 - F(q_\mu|\mu) = 1 - \Phi(\sqrt{q_\mu})
\end{equation}
where $F(q_\mu|\mu)$ is the cumulative distribution function of $f(q_\mu|\mu)$
and $\Phi$ is the cumulative distribution function of a Gaussian centered around
zero and with unit variance. If the calculated p-value $p_\mu$ is below some
threshold $\alpha$:
\begin{equation}
  \label{eq:124}
  p_\mu < \alpha
\end{equation}
then the hypothesis $\mu$ is said to be excluded with a $1 - \alpha$
\gls{cl}. As mentioned earlier, as a convention, for exclusion purposes, the
value of $\alpha$ is taken to be 0.05 and the excluded $\mu$ is thus quoted with
a 95\%~\gls{cl}. The upper limit on $\mu$ is the largest $\mu$ that satisfies
$p_\mu \leq \alpha$. It can be calculated analytically by setting $p_\mu =
\alpha$ and solving for $\mu$ obtaining~\cite{StatProcedure}:
\begin{equation}
  \label{eq:125}
  \mu_{\mathrm{up}} = \hat{\mu} + \sigma \Phi^{-1}(1 - \alpha).
\end{equation}
Since $\sigma$ depends on $\mu$~\cite{StatProcedure}, it is common to calculate
the upper limit numerically as the value of $\mu$ for which $p_\mu = \alpha$.
%%% Local Variables:
%%% mode: latex
%%% TeX-master: "../search_for_DM_LED_with_ATLAS"
%%% End:
