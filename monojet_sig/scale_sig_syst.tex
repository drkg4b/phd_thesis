The cross section can be separated in a short distance partonic component coming
from the hard scatter processes and long distance terms describing the original
hadron~\cite{FactorizationScale}. The \emph{factorization scale} is the scale
that separates the long distance effects from the short distance
ones~\cite{PerturbativeQCDHandbook}. The \emph{renormalization scale} is the
cut-off energy after which ultraviolet divergences are expected to brake the
renormalizability of the theory and thus make the perturbative expansion no more
reliable. These two scales are varied simultaneously by factor 2 and 1/2, in
truth-level \gls{mc} samples. The final uncertainty is the average between the
up and down variations. The scale variation affects mainly the cross sections,
which vary between 23\% and 36\%.

The uncertainty on the SUSY cross sections was derived by the squark-gluino team
and derived to be 13\% on the sample normalisation. This uncertainty is added in
quadrature to the uncertainty on $\sigma$ listed in
% \begin{table}[!h]
%   \centering
%   \resizebox{\linewidth}{!}{\begin{tabular}{lccccccc}
%     \toprule
%     \multicolumn{8}{c}{Renormalization Scales} \\
%     \midrule \midrule
%     $\met$ [GeV] & 250-300 & 300-350 & 350-400 & 400-500 & 500-600 & 600-700 &
%                    > 700 \\
%     $\Delta A$ & 11 & 18 & 11 & 8 & 7 & 9 & 8 \\
%     \bottomrule
%   \end{tabular}}
%   \caption{Tune, initial and final state radiation and MADGRAPH level systematic
%     uncertainties in \% on the acceptance of the signal region $\met$ bins used
%     in the analsysis. The final value is a common envelope valid for all the
%     compressed \gls{susy} signal models.}
%   \label{tab:susy_scales}
% \end{table}
%%% Local Variables:
%%% mode: latex
%%% TeX-master: "../search_for_DM_LED_with_ATLAS"
%%% End:
