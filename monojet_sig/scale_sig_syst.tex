The cross section can be separated in a short distance partonic component coming
from the hard scatter processes and long distance terms describing the original
hadron~\cite{FactorizationScale}. The \emph{factorization scale} is the scale
that separates the long distance effects from the short distance
ones~\cite{PerturbativeQCDHandbook}. The \emph{renormalization scale} is the
cut-off energy after which ultraviolet divergences are expected to brake the
renormalizability of the theory and thus make the perturbative expansion no more
reliable.

The uncertainty on these two scales affects the cross section and is obtained by
the squark-gluino team~\cite{SquarkGluinoTeam} by varying them simultaneously by
a factor 2 and 1/2 in truth-level \gls{mc} samples, and derived to be 13\%. In
the monojet analysis this is a conservative approach due to the range and higher
value of masses investigated by the squark-gluino team. This uncertainty is
added in quadrature to the cross section uncertainty on the \glspl{pdf}
collected in \ref{tab:susy_pdfsysts}.
% The uncertainty on these two scales is obtained by varying them simultaneously
% by a factor 2 and 1/2 in truth-level \gls{mc} samples. This procedure is
% computationally and time consuming, for this reason for the \gls{susy}
% compressed

% These two scales are varied simultaneously by factor 2 and 1/2, in
% truth-level \gls{mc} samples. The final uncertainty is the average between the
% up and down variations. The scale variation affects mainly the cross sections,
% which vary between 23\% and 36\%.

% The uncertainty on the SUSY cross sections was derived by the squark-gluino team
% and derived to be 13\% on the sample normalisation. This uncertainty is added in
% quadrature to the uncertainty on $\sigma$ listed in
%%% Local Variables:
%%% mode: latex
%%% TeX-master: "../search_for_DM_LED_with_ATLAS"
%%% End:
