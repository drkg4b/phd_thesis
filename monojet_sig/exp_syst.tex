Experimental systematic uncertainties can be divided in the following broad
categories:
\begin{itemize}
\item \textbf{Jet/$\met$}: This category includes uncertainties on the $\met$
  and jet energy scale and resolution, jet quality, pile-up estimation and
  $\pt$ measurement.
\item \textbf{Leptons}: This category includes uncertainties related to the
  identification, simulation, reconstruction efficiencies and energy and
  momentum resolution of electrons and muons.
\item \textbf{Luminosity}: Is the uncertainty on the integrated luminosity.
\end{itemize}
A $\pm 5\%$ uncertainty is assigned to the \emph{integrated luminosity}. Since
the efficiency plateau of the HLT\_xe70 is reached below 250~GeV $\met$ values,
no uncertainty is assigned to the trigger. Uncertainties in the jet and $\met$
\emph{energy scale} and \emph{resolution} contribute to the total background
with a variation of the number of predicted events of $\pm 0.5\%$ and
$\pm 1.6\%$ in the IM1 and IM7 respectively. The jet quality requirement,
pile-up estimation and correction to the measured $\met$ and jet $\pt$ results
in an uncertainty in the estimation of the total background of $\pm 0.2\%$ in
the IM1 and $\pm 0.9 \%$ in the IM7 control region. Lepton \emph{identification}
simulation and \emph{reconstruction efficiency} lead to a $\pm 0.1\%$
uncertainty in the IM1 and IM7 regions while energy and momentum
\emph{resolution} results in a $\pm 1.4\%$ variation in the IM1 selection and a
$\pm 2.6\%$ uncertainty in the IM7 signal region. A flat $\pm 100\%$ uncertainty
is assigned to \emph{multi-jet} and \emph{NCB} that translates to a $\pm 0.2\%$
variation in the total background in the IM1 signal region.
%%% Local Variables:
%%% mode: latex
%%% TeX-master: "../search_for_DM_LED_with_ATLAS"
%%% End:
