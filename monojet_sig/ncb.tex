Non collision background is a term used to refer to BG processes coming from
cosmic particles, beam induced muons resulting from proton-gas inelastic
interaction or beam halo protons intercepting the LHC collimators and detector
noise. The characteristic signature of NCB is that of a jet recoiling against
invisible energy thus resembling the monojet final state signature. The jet
quality selection criteria mentioned in \cref{sec:event-selection} manage to
reduce the rate of jet coming from cosmic muons to a negligible amount compared
to the rate of data in the SR thus the main source of NCB is \gls{bib}. In order
to estimate the BIB contribution the two-sided no-time
method~\cite{BeamInducedBackground} was used. This method tries to match a
calorimeter energy cluster with a muon segment in both A and C side of the
detector. This topology corresponds to a particle moving parallel to the beam
line but several meters away from the beam axis and thus presumably arising from
beam background. An estimate of the number of NCB events in the SR is obtained
by correcting the number of events tagged as BIB inside the signal region for
the efficiency of the method. The efficiency of the tagger is estimated in a
sample of events in the SR failing the jet tight cleaning criteria that is
dominated by BIB jets.

The NCB contribution in the IM1 results in $112 \pm 23$ events and only
$19 \pm 9$ events in the EM3, this constitutes about 0.5\% of the total
background in these regions. For $\met > 500~$GeV there is no NCB contribution
in the signal region.
%%% Local Variables:
%%% mode: latex
%%% TeX-master: "../search_for_DM_LED_with_ATLAS"
%%% End:
