There are two possible cases that results in an energy imbalance in the
detector, the first one occurs in beyond Standard Model physics, that involves
the presence of particles that interact weakly or not at all with normal
matter. These particles are not detected thus leave an energy imbalance in the
detector. In the second case, the decay products in the final state involve
neutrinos that are not detectable by \gls{atlas}\@. To better understand the
first category of events, one considers \cref{fig:susy_standard} that shows the
decay topology of squark pair production with a neutralino and two jets in the
final state. Using the two body decay energy and momentum relations~\cite{PDG}:
\begin{equation}
  \label{eq:95}
  E_q = \frac{M_{\, \tilde{q}}^2 - m_{\, \tilde{\chi}_{\, 1}^{\, 0}}^2 + m_q^2}{2
    M_{\, \tilde{q}}},
\end{equation}
\begin{equation}
  \label{eq:96}
  |\vec{p}_q| = |\vec{p}_{\, \tilde{\chi}_{\, 1}^{\, 0}}| = \frac{\left[ \left(
        M_{\, \tilde{q}}^2 - (m_q + m_{\, \tilde{\chi}_{\, 1}^{\, 0}})^2
      \right) \left( M_{\, \tilde{q}}^2 - (m_q - m_{\, \tilde{\chi}_{\, 1}^{\,
            0}})^2 \right) \right]^{1/2}}{2 M_{\, \tilde{q}}}
\end{equation}
where $M_{\, \tilde{q}}$ is the squark center of mass energy,
$m_{\, \tilde{\chi}_{\, 1}^{\, 0}}$ is the neutralino mass and $m_q$ is the
quark mass. Neglecting the quark mass ($m_q = 0$) we get that:
\begin{equation}
  \label{eq:97}
  E_q = \frac{M_{\, \tilde{q}}^2 - m_{\, \tilde{\chi}_{\, 1}^{\, 0}}^2}{2 M_{\,
      \tilde{q}}},
\end{equation}
\begin{equation}
  \label{eq:98}
  |\vec{p}_q| = |\vec{p}_{\, \tilde{\chi}_{\, 1}^{\, 0}}| = \frac{M_{\,
      \tilde{q}}^2 - m_{\, \tilde{\chi}_{\, 1}^{\, 0}}^2}{2 M_{\, \tilde{q}}}.
\end{equation}
The quark will hadronize and the corresponding hadron will shower in the
calorimeter resulting in a jet that can be detected by the \gls{atlas}
detector. If for example the mass of the squark is $M_{\, \tilde{q}} = 450$~GeV
and the mass of the neutralino is $m_{\, \tilde{\chi}_{\, 1}^{\, 0}} = 445$~GeV
then it can be seen from \cref{eq:98} that the quark and neutralino momenta are
given by $|\vec{p}_q|~=~|\vec{p}_{\, \tilde{\chi}_{\, 1}^{\,
    0}}|~\simeq~5$~GeV. Since the neutralino escape detection, this results in
low $\met$ thus when the mass of the neutralino approaches the mass of the quark
this results in a low energy jet and $\met$. Since the missing energy resolution
of the \gls{atlas} detector is approximately 15~GeV in data as can be seen from
\cref{fig:met_resolution}, such events can be difficult to trigger on and cannot
be extracted from the multijet background using missing transverse energy. This
means that there is no sensitivity to \gls{susy} models with compressed mass
spectra (when the mass difference between the particles is small). This problem
applies to several \gls{susy} production channels.
\begin{figure}[!h]
  \centering
  \begin{subfigure}[t]{.48\linewidth}
    \includegraphics[width=\linewidth]{susy_standard}
    \caption{Event without initial state radiation~\cite{SUSYPub}.}
    \label{fig:susy_standard}
  \end{subfigure} \quad
  \begin{subfigure}[t]{.48\linewidth}
    \includegraphics[width=\linewidth]{compressed}
    \caption{Event with initial state radiation~\cite{ExotPub}.}
    \label{fig:susy_compressed}
  \end{subfigure}
  \caption{Event topology of squark pair production resulting in a neutralinos
    with two jets final state with (\cref{fig:susy_compressed}) and without
    (\cref{fig:susy_standard}) initial state radiation.}
  \label{fig:motivation}
\end{figure}

\cref{fig:susy_exclusion} illustrates this effect for the search for squark pair
production in the case of the squark decaying directly to a quark and a
neutralino through the mechanism illustrated in \cref{fig:susy_standard}. This
search uses a classical multijet + $\met$ analysis, it can be seen that there is
no sensitivity close to the diagonal (dashed line) in the region
$400~<~M_{\, \tilde{q}}~<~600$~GeV.

If an initial state radiation jet is present in the event, as depicted in
\cref{fig:susy_compressed}, the squark-squark system gets boosted in the
opposite direction thus increasing the momentum of the decay products and the
missing energy leading to a signature of a high $\pt$ jet on one side and
additional jets and $\met$ on the other side of the event.

Events with an energetic jet $\pt$ and large $\met$ in the final state
constitute a clean signature for new physics searches at hadron colliders.
Other \gls{bsm} signals that can be studied with this experimental signature
include the production of \glspl{wimp}, the \gls{add} model for large extra
dimensions and \gls{susy}.
\begin{figure}[!h]
  \centering
  \includegraphics[width=.81\textwidth]{met_resolution}
  \caption{Distribution of the $x$ and $y$ components of the \gls{tst} $\met$
    resolution as a function of the number of primary vertexes in
    $\zmumuplusjets$ events~\cite{METResolution}.}
  \label{fig:met_resolution}
\end{figure}
\begin{figure}[!h]
  \centering
  \includegraphics[width=.81\linewidth]{susy}
  \caption{Exclusion limits for direct production of squark
    pairs where the squark decays into a quark and a neutralino. The $x$-axis
    represents the mass of the squark and the $y$-axis represents the mass of
    the lightest neutralino. The black stars represent a benchmark model as
    explained in more details in Ref.~~\cite{SUSYPub}.}
  \label{fig:susy_exclusion}
\end{figure}
%%% Local Variables:
%%% mode: latex
%%% TeX-master: "../search_for_DM_LED_with_ATLAS"
%%% End:
