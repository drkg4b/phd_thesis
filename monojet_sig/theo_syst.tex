Since the normalization factor for the $Z(\rightarrow \nu \bar{\nu})$ is taken
from the single muon region, following the Run~1 analysis
strategy~\cite{RunIPaper}, a conservative flat $\pm 3\%$ uncertainty
corresponding to the maximum variation in several distributions for different
$\met$ regions is assumed. This number includes MC modeling differences between
the $\met$ in the $Z(\rightarrow \nu \bar{\nu})$ and in
$W(\rightarrow \mu \nu)$, parton shower parameters, PDFs related uncertainties
and the lepton reconstruction acceptance. In practice we look for small
differences in the $\wmunuplusjets$ and $\znunuplusjets$ $\met$ distribution and
since our background calculation builds on the fact that these quantities should
be the same, any difference needs to be taken into account as a
systematic. These differences affect the ratio in \cref{eq:84}.

In addition electroweak correction that takes into account the theoretical
differences between the production rate of the $W$ and $Z$ bosons as described
in Ref.~\cite{EWCorrections} are added in quadrature leading to a $\pm 3.5\%$
and $\pm 6\%$ uncertainty in the IM1 and IM7 region respectively.

Processes related to the \emph{top quark} production gets their uncertainties
from the $t \bar{t}$ and single--top production cross section, levels of initial
and final state radiation, the parameters used to generate the parton shower MC
samples and the normalization factor. This introduce a $\pm 2.7\%$ and
$\pm 3.3 \%$ variation in the background estimation in the IM1 and IM7 signal
region respectively. Uncertainties coming from \emph{diboson} processes are
estimated using different MC generators and account for an uncertainty between
$\pm 0.05\%$ and $\pm 0.4$ on the number of events in the signal region. A flat
$\pm 100\%$ uncertainty is assigned to \emph{multi-jet} and \emph{NCB} that
translates to a $\pm 0.2\%$ variation in the total background in the IM1 signal
region.
%%% Local Variables:
%%% mode: latex
%%% TeX-master: "../search_for_DM_LED_with_ATLAS"
%%% End:
