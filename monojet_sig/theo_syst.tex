The normalization factor for the $\znunuplusjets$ is constrained primarily from
the single muon control region. Following the run 1 analysis
strategy~\cite{RunIPaper}, a conservative 3\% uncertainty is used. It was
obtained by taking the maximum variation of the number of $\znunuplusjets$
events in bins of the leading jet $\pt$ and $\met$ predicted by
\textsc{sherpa}~\cite{SHERPA} and \textsc{alpgen}~\cite{ALPGEN}, with two
different parton showering algorithms. This 3\% uncertainty was therefore a
simple way to estimate generator uncertainty and parton shower uncertainty.

In addition an electroweak correction that takes into account the theoretical
differences between the $W$ and $Z$ bosons $\pt$ as described in
Ref.~\cite{EWCorrections} is added in quadrature to the flat 3\% introduced
earlier leading to a $\pm 3.5\%$ and $\pm 6\%$ uncertainty in the IM1 and IM7
region respectively.

\emph{Top quark} production processes get their uncertainties from the
$t \bar{t}$ and single-top production cross section, levels of initial and final
state radiation, the model used to generate the parton shower. This introduces a
$\pm 2.7\%$ and $\pm 3.3 \%$ variation in the top background estimation in the
IM1 and IM7 signal region respectively.

Uncertainties coming from \emph{diboson} processes are estimated using different
\gls{mc} generators and account for an uncertainty between $\pm 0.05\%$ and
$\pm 0.4$ on the number of events in the signal region.
%%% Local Variables:
%%% mode: latex
%%% TeX-master: "../search_for_DM_LED_with_ATLAS"
%%% End:
